LaTeX_Input/Appendix/json_inputs.tex:645:%        ECI\_ comp          & \colorbox{yellow}{TODO}                                                              & N/A & Default = .FALSE. \\ \hline
LaTeX_Input/Appendix/namelist_inputs.tex:649:%        ECI\_ comp          & \colorbox{yellow}{TODO}                                                              & N/A & Default = .FALSE. \\ \hline
LaTeX_Input/Appendix/Numerical_Solution_Plenum_Energy_Equations.tex:95:	\item \colorbox{yellow}{Cladding Exterior Node}

LaTeX_Input/External_interfaces/FAST_EPIC_interface.tex:4:transport and depletion to calculate the following parameters:\colorbox{yellow}{insert ref for EPIC}
LaTeX_Input/External_interfaces/FAST_EPIC_interface.tex:108:        \texttt{fissium} (A)               & Array of isotopes and their \colorbox{yellow}{xxx} & \colorbox{yellow}{unknown}           & Yes       \\
LaTeX_Input/External_interfaces/FAST_TRACE_interface.tex:3:TRACE is the NRC's code for system wide thermal hydraulic analysis.\colorbox{yellow}{insert ref for

LaTeX_Input/Section-1-input/subsection-1.tex:27:These properties are documented elsewhere (\colorbox{yellow}{Luscher and Geelhood, 2014}). The only
LaTeX_Input/Section-1-input/subsection-1.tex:41:releases, starting with FRAPCON-3 v1.0 (\colorbox{yellow}{Berna et al., 1997}) in 1997 through
LaTeX_Input/Section-1-input/subsection-1.tex:42:FRAPCON-4.0 Patch 1 (\colorbox{yellow}{Geelhood et al., 2015}) in 2015. The FRAPTRAN-1 and
LaTeX_Input/Section-1-input/subsection-1.tex:56:(\colorbox{yellow}{Knuttilla, 2006}). This document is posted on the FRAPCON/FRAPTRAN code users'
LaTeX_Input/Section-1-input/subsection-3.tex:75:            VVER fuel and cladding models 							&					 NUREG/IA-0164 (\colorbox{yellow}{Shestopalov et al., 1999})			\\\hline
LaTeX_Input/Section-1-input/subsection-3.tex:76:            Cladding FEA model 										&					 VTT-R-11337-06 (\colorbox{yellow}{Knuttilla, 2006})			        \\\hline

LaTeX_Input/Section-2-input/subsection-11.tex:6:strain given by MATPRO (\colorbox{yellow}{\colorbox{yellow}{Hagrman et al. 1981}}). If the cladding
LaTeX_Input/Section-2-input/subsection-11.tex:9:\colorbox{yellow}{Hagrman (1981)} for the details of the BALON2 model. Once the instability strain
LaTeX_Input/Section-2-input/subsection-11.tex:78:(\colorbox{yellow}{Powers and Meyer, 1980}). These comparisons are shown in
LaTeX_Input/Section-2-input/subsection-12.tex:16:uniform plastic elongation from irradiated cladding (\colorbox{yellow}{Geelhood et al. 2008}) is
LaTeX_Input/Section-2-input/subsection-12.tex:76:the uniform elongation data are provided in \colorbox{yellow}{Geelhood et al. (2008)}. It is noted
LaTeX_Input/Section-2-input/subsection-12.tex:78:non-failure in RIA tests \colorbox{yellow}{(Geelhood and Luscher 2014b)}.
LaTeX_Input/Section-2-input/subsection-13.tex:31:experiments performed in the FLECHT facility (\colorbox{yellow}{Cadek et al. 1972}). A description
LaTeX_Input/Section-2-input/subsection-13.tex:32:of these models is presented in \colorbox{yellow}{Appendix D}.
LaTeX_Input/Section-2-input/subsection-1.tex:51:For the FRACAS-I (\colorbox{yellow}{Bohn et al. 1977}) mechanical model, the fuel temperature and
LaTeX_Input/Section-2-input/subsection-3.tex:28:\colorbox{yellow}{We should update Figure~\ref{fig:mat-node-placement}, really looks bad.}
LaTeX_Input/Section-2-input/subsection-3.tex:120:\ind{   \(r_{p}\)               =            Number of radial boundaries in the fuel (\colorbox{yellow}{See B.2})   }
LaTeX_Input/Section-2-input/subsection-3.tex:233:    \caption{A schematic of the impact of dishes and chamfers on void volume \colorbox{yellow}{We
LaTeX_Input/Section-2-input/subsection-4.tex:23:    \caption{Energy Flow in Plenum Model-Spring Model with Two Nodes \colorbox{yellow}{We should consider re-making}}
LaTeX_Input/Section-2-input/subsection-4.tex:29:    \caption{Energy Flow in Plenum Model -- Energy Exchange Mechanisms \colorbox{yellow}{We should consider re-making}}
LaTeX_Input/Section-2-input/subsection-4.tex:35:    \caption{Cladding Nodalization \colorbox{yellow}{We should consider re-making}}
LaTeX_Input/Section-2-input/subsection-4.tex:41:    \caption{Geometrical Relationship Between the Cladding and Spring \colorbox{yellow}{We should consider re-making}}
LaTeX_Input/Section-2-input/subsection-4.tex:47:    \caption{Flowchart of Plenum Temperature Calculation \colorbox{yellow}{We should consider re-making}}
LaTeX_Input/Section-2-input/subsection-4.tex:112:(\colorbox{yellow}{Crank and Nicolson 1974}) implicit finite difference form. This formulation
LaTeX_Input/Section-2-input/subsection-4.tex:116:are given in \colorbox{yellow}{Appendix E.}
LaTeX_Input/Section-2-input/subsection-4.tex:132:for these types of heat transfer are obtained from \colorbox{yellow}{Kreith (1964)} and
LaTeX_Input/Section-2-input/subsection-4.tex:133:\colorbox{yellow}{McAdams (1954)}.
LaTeX_Input/Section-2-input/subsection-4.tex:271:energy exchange equation for two gray bodies in thermal equilibrium (\colorbox{yellow}{Kreith~1964})
LaTeX_Input/Section-2-input/subsection-5.tex:12:(\colorbox{yellow}{Knutilla 2006}).
LaTeX_Input/Section-2-input/subsection-5.tex:75:substitutions (also called the method of successive elastic solutions) (\colorbox{yellow}{Mendelson
LaTeX_Input/Section-2-input/subsection-5.tex:136:    \caption{Typical Isothermal Stress-Strain Curve \colorbox{yellow}{Should consider updating}}
LaTeX_Input/Section-2-input/subsection-5.tex:292:number of technologically useful examples, is contained in \colorbox{yellow}{Knuutila (2006)}.
LaTeX_Input/Section-2-input/subsection-5.tex:333:    \caption{Schematic of the Method of Successive Elastic Solutions \colorbox{yellow}{Should Consider re-making}}
LaTeX_Input/Section-2-input/subsection-5.tex:397:\\ A model described by Limb\"{a}ck and Andersson (\colorbox{yellow}{Limb\"{a}ck and Andersson
LaTeX_Input/Section-2-input/subsection-5.tex:399:in FAST.  This model uses a thermal creep model described by \colorbox{yellow}{Matsuo (1987)} and an
LaTeX_Input/Section-2-input/subsection-5.tex:401:\colorbox{yellow}{Franklin et al. (1983)}. The Limb\"{a}ck model was further modified by PNNL to use
LaTeX_Input/Section-2-input/subsection-5.tex:405:accommodate this change based on comparisons to several data sets (\colorbox{yellow}{Franklin et al.
LaTeX_Input/Section-2-input/subsection-5.tex:469:(\colorbox{yellow}{Geelhood 2013}).  In FAST, the first term in Equation~\ref{eq:strain_rate} is
LaTeX_Input/Section-2-input/subsection-5.tex:477:(\colorbox{yellow}{Limb\"{a}ck and Andersson 1996}), with the exception of the ``A'' parameter and
LaTeX_Input/Section-2-input/subsection-5.tex:544:Zircaloy reduced by a factor of 0.8 is used (\colorbox{yellow}{Sabol et al. 1994}).
LaTeX_Input/Section-2-input/subsection-5.tex:1456:shown in \colorbox{yellow}{Geelhood et al. (2008)}.
LaTeX_Input/Section-2-input/subsection-5.tex:1508:\sinum{GWd/MTU} beyond \SInum{80}{GWd/MTU} \colorbox{yellow}{(Luscher and Geelhood 2014)}.
LaTeX_Input/Section-2-input/subsection-5.tex:1512:(\colorbox{yellow}{Mogensen 1985}) and was developed after ramp test results suggested gaseous
LaTeX_Input/Section-2-input/subsection-5.tex:1555:thermal expansion (\colorbox{yellow}{Lanning 1982}). It has long been concluded, based on
LaTeX_Input/Section-2-input/subsection-5.tex:1556:microscopic examination of fuel cross sections (\colorbox{yellow}{Galbraith 1973};
LaTeX_Input/Section-2-input/subsection-5.tex:1557:\colorbox{yellow}{Cunningham and Beyer 1984}), that fuel pellet cracking promotes an outward
LaTeX_Input/Section-2-input/subsection-5.tex:1559:and quickly reaches equilibrium. \colorbox{yellow}{Oguma (1983)} characterized this approach to
LaTeX_Input/Section-2-input/subsection-5.tex:1572:The best estimate pellet relocation model developed for GT2R2 (\colorbox{yellow}{Cunningham and
LaTeX_Input/Section-2-input/subsection-5.tex:1579:power for ramps to power at 0.1, 0.6, 4, and \SInum{5}{GWd/MTU} (\colorbox{yellow}{Thérache 2005}).
LaTeX_Input/Section-2-input/subsection-6.tex:300:(1982) (\colorbox{yellow}{Rausch and Panisko 1979}); ANS-5.4(2011); the modified Forsberg and Massih
LaTeX_Input/Section-2-input/subsection-6.tex:301:model (\colorbox{yellow}{Forsberg and Massih 1985}), modified at PNNL; and the FRAPFGR model
LaTeX_Input/Section-2-input/subsection-6.tex:303:from a sphere by \colorbox{yellow}{Booth (1957)} and are discussed below.
LaTeX_Input/Section-2-input/subsection-6.tex:327:The ANS-5.4 fission gas release model (\colorbox{yellow}{ANS 1982}) is incorporated as specified by
LaTeX_Input/Section-2-input/subsection-6.tex:329:has been recently approved as a standard (\colorbox{yellow}{ANS 2011}). The 1982 model is not
LaTeX_Input/Section-2-input/subsection-6.tex:333:ANS-5.4 standard (\colorbox{yellow}{ANS 2011}) is also available and is described below.
LaTeX_Input/Section-2-input/subsection-6.tex:366:The fission gas release model from ANS 5.4 2011 (\colorbox{yellow}{ANS 2011}) implements the model
LaTeX_Input/Section-2-input/subsection-6.tex:374:sixty days. It should be noted that Table 1 of (\colorbox{yellow}{ANS 2011}) does not contain all
LaTeX_Input/Section-2-input/subsection-6.tex:376:the complete nuclide list and associated physical parameters from (\colorbox{yellow}{Turbull and
LaTeX_Input/Section-2-input/subsection-6.tex:677:uses a grain growth model proposed by \colorbox{yellow}{Khoruzhii et al. (1999)} given by
LaTeX_Input/Section-2-input/subsection-6.tex:700:The size of the high burnup rim has been measured by optical microscopy (\colorbox{yellow}{Manzel
LaTeX_Input/Section-2-input/subsection-6.tex:728:(\colorbox{yellow}{Spino et al. 1996}; \colorbox{yellow}{Une et al.  2001}; and
LaTeX_Input/Section-2-input/subsection-6.tex:729:\colorbox{yellow}{Manzel and Walker 2000}). This model is given by
LaTeX_Input/Section-2-input/subsection-6.tex:920:CABRI and NSRR. (See data comparisons in \colorbox{yellow}{Geelhood and Luscher (2014b)}.)
LaTeX_Input/Section-2-input/subsection-6.tex:926:gas conductivity. The model proposed by \colorbox{yellow}{Booth (1957)} is used, given the following
LaTeX_Input/Section-2-input/subsection-6.tex:984:From the experimental data of \colorbox{yellow}{Ferrari (1963, 1964)}
LaTeX_Input/Section-2-input/subsection-7.tex:12:\colorbox{yellow}{Garzarolli et al. 1982}:
LaTeX_Input/Section-2-input/subsection-7.tex:46:from \colorbox{yellow}{Garzarolli et al. (1982)} is used. This solution has the form
LaTeX_Input/Section-2-input/subsection-7.tex:75:\renewcommand{\captiontext}{Enthalpy and Transition Thickness under PWR Conditions.  \colorbox{yellow}{(Garzarolli et al. 1982)}} 
LaTeX_Input/Section-2-input/subsection-7.tex:112:\colorbox{yellow}{Garzarolli et al. (1982)} is used. This solution has the form
LaTeX_Input/Section-2-input/subsection-7.tex:145:\renewcommand{\captiontext}{Hydrogen pickup fraction under PWR conditions. (\colorbox{yellow}{Geelhood and Beyer 2011})} 
LaTeX_Input/Section-2-input/subsection-7.tex:161:(\colorbox{yellow}{Geelhood and Beyer, 2008}). For Zircaloy-2 prior to 1998 (when the vendors did
LaTeX_Input/Section-2-input/subsection-7.tex:194:corrosion models, the Cathcart/Pawel (\colorbox{yellow}{Cathcart et al. 1977}) model and the
LaTeX_Input/Section-2-input/subsection-7.tex:195:Baker/Just (\colorbox{yellow}{Baker and Just 1962}) model, that are selected using the input
LaTeX_Input/Section-2-input/subsection-7.tex:197:on model selection is given in the input instructions shown in \colorbox{yellow}{Appendix A}.
LaTeX_Input/Section-2-input/subsection-7.tex:204:\colorbox{yellow}{Appendix G}. These models are described below.
LaTeX_Input/Section-2-input/subsection-9.tex:63:(\colorbox{yellow}{Wagner 1963}). The solution method accounts for temperature- and time-dependent
