\addtocontents{toc}{~\hfill\textbf{Page}\par}
\pagenumbering{roman}
%\setcounter{page}{1}
\chapter*{Abstract}
\addcontentsline{toc}{chapter}{Abstract}
\noindent
\underline{{\color{red}F}}uel \underline{{\color{red}A}}nalysis under \underline{{\color{red}S}}teady-state and
\underline{{\color{red}T}}ransients (FAST) is the US Nuclear Regulatory Commission (NRC)'s computer code
that calculates the steady-state and transient response of nuclear reactor fuel rods
during long-term in-reactor burnup, anticipated operational occurrences (AOOs), design basis
accidents (DBAs), and dry storage conditions. The code calculates the temperature, pressure, 
and deformation of a fuel rod as functions of time-dependent fuel rod power and coolant
boundary conditions. The phenomena modeled by the code include:

\begin{itemize}
    \item heat conduction through the fuel and other materials
    \item heat transfer from the cladding-to-coolant
    \item cladding elastic and plastic deformation, including creep
    \item fuel-cladding mechanical interaction
    \item fission gas release from the fuel
    \item rod internal pressure and void volume
    \item cladding oxidation
\end{itemize}

The code contains necessary material and coolant properties, as well as
clad-to-coolant heat-transfer correlations, for normal operation through postulated 
accidents for today's US-based light water reactor (LWR) fuel designs. FAST-1.0 also
contains preliminary materials and models for new LWR fuel concepts, such as accident
tolerant fuel (ATF), and non-LWR fuel concepts such as metallic fuels for sodium
fast reactors (SFRs). FAST has been developed for use on Windows and Linux 
operating systems.
\\
\\
This document describes FAST-1.0, which is the first official version of this code.
This document is one of a series of documents on FAST; the other documents detail
the material properties used by FAST as well as its integral assessment to experiments
and commercial data.
\cleardoublepage
