\chapter*{Foreward}
\addcontentsline{toc}{chapter}{Foreward}
\noindent
Computer codes related to fuel performance have played an important role in the work of the U.S.
Nuclear Regulatory Commission (NRC) since the agency's inception in 1975. Formal requirements for
fuel performance analysis appear in several of the agency's regulatory guides and regulations,
including those related to emergency core cooling system evaluation models, as set forth in Appendix
K to Title 10, Part 50, of the \emph{Code of Federal Regulations} (10 CFR Part 50), ``Domestic
Licensing of Production and Utilization Facilities.''
\\
\\
\noindent
This document describes the initial version of the NRC's fuel performance code, FAST (Fuel Analysis
under Steady-state and Transients) Version-1.0.  FAST is a merger of the NRC's previous steady-state
fuel performance code FRAPCON-4.0 and transient fuel performance code FRAPTRAN-2.0. This code
provides the ability to accurately calculate the long-term burnup response of a single light-water
reactor fuel rod, as well as various operational transients and hypothetical accidents.  Together,
this code accomplishes a key objective of the NRC's reactor safety research program. The FAST code
serves as an independent audit tool used in the NRC's review of industry fuel performance codes and
industry analyses that demonstrate a given fuel design application meeting specified acceptable
design limits in U.S. NRC Standard Review Plan (SRP) Section 4.2 (\cite{ref:USNRC2007}).
\\
\\
This version of FAST is built off the foundation of FRAPCON-4.0 with the addition of the transient
conduction solution, new clad-to-coolant heat transfer models, a more detailed coolant enthalpy rise
model, high temperature cladding oxidation, new cladding deformation (ballooning) and failure
models, transient internal gas flow, an updated fission gas release model and new material
properties to scope out new fuel and cladding materials under consideration in the US nuclear
industry.
