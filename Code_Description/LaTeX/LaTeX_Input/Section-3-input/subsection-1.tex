\section{Heat Conduction Equation} \label{section:heat_cond_eq}
The numerical solution of the heat conduction Equation \ref{eq:one_d_heat_conduction_radial_tr}
requires solving a set of tri-diagonal equations. This set of equations is shown as follows:
\begin{equation}
    \label{eq:heat_cond_eq_tridiagonal}
    \begin{bmatrix}
        b_{1}   &   c_{1}   &   0       &  0        &   \dots     &   0         \\
        a_{2}   &   b_{2}   &   c_{2}   &  0        &   \dots     &   0         \\
        0       &   a_{3}   &   b_{3}   &  c_{3}    &   \dots     &   0         \\
        \vdots  &   \hdots  &   \ddots  &  \ddots   &   \ddots    &   \vdots    \\
        \vdots  &   \hdots  &   0       &  a_{N-1}  &   b_{N-1}   &   c_{N-1}   \\
        0       &   \hdots  &   \vdots  &   0       &   a_{N}     &     b_{N}   \\  
    \end{bmatrix}
    \begin{bmatrix}
        T_{1}^{m+1}         \\
        T_{2}^{m+1}         \\
        T_{3}^{m+1}         \\
        \vdots              \\
        \vdots              \\
        T_{N-1}^{m+1}       \\
        T_{N}^{m+1}         \\
    \end{bmatrix}
    =
    \begin{bmatrix}
        d_{1}               \\
        d_{2}               \\
        d_{3}               \\
        \vdots              \\
        \vdots              \\
        d_{N-1}             \\
        d_{N}               \\
    \end{bmatrix}
\end{equation}

Where,
\ind{	$a_{n}$, $\ b_{n}$, $c_{n}$, $d_{n}$    =  Terms of the heat conduction equation in finite difference form at the $n-th$ mesh point}	
\ind{	$T_{N}^{m + 1}$                               =  Temperature at $n$-th mesh point at time step $m$+1}	
\ind{	$N$                                           =  Number of mesh point at outer surface}	

The mesh point temperatures are solved by the Gaussian elimination method.

\begin{equation}
    \label{eq:heat_Tn_Gaussian}
    T_{N}^{m + 1} = \frac{d_{n} - a_{n}F_{n - 1}}{b_{n} - a_{n}E_{n - 1}}
\end{equation}

\begin{equation}
    \label{eq:heat_Tj_Gaussian}
    T_{j}^{m + 1} = - E_{j}T_{j + 1}^{m + 1} + F_{j}
\end{equation}

Where,
\begin{subequations}
    \begin{equation}
        E_{1} = \frac{c_{1}}{b_{1}}  
        \label{eq:heat-Gaussian-E}
    \end{equation}
    \begin{equation}
        F_{1} = \frac{d_{1}}{b_{1}} 
        \label{eq:heat-Gaussin-F}
    \end{equation}
\end{subequations}

And,

\begin{subequations}
    \begin{equation}
        E_{j} = \frac{c_{j}}{b_{j} - a_{j}E_{j - 1}} \text{\hspace{8pt}for $j=1,2,3\ldots, n-1$}
    \end{equation}
    \begin{equation}
        F_{j} = \frac{d_{j} - a_{j}F_{j - 1}}{b_{j} - a_{j}E_{j - 1}} \text{\hspace{8pt}for $j=1,2,3\ldots, n-1$} 
    \end{equation}
    \label{eq:heat-Gaussian-E-F-j}
\end{subequations}

The coefficients \(a_{n}\), \(b_{n}\), and \(d_{n}\) in Equation~\ref{eq:heat_Tn_Gaussian} are derived from the energy balance equation
for the half mesh interval bordering the outside surface. The continuous
form of the energy balance equation for this half mesh interval is:

\begin{equation}
    \label{eq:heat_continuous_form_energy_balance}
    \rho C_{p}\mathrm{\Delta}V\frac{\partial T}{\partial t} = - A_{n - \frac{1}{2}}K\left. \ \frac{\partial T}{\partial t} \right|_{r = r_{n} - \frac{\mathrm{\Delta}r}{2}}^{- \theta A_{n} + q\mathrm{\Delta}V}
\end{equation}

Where all the terms in Equation~\ref{eq:heat_continuous_form_energy_balance} are defined below.

The finite difference form of Equation~\ref{eq:heat_continuous_form_energy_balance} is
\begin{equation}  
    \label{eq:heat_fd_form}
    \begin{aligned}
        \underbrace{\frac{-0.5 A_{n-\frac{1}{2}} K}{\Delta r}}_{a_{n}} T^{m+1}_{n} + &
    \left[\underbrace{\frac{\rho C_{p} \Delta V}{\Delta t} + \frac{0.5A_{n-\frac{1}{2}} K}{\Delta r}}_{b_{n}} T^{m+1}_{n}\right]       \\
        & \underbrace{\frac{\rho C_{p} \Delta V}{\Delta r} T^{m}_{n} \frac{-0.5A_{n-\frac{1}{2}} K}{\Delta r} \left(T_{n}^{m} - T_{n-1}^{m}\right)-0.5A_{n}\left(\theta^{m} + \theta^{m+1}\right)+q^{m+\frac{1}{2}} \Delta V}_{d_{n}}
    \end{aligned}
\end{equation}  

The complete expressions for coefficients $a_{n}$, $b_{n}$, and $d_{n}$ are then:
\begin{subequations}
    \begin{equation}
        a_{n} = \frac{- 0.5A_{n - \frac{1}{2}}K}{\mathrm{\Delta}r}
    \end{equation}
    \begin{equation}
        b_{n} = \frac{\rho C_{p}\mathrm{\Delta}V}{\mathrm{\Delta}t} + \frac{0.5A_{n - \frac{1}{2}}K}{\mathrm{\Delta}r}
    \end{equation}
    \begin{equation}
        d_{n} = \frac{\rho C_{p}\mathrm{\Delta}V}{\mathrm{\Delta}r}T_{n}^{m}\frac{- 0.5A_{n - \frac{1}{2}}K}{\mathrm{\Delta}r}\left( T_{n}^{m} - T_{n - 1}^{m} \right) - 0.5A_{n}\left( \theta^{m} + \theta^{m + 1} \right) + q^{m + \frac{1}{2}}\mathrm{\Delta}V
    \end{equation}
    \begin{equation}
        A_{n - \frac{1}{2}} = 2\pi\left( r_{n} - \frac{\mathrm{\Delta}r}{2} \right)
    \end{equation}
    \begin{equation}
        A_{n} = 2\pi r_{n}
    \end{equation}
    \begin{equation}
        V = \pi\left( r_{n}\mathrm{\Delta}r - \frac{\mathrm{\Delta}r^{2}}{4} \right)
    \end{equation}
\end{subequations}

Where,
\ind{	$K$                     =  Thermal conductivity of material in half mesh interval bordering the surface}		
\ind{	$C_{p}$                 =  Specific heat of material in half mesh interval bordering the surface}		
\ind{	$\rho$                  =  Density of material in half mesh interval bordering the surface}		
\ind{	$r_{n}$                 =  Radius to outside surface}		
\ind{	$\mathrm{\Delta}r$      =  Width of mesh interval bordering outside surface}		
\ind{	$\mathrm{\Delta}t$      =  Time step}		
\ind{	$\theta^{m}$            =  Surface heat flux at $m$-th time step}		
\ind{	$T_{n}^{m}$             =  Surface temperature at $m$-th time step}		
\ind{	$q^{m - \frac{1}{2}}$   =  Heat generation rate in half mesh interval bordering outside surface (heat generation caused by cladding oxidation)}  	

Because the coefficients $a_{n}$, $b_{n}$, $d_{n}$, $E_{n - 1}$ and $F_{n - 1}$ in
Equation~\ref{eq:heat_Tn_Gaussian} do not contain temperature, the equation can be written as
follows:

\begin{equation}
    \label{eq:heat_A1Tn_B1_theta}
    A_{q}T_{n}^{m + 1} + B_{q} = \theta^{m + 1}
\end{equation}

Where,

\begin{equation}
    \label{eq:heat_A1}
    A_{1} = \frac{- \left( b_{n} - a_{n}E_{n - 1} \right)}{0.5A_{n}}
\end{equation}

\begin{equation}
    \label{eq:heat_B1}
    B_{1} = - \left\lbrack \frac{0.5\theta^{m}A_{n} + a_{n}F_{n - 1} - \frac{\rho C_{p}}{\mathrm{\Delta}t}\mathrm{\Delta}VT_{n}^{m} - a_{n}\left( T_{n}^{m} - T_{n - 1}^{m} \right) - q^{m - \frac{1}{2}}\mathrm{\Delta}V}{0.5A_{n}} \right\rbrack
\end{equation}

As shown in Equation~\ref{eq:heat_Tn_Gaussian}, the coefficients $E_{n - 1}$ and $F_{n - 1}$ are
evaluated by forward reduction of Equation~\ref{eq:heat_cond_eq_tridiagonal}.  Therefore,
Equation~\ref{eq:heat_A1Tn_B1_theta} contains only $T_{n}^{m + 1}$ and $\theta^{m + 1}$ as unknown
quantities.

Empirically derived clad-to-coolant heat transfer correlations are available from which the surface
heat flux due to convection can be calculated in terms of surface temperature, geometric parameters,
and flow conditions. Also, the equation for radiation heat transfer from a surface to surrounding
water is known. Thus, the total surface heat flux can be expressed by the equation
\begin{equation}
    \label{eq:heat_theta_m1}
    \theta^{m + 1} = f_{i}\left( C,G,T_{n}^{m + 1} \right) + \sigma F_{A}F_{E}\left\lbrack {(T_{n}^{m + 1})}^{4} - {T_{w}}^{4} \right\rbrack
\end{equation}

Where,
\ind{	$\theta^{m + 1}$                =   	Surface heat flux at time step m+1}
\ind{	$f_{i}\left( \hspace{5pt} \right)$    =   	Function specifying rate at which heat is
transferred from surface by convention heat transfer during heat transfer mode $i$ (These functions are defined in \colorbox{yellow}{Section~D.6 of Appendix D.)}}
\ind{	$i$                             =   	Number identification of convective heat transfer mode (nucleate boiling, film boiling, etc.)}
\ind{	$C$                             =   	Set of parameters describing coolant conditions}
\ind{	$G$                             =   	Set of parameters describing geometry}
\ind{	$\sigma$                        =   	Stefan-Boltzmann constant}
\ind{	$F_{A}$                         =   	Configuration factor for radiation heat transfer}
\ind{	$F_{E}$                         =   	Emissivity factor for radiation heat transfer}
\ind{	$T_{w}$                         =   	Bulk temperature of water surrounding fuel rod surface}

Equations~\ref{eq:heat_A1Tn_B1_theta} and ~\ref{eq:heat_theta_m1} are two independent equations with
unknowns $T_{n}^{m + 1}$ and $\theta^{m + 1}$. Simultaneous solution of the two equations yields the
new surface temperature $T_{n}^{m + 1}$.
