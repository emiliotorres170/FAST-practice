%-------------------------------------------------------------------------%
% Section 1.2                                                             %
%-------------------------------------------------------------------------%
\section{Limitation of FAST-1.0} \label{section:limitations}
The FAST-1.0 code has inherent limitations. The major limitations are as follows:
\begin{enumerate}

    \item The code is limited to modeling fuel consisting of \UO, \UO-($<10\percent$
        wt\percent)\element{Pu}{}\element{O}{2}(MOX), and \UO-($<10\percent$
        wt\percent\element{Gd}{2}\element{O}{3}) pellets in zirconium alloy cladding with a gas gap
        under light and heavy water reactor conditions from STP up to PWR operating conditions. Some
        additional fuel, cladding and coolant materials are available (see MatLib Documentation) but
        have not been validated, and additional model changes may be required to accommodate them.

    \item The code has been validated up to a rod-average burnup of \SInum{62}{GWd/MTU}, although
        the code should give reasonable predictions for burnup beyond this level for some
        parameters.  Also, the code is not validated beyond the fuel or cladding melting
        temperature. If melting of the fuel or the cladding occurs, the code will stop.

    \item The thermal models of the code are based on 1-D radial heat flow. This assumption is valid
        for modeling a typical fuel rod (i.e., with a large length-to-diameter ratio). Similarly,
        the fission gas release models are based on steady-state and slow power ramp data and do not
        reflect release rates expected for rapid power changes. Therefore, under normal-operation
        analysis, time steps should be no less than \SInum{0.1}{day} but no greater than
        \SInum{50}{days.} When modeling transients or power ramps, fission gas release model will
        not allow for re-solutioning of the gases.

    \item Only small cladding deformations ($<5\percent$ hoop strain) are meaningfully calculated.
        All of the thermal and mechanics modeling assumes an axisymmetric fuel rod with no axial
        constraints. These assumptions are reasonable for modeling an LWR fuel rod.

    \item The code's ability to predict cladding strains resulting from pellet-cladding mechanical
        interaction has been assessed against power ramp data. FAST has been found to slightly
        over-predict cladding strain up to a burnup of about \SInum{65}{GWd/MTU}. The limited high
        burnup data suggests that FAST may under predict the cladding strain during power ramps at
        very high burnup (i.e., $< \SInum{65}{GWd/MTU}$) for hold times greater than
        \SInum{30}{minutes}.

\end{enumerate}
