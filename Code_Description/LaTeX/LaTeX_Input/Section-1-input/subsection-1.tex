%-------------------------------------------------------------------------%
% Section 1.1                                                             %
%-------------------------------------------------------------------------%
\section{Objective of the FAST Code} \label{section:objective}
The ability to accurately calculate the performance of light-water reactor (LWR) fuel rods under
long-term burnup conditions is a major objective of the reactor safety research program being
conducted by the U.S. Nuclear Regulatory Commission (NRC). To achieve this objective, the NRC has
sponsored an extensive program of analytical computer code development, as well as both in-pile and
out-of-pile experiments to benchmark and assess the analytical code capabilities. The computer code
originally developed to calculate the long-term burnup response of a single fuel rod is FRAPCON-3.
This report describes FAST-1.0, the first-release of this code.
\\
\\
FAST is an analytical tool that calculates the thermal-mechanical behavior of nuclear fuel when
given power and boundary conditions. There are two conditions that are commonly referred to as
``steady-state'' and ``transient''. Steady-state means changes are sufficiently slow for the term
``steady-state'' to apply (i.e., time-independent). This includes situations such as long periods at
constant power and slow power ramps that are typical of normal power reactor operations.  Transient
means rapid power and/or boundary condition changes (time-dependent) such as an AOO, RIA or LOCA
event.  The code calculates the variation with time of all significant fuel rod variables, including
fuel and cladding temperatures, cladding hoop strain, cladding oxidation, hydriding, fuel
irradiation swelling, fuel densification, fission gas release, and rod internal gas pressure.
\\
\\
FAST uses fuel, cladding, and gas material properties from MatLib that have been recently updated to
include burnup-dependent properties and properties for advanced zirconium-based cladding alloys.
These properties are documented elsewhere (\cite{ref:Luscher2014b}). The only
material properties not included in the updated MATPRO document are fission gas release, cladding
corrosion, and cladding hydrogen pickup, and these properties are described in this document. The
material properties in FRAPCON-3 are contained in modular subroutines that define material
properties for temperatures ranging from room temperatures to temperatures above melting and for
rod-average burnup levels between 0 and \SInum{62}{GWd/MTU}. Each subroutine defines only a single
material property. For example, FAST contains subroutines defining fuel thermal conductivity as a
function of fuel temperature, fuel density, and burnup; fuel thermal expansion as a function of fuel
temperature; and the cladding stress-strain relation as a function of cladding temperature, strain
rate, cold work, hydride content, and fast neutron fluence.
\\
\\
The predecessors of FAST are the FRAPCON-4.0 and FRAPTRAN-2.0 codes.  The FRAPCON-3 and FRAPCON-4
codes were developed at Pacific Northwest National Laboratory (PNNL). The code underwent 11 total
releases, starting with FRAPCON-3 v1.0 (\cite{ref:Berna1997}) in 1997 through
FRAPCON-4.0 Patch 1 (\cite{ref:Geelhood2015a}) in 2015. The FRAPTRAN-1 and
FRAPTRAN-2 codes
\\
\\
FAST-1.0, takes a major step toward code simplification by removing extra input parameters and model
selection features that cannot easily be measured and have a large impact on results. Also,
reasonable default values are set for some parameters. The only model options available to the user
are in the selection of the mechanical model and in the selection of the fission gas release model.
\\
\\
For the mechanical model, the user may select the FRACAS-I model (finite difference model) or the
FEA (finite element analysis) model. The FRACAS-I model is recommended by PNNL and is the default
selection. The FEA model is useful for modeling cladding axial strain in cases where there is slip
between the fuel and cladding. The details of the FEA model are described elsewhere
(\cite{ref:Knuutila2006a}). This document is posted on the FRAPCON/FRAPTRAN code users'
group website \url{http://frapcon.labworks.org}. Only the FRACAS-1 mechanical model will be
described in this document.
\\
\\
For the fission gas release model, the user can select the Massih model, one of the ANS-5.4 models
(ANS-1982 or ANS-2011), or the FRAPFGR model. The Massih model is recommended and is the default
model. The ANS-5.4 1982 model is useful for calculating the release of short-lived radioactive gas
nuclides and has been shown to provide very conservative release values. The ANS-5.4 2011 model also
calcualtes the short-lived radioactive gas nuclides but is non-conservative with respect to total
fission gas release as it does not include any long-lived nuclides (such as \element{Kr}{} and
\element{Xe}{}).  The FRAPFGR model is useful for determining the distribution of fission gases
within the fuel pellet (grain boundary vs. retained within grains). The Massih and FRAPFGR models
will be described in this document.
\\
\\
FAST-1.0 includes fuel models for uranium dioxide (\UO), mixed oxide fuel or MOX ((\element{U}{},
\element{Pu}{})\element{O}{2}), integral fuel burnable absorber (IFBA) and gadolinia doped fuel, and
cladding models for Zircaloy-2, Zircaloy-4, M5, ZIRLO, and Optimized ZIRLO.  For scoping studies,
FAST has been updated with properties for ATF candidate fuels (Uranium Silicde,
\element{U}{3}\element{Si}{2}) and claddings (\element{FeCrAl}{}, \element{SiC}{}), as well as
metallic fuels (\element{UPuZr}{}).  Other code improvements include an Excel-based input generator,
an Excel-based plot routine, and the ability to bias model predictions for uncertainty analyses.
