%-------------------------------------------------------------------------%
% Section 2.5 Fuel Rod Mechanical Response
%-------------------------------------------------------------------------%
\section{Fuel Rod Mechanical Response}\label{section:fuel-rod-mechanical-response}

An accurate calculation of fuel and cladding deformation is necessary in any fuel rod response
analysis because the heat transfer coefficient across the fuel-cladding gap is a function of both
the effective fuel-cladding gap size and the fuel-cladding interfacial pressure. In addition, an
accurate calculation of stresses in the cladding is needed to accurately calculate the strain and
the onset of cladding failure (and subsequent release of fission products). This section describes
the default mechanical model, FRACAS-I. The optional cladding FEA model is described elsewhere
(\cite{ref:Knuutila2006a}).

%-------------------------------------------------------------------------%
% Section 2.5.1 The FRACAS-1 Model
%-------------------------------------------------------------------------%

\subsection{The FRACAS-1 Model}\label{section:the-fracas-1-model}

The FRACAS-I model is available for the calculation of the small displacement deformation of the
fuel and cladding. The simplified model, FRACAS-I, neglects the stress-induced deformation of the
fuel, and is called the ``rigid pellet model.''
\\
\\
In analyzing the deformation of fuel rods, two physical situations are envisioned. The first
situation occurs when the fuel and cladding are not in contact. Here the problem of a cylindrical
shell (the cladding) with specified internal and external pressures and a specified cladding
temperature distribution must be solved. This situation is called the ``open gap'' regime.
\\
\\
The second situation envisioned is when the fuel (considerably hotter than the cladding) has
expanded so as to be in contact with the cladding. Further heating (thermal expansion) of the fuel
``drives'' the cladding outward. This situation is called the ``closed gap'' regime. In addition,
this closed gap can occur due to fuel swelling, relocation, and the creep of the cladding onto the
fuel due to a high coolant pressure.
\\
\\
The deformation analysis in FAST consists of a small deformation analysis that includes stresses,
strains, and displacements in the fuel and cladding for the entire fuel rod. This analysis assumes
that the cladding retains its cylindrical shape during deformation, and includes the effects of the
following:

\begin{itemize}
    \item Fuel thermal expansion, swelling, densification, and relocation

    \item Cladding thermal expansion, creep, and plasticity

    \item Fission gas and external coolant pressures

\end{itemize}

As part of the small displacement analysis, the applicable local deformation regime (open gap, or
closed gap) is determined. Finally, an analysis is performed to determine cladding stresses and
strains.
\\
\\
In Section~\ref{section:general-theory-and-method-of-solution}, the general theory of plastic
analysis is outlined and the method of solution used in the FRACAS-I model is presented. This method
of solution is used in the rigid pellet model. In
Section~\ref{section:rigid-pellet-cladding-deformation-model}, the equations for the rigid pellet
model are described.
%-------------------------------------------------------------------------%
% Section 2.5.1.1 General Theory and Method of Solution
%-------------------------------------------------------------------------%
\subsubsection{General Theory and Method of Solution}\label{section:general-theory-and-method-of-solution}

The general theory of plastic analysis and the method of solution are
used in the rigid pellet model.

\subsubsection{General Considerations in Elastic-Plastic Analysis}\label{section:general-considerations-in-elastic-plastic-analysis}

Problems involving elastic-plastic deformation and multiaxial stress states involve aspects that do
not require consideration in a uniaxial problem. In the following discussion, an attempt is made to
briefly outline the structure of incremental plasticity and to outline the method of successive
substitutions (also called the method of successive elastic solutions) (\cite{ref:Mendelson1968a}),
which has been used successfully in treating multiaxial elastic-plastic problems. The method can be
used for any problem for which a solution based on elasticity can be obtained.  This method is used
in the rigid pellet model.
\\
\\
In a problem involving only uniaxial stress, \(\sigma_{1}\), the strain, \(\varepsilon_{1}\), is
related to the stress by an experimentally determined stress-strain curve as shown in
Figure~\ref{fig:isothermal_stress_strain_curve} (including the elastic strains and plastic strains,
but without thermal expansion strains) so Hooke's law is taken as:

\begin{equation}
    \label{eq:hooks_law}
    \varepsilon_{1} = \frac{\sigma_{1}}{E} + \varepsilon_{1}^{P} + \int_{}^{}\alpha dT
\end{equation}

Where,
        \ind{   $\varepsilon_{1}^{P}  = $ Stored energy                 \sinum{J/kg}  }
        \ind{   $E                    = $ Mass of ring segment $i$      \sinum{kg}    }

The onset of yielding occurs at the yield stress, which can be determined directly from
Figure~\ref{fig:isothermal_stress_strain_curve}. Given a load (stress) history, the resulting
deformation can be determined in a simple manner. The increase of yield stress with work-hardening
is easily computed directly from Figure~\ref{fig:isothermal_stress_strain_curve}.
\\
\\
In a problem involving multiaxial states of stress, as with a fuel rod, the situation is not as
clear. In such a problem, a method of relating the onset of plastic deformation to the results of a
uniaxial test is required, and further, when plastic deformation occurs, some means is needed for
determining how much plastic deformation has occurred and how that deformation is distributed among
the individual components of strain. These two complications are taken into account by use of the
so-called ``yield function'' and ``flow rule,'' respectively.
\\
\\
A wealth of experimental evidence exists on the onset of yielding in a multiaxial stress state. Most
of this evidence supports the von Mises yield criterion, which asserts that yielding occurs when the
stress state is such that

\begin{equation}
    {\sigma_{y}}^{2} = \frac{1}{2}\left\lbrack \left( \sigma_{1} -
    \sigma_{2} \right)^{2} + \left( \sigma_{2} - \sigma_{3} \right)^{2} +
    \left( \sigma_{3} - \sigma_{1} \right)^{2} \right\rbrack
    \label{eq:von_mises_yield_criterion}
\end{equation}

Where,
\ind{   \(\sigma_{i}\)  =  Principle stresses in direction \(i,i = 1,2,3\)                }
\ind{   \(\sigma_{y}\)  =  Yield stress as determined in a uniaxial stress-strain test.   }

The square root of the left side of Equation~\ref{eq:von_mises_yield_criterion} is referred to as
the ``effective stress,'' \(\sigma_{e}\), and this effective stress is one commonly used type of yield
function.
\\
\\
To determine how the yield stress changes with permanent deformation, the yield stress is
hypothesized to be a function of the equivalent plastic strain, \(\varepsilon^{p}\). An increment of
equivalent plastic strain is determined at each load step, and \(\varepsilon^{p}\) is defined as the
sum of all increments incurred, as shown in Equation~\ref{eq:sum_plastic_strain_increments}.

\begin{figure}
    \includegraphics[height=\figheight\textheight]{../media/image15}
    \caption{Typical Isothermal Stress-Strain Curve \colorbox{yellow}{Should consider updating}}
    \label{fig:isothermal_stress_strain_curve}
\end{figure}

\begin{equation}
    \label{eq:sum_plastic_strain_increments}
    \varepsilon^{p} = \sum_{}^{}{d\varepsilon^{p}}
\end{equation}

Each increment of effective plastic strain is related to the individual plastic strain components by

\begin{equation}
    \label{eq:eff_plastic_strain_to_individual_plastic_strain}
    d\varepsilon^{p} = \frac{\sqrt{2}}{3}\sqrt{\left( \varepsilon_{1}^{p} - d\varepsilon_{2}^{p} \right)^{2} + \left( d\varepsilon_{2}^{p} - d\varepsilon_{3}^{p} \right)^{2} + \left( d\varepsilon_{3}^{p} - d\varepsilon_{1}^{p} \right)^{2}}
\end{equation}

Where,
\ind{   $d\varepsilon_{i}^{p}  = $ Plastic strain components in principle coordinates }

Experimental results indicate that at pressures on the order of the yield stress, plastic
deformation occurs with no change in volume, which implies that

\begin{equation}
    d\varepsilon_{1}^{p} + d\varepsilon_{2}^{p} + d\varepsilon_{3}^{p} = 0
    \label{eq:plastic_deformation_no_volume_change}
\end{equation}

Therefore, in a uniaxial test with $\sigma_{1}=\sigma$, $\sigma_{2}=\sigma_{3}= 0$, the plastic
strain increments are

\begin{equation}
    \label{eq:plastic_strain_increment_uniaxial_test}
    d\varepsilon_{2}^{p} = d\varepsilon_{3}^{p} = - \frac{1}{2}d\varepsilon_{1}^{p}
\end{equation}

Therefore, in a uniaxial test, Equations~\ref{eq:von_mises_yield_criterion}
and~\ref{eq:eff_plastic_strain_to_individual_plastic_strain} reduce to

\begin{equation}
    \label{eq:stress_equals_yield_stress}
    \sigma = \sigma_{y}
\end{equation}

\begin{equation}
    \label{eq:plastic_strain_equals_dep1}
    d\varepsilon^{p} = d\varepsilon_{1}^{p} 
\end{equation}

Thus, when the assumption is made that the yield stress is a function of the total effective plastic
strain (called the ``strain-hardening hypothesis''), the functional relationship between yield
stress and plastic strain can be taken directly from a uniaxial stress-strain curve by virtue of
Equations~\ref{eq:stress_equals_yield_stress} and~\ref{eq:plastic_strain_equals_dep1}.

The relationship between the magnitudes of the plastic strain increments and the effective plastic
strain increment is provided by the Prandtl-Reuss flow rule:

\begin{equation}
    \label{eq:prandtl_reuss_flow_rule}
    d\varepsilon_{i}^{p} + \frac{3}{2}\frac{d\varepsilon_{1}^{p}}{\sigma_{e}}S_{i},\; i = 1,\; 2,\; 3
\end{equation}

Where,
\ind{   \(S_{i}\) values are the deviatoric stress components (in principal coordinates)} 
Furthermore, the deviatoric stress components, $S_{i}$, are defined by
\begin{equation}
    S_{i} = \sigma_{i} - \frac{1}{3}\left( \sigma_{1} + \sigma_{2} + \sigma_{3} \right),\ i = 1,\ 2,\ 3
    \label{eq:deviatoric_stress_components}
\end{equation}

Equation~\ref{eq:prandtl_reuss_flow_rule} embodies the fundamental observation of plastic
deformation; that is, plastic strain increments are proportional to the deviatoric stresses. The
constant of proportionality is determined by the choice of the yield function. Direct substitution
shows that
Equations~\ref{eq:von_mises_yield_criterion},~\ref{eq:eff_plastic_strain_to_individual_plastic_strain},~\ref{eq:prandtl_reuss_flow_rule},
and~\ref{eq:deviatoric_stress_components} are consistent with one another.
\\
\\
Once the plastic strain increments have been determined for a given load step, the total strains are
determined from a generalized form of Hooke's law given by

\begin{subequations}
    \begin{equation}
        \varepsilon_{1} = \frac{1}{E}\left( \sigma_{1} - \nu\left( \sigma_{2} + \sigma_{3} \right) \right) + \varepsilon_{1}^{p} + d\varepsilon_{1}^{p} + \int_{}^{}{\alpha_{1}\text{dt}}  \\
        \label{eq:total_strains_hooks_law-1}
    \end{equation}
    \begin{equation}
        \varepsilon_{2} = \frac{1}{E}\left( \sigma_{2} - \nu\left( \sigma_{1} + \sigma_{3} \right) \right) + \varepsilon_{2}^{p} + d\varepsilon_{2}^{p} + \int_{}^{}{\alpha_{2}\text{dt}}  \\
        \label{eq:total_strains_hooks_law-2}
    \end{equation}
    \begin{equation}
        \varepsilon_{3} = \frac{1}{E}\left( \sigma_{3} - \nu\left( \sigma_{2} + \sigma_{1} \right) \right) + \varepsilon_{3}^{p} + d\varepsilon_{3}^{p} + \int_{}^{}{\alpha_{3}\text{dt}}
        \label{eq:total_strains_hooks_law-3}
    \end{equation}
    \label{eq:total_strains_hooks_law}
\end{subequations}
    
%\begin{equation}
%    \label{eq:total_strains_hooks_law}
%    \begin{aligned}
%       \varepsilon_{1} = \frac{1}{E}\left( \sigma_{1} - \nu\left( \sigma_{2} + \sigma_{3} \right) \right) + \varepsilon_{1}^{p} + d\varepsilon_{1}^{p} + \int_{}^{}{\alpha_{1}\text{dt}}  \\
%       \varepsilon_{2} = \frac{1}{E}\left( \sigma_{2} - \nu\left( \sigma_{1} + \sigma_{3} \right) \right) + \varepsilon_{2}^{p} + d\varepsilon_{2}^{p} + \int_{}^{}{\alpha_{2}\text{dt}}  \\
%       \varepsilon_{3} = \frac{1}{E}\left( \sigma_{3} - \nu\left( \sigma_{2} + \sigma_{1} \right) \right) + \varepsilon_{3}^{p} + d\varepsilon_{3}^{p} + \int_{}^{}{\alpha_{3}\text{dt}}
%    \end{aligned}
%\end{equation}

Where,
\ind{   $\varepsilon_{i}^{p}  =$  Total plastic strain components at the end of the previous load increment }
\ind{   $E                    =$  Modulus of elasticity (See MatLib)                                        }
\ind{   $\nu                  =$  Poisson's ratio (See MatLib)                                              }

The remaining continuum field equations of equilibrium, strain displacement, and strain
compatibility are unchanged. The complete set of governing equations is presented in
Table~\ref{tab:summary_FRACAS1_governing_equations}, written in terms of rectangular Cartesian
coordinates and employing the usual indicial notation in which a repeated Latin index implies
summation (index notation). This set of equations is augmented by an experimentally determined
uniaxial stress-strain relation.

\renewcommand{\captiontext}{Summary of FRACAS-I Governing Equations}
\begin{longtable}[c]{|D{1.0in}A{5.0in}|}
    \caption{\captiontext} \label{tab:summary_FRACAS1_governing_equations}                      \\  \hline
    \endfirsthead
    \caption{\captiontext~(continued)} \\   \hline
    \endhead
        \multicolumn{2}{|D{6.0in}|}{\textbf{Equilibrium}}                                       \\  
        \multicolumn{2}{|A{6.0in}|}{$\sigma_{ji,j} + \rho f_{i} = 0 $}                          \\  \hline

        \multicolumn{2}{|D{6.0in}|}{\textbf{Stress strain}}                                     \\  
        \multicolumn{2}{|A{6.0in}|}{$\varepsilon{ij} = \frac{1+\nu}{E}\sigma_{ij}-\delta_{ij}\left(\frac{\nu}{E}-\int^{}_{}\alpha dT\right)+\varepsilon_{ij}^{p}+d\varepsilon_{ij}^{p}  $ } \\  \hline
        
        \multicolumn{2}{|D{6.0in}|}{\textbf{Compatibility}}                                     \\  
        \multicolumn{2}{|A{6.0in}|}{$\varepsilon_{ij,kl} + \varepsilon_{kl,ij} -\varepsilon_{ik,jl} - \varepsilon_{jl,ik} = 0 $}    \\  \hline
        
        \multicolumn{2}{|D{6.0in}|}{\textbf{Definitions used in plasticity}}                    \\  
        \multicolumn{2}{|A{6.0in}|}{$\sigma_{e} \triangleq \sqrt{\frac{3}{2}S_{ij}S_{ij}}$}     \\  
        \multicolumn{2}{|A{6.0in}|}{$S_{ij} \triangleq \sigma_{ij} - \frac{1}{3}\sigma_{kk}$}   \\ \hline
        
        \multicolumn{2}{|D{6.0in}|}{\textbf{Prandtl-Reuss flow rule}}                           \\  
        \multicolumn{2}{|A{6.0in}|}{$d\varepsilon_{ij}^{p} = \frac{3}{2}\frac{d\varepsilon^{p}}{\sigma_{e}}S_{ij}$} \\ \hline
\end{longtable}
Where,
\ind{   $\sigma  =$  stress tensor                           }
\ind{   $\rho    =$  mass density                            }
\ind{   $f_{i}   =$  components of body force per unit mass  }

\subsubsection{The Method of Solution} \label{section:method_of_solution}

When the problem under consideration is statically determinate so that stresses can be found from
equilibrium conditions alone, the resulting plastic deformation can be determined directly. However,
when the problem is statically indeterminate and the stresses and deformation must be found
simultaneously, the full set of plasticity equations proves to be quite formidable, even in the case
of simple loadings and geometries.
\\
\\
One numerical procedure which has been used with considerable success is the method of successive
substitutions. This method can be applied to any problem for which an elastic solution can be
obtained, either in closed form or numerically. A full discussion of this technique, including a
number of technologically useful examples, is contained in \cite{ref:Knuutila2006a}.
\\
\\
Briefly, the method involves dividing the loading path into small increments. For example, in the
present application, the loads are external pressure, temperature, and either internal pressure or a
prescribed displacement of the inside surface of the cladding. These loads all vary during the
operating history of the fuel rod. For each new increment of the loading, the solution to all the
plasticity equations listed in Table~\ref{tab:summary_FRACAS1_governing_equations} is obtained as
follows.
\\
\\
First, an initial estimate of the plastic strain increments, \(d\varepsilon_{ij}^{p}\), is made.
Based on these values, the equations of equilibrium, Hooke's law, and strain-displacement and so
obtained, the deviatoric stresses, \(S_{ij}\), may be computed. This ``pseudo-elastic'' solution
represents one path in the computational scheme.
\\
\\
Independently, through use of the assumed \(d\varepsilon_{ij}^{p}\) values, the increment of
effective plastic strain, \(d\varepsilon^{p}\), may be computed. From this result and the
stress-strain curve, a value of the effective stress, \(\sigma_{e}\), is obtained from
Equation~\ref{eq:von_mises_yield_criterion}.
\\
\\
Finally, a new estimate of the plastic strain increments is obtained from the Prandtl-Reuss flow
rule:

\begin{equation}
    d\varepsilon_{ij}^{p} = \frac{3}{2}\frac{d\varepsilon^{p}}{\sigma_{e}}S_{ij}
    \label{eq:new_estimate_prandtl_reuss_flow_rule}
\end{equation}
The entire process is continued until the \(d\varepsilon_{ij}^{p}\) converges. A schematic of the
iteration scheme is shown in Figure~\ref{fig:schematic_method_success_elastic_solns}.
\\
\\
The mechanism by which improved estimates of \(d\varepsilon_{ij}^{p}\) are obtained results from the
fact that the effective stress obtained from \(d\varepsilon^{p}\) and the stress-strain curve will
not be equal to the effective stress that would be obtained with the stresses from the elastic
solution. The effective stresses will only agree when convergence is obtained.

\begin{figure}
    \includegraphics[height=\figheight\textheight]{../media/image17}
    \caption{Schematic of the Method of Successive Elastic Solutions \colorbox{yellow}{Should Consider re-making}}
    \label{fig:schematic_method_success_elastic_solns}
\end{figure}

The question of convergence is one that cannot, in general, be answered \textit{a priori}. However,
convergence can be shown to be obtained for sufficiently small load increments. Experience has shown
that this technique is suitable for both steady-state and transient fuel rod analyses.

\subsubsection{Extension to Creep}\label{section:extension-to-creep}

The method of solution described for the time-independent plasticity calculations can also be used
for time-dependent creep calculations. In this context, the term ``creep'' refers to any
time-dependent constant volume permanent deformation. Creep is a stress-driven process and is
usually highly dependent on temperature.  The only change required to extend the method of
successive elastic solutions to allow consideration of creep is to rewrite the Prandtl-Reuss flow
rule, Equation~\ref{eq:prandtl_reuss_flow_rule}, as

\begin{subequations}
    \begin{equation}
        d \varepsilon^{c}_{1} = 1.5 \frac{\dot{\varepsilon} \Delta t}{\sigma_{\varepsilon}}S_{1} + \frac{\dot{V^{c}}\Delta t}{9} \frac{\left(\sigma_{1} + \sigma_{2} + \sigma_{3} \right)}{\sigma_{m}} \\ 
        \label{eq:prandtl_reuss_with_creep-1}
    \end{equation}
    \begin{equation}
        d \varepsilon^{c}_{2} = 1.5 \frac{\dot{\varepsilon} \Delta t}{\sigma_{\varepsilon}}S_{2} + \frac{\dot{V^{c}}\Delta t}{9} \frac{\left(\sigma_{1} + \sigma_{2} + \sigma_{3} \right)}{\sigma_{m}} \\
        \label{eq:prandtl_reuss_with_creep-2}
    \end{equation}
    \begin{equation}
        d \varepsilon^{c}_{3} = 1.5 \frac{\dot{\varepsilon} \Delta t}{\sigma_{\varepsilon}}S_{3} + \frac{\dot{V^{c}}\Delta t}{9} \frac{\left(\sigma_{1} + \sigma_{2} + \sigma_{3} \right)}{\sigma_{m}} \\
        \label{eq:prandtl_reuss_with_creep-3}
    \end{equation}
    \label{eq:prandtl_reuss_with_creep}
\end{subequations}
Where,
\ind{   \(\sigma_{m}\) is the mean stress.}  

The first term on the right-hand side of each of these equations
(Equation~\ref{eq:prandtl_reuss_with_creep}) computes the constant volume creep strain, whereas the
second term in each equation computes the permanent change in volume. To use this form of the flow
rule, two additional material property correlations must be available. The first is a correlation
for constant volume creep strain, \(\varepsilon^c\) (taken in a uniaxial test), as a function of
stress, time, temperature, and neutron flux, namely

\begin{equation}
    \varepsilon^{c} = f\left(\sigma, T, t, \phi \right)
    \label{eq:constant_creep_volume_strain}
\end{equation}

Where,
\ind{   $\sigma  =$  uniaxial stress \sinum{MPa}                      }
\ind{   $T       =$  temperature     \sinum{K}                        }
\ind{   $t       =$  time            \sinum{s}                        }
\ind{   $\phi    =$  neutron flux    \sinum{n/m^{2}-s} }

In the FRACAS-I model, the strain hardening hypothesis is assumed, which implies that the strain
correlation can be differentiated with respect to time and solved for creep strain rate in the form

\begin{equation}
    \label{eq:creep_strain_rate}
    \dot{\varepsilon} = h\left(\sigma, \varepsilon^{c}, t, T, \phi\right)
\end{equation}

which is no longer an explicit function of time. The function ``h'' is contained in subroutine
CREPR, and is described as follows.
\\
\\ 
A model described by Limb\"{a}ck and Andersson (\cite{ref:Limback1996a}) of ABB Atom and AB
Sandvik Steel, respectively, was selected for cladding irradiation creep in FAST.  This model uses a
thermal creep model described by \cite{ref:Matsuo1987b} and an empirical irradiation creep rate with
tuned model parameters that were fit to the data set given by \cite{ref:Franklin1983}. The
Limb\"{a}ck model was further modified by PNNL to use effective stress rather than hoop stress as an
input so that the principal stresses could be included and account for the difference in creep
behavior during tensile and compressive creep.  Several of the fitting coefficients from the
Limb\"{a}ck paper were consequently changed to accommodate this change based on comparisons to
several data sets ( \cite{ref:Franklin1983}; \cite{ref:Soniak2002b}; \cite{ref:Gilbon2000a}; and
\cite{ref:Sontheimer1994}).  In addition, a temperature-dependent term was added to the formula for
irradiation creep strain rate. This was done because creep data were used with temperature greater
than the temperature of the data given by Franklin, and these data along with the Franklin data
showed a dependence on temperature. This model has different parameters for stress relief annealed
(SRA) and re-crystallized annealed (RXA) cladding types, and provides reasonable creep strains in
the LWR range of temperature and cladding hoop stresses that compare well to data. This model is
described below.
\\
\\
The steady state thermal and irradiation creep rates are given by

\begin{equation}
    \dot{\varepsilon}_{th} = A \frac{E}{T} \left( \sinh \frac{a_{i}\sigma_{eff}}{E}\right)^{n} \exp\left(\frac{-Q}{RT}\right)
    \label{eq:steady_state_thermal_creep_rate}
\end{equation}

\begin{equation}
    \label{eq:steady_state_irradiation_creep_rate}
    \varepsilon = C_{0}  \phi^{C_{1}}  \sigma^{C_{2}}_{eff}  f\left(T\right)
\end{equation}

Where

\ind{   $\dot{\varepsilon}_{th},\dot{\varepsilon}_{irr}  = $  thermal and irradiation strain rate, respectively \sinum{m/m-hr} }

These rates are added together, so 

\begin{equation}
    \dot{\varepsilon}_{th,irr} = \dot{\varepsilon}_{th} + \dot{\varepsilon}_{irr}
    \label{eq:steady_state_total_creep_rate}
\end{equation}

The saturated primary hoop strain is given by

\begin{equation}
    \varepsilon^{s}_{p} = 0.0216  \dot{\varepsilon}^{0.109}_{th+irr}\left(2-\tanh\left(35500 \varepsilon_{th+irr}\right)\right)^{-2.05} 
    \label{eq:saturated_primary_hoop_strain}
\end{equation}

The total thermal strain is given by

\begin{equation}
    \varepsilon_{H}  = \varepsilon^{s}_{p} \left(1-\exp\left(-52\sqrt{\dot{\varepsilon}_{th+irr} t} \right)\right) + \dot{\varepsilon}_{th +irr} t
    \label{eq:total_thermal_strain}
\end{equation}

In FAST, strain rate is used. Taking the derivative with respect to time of the expression above
gives

\begin{equation}
    \dot{\varepsilon_{H}}  = \frac{52  \varepsilon^{s}_{p}  \dot{\varepsilon}^{\frac{1}{2}}_{th + irr}}{2  t^{\frac{1}{2}}} \exp\left(-52 \sqrt{\dot{\varepsilon}_{th+irr} t} \right) + \dot{\varepsilon}_{th+irr}
    \label{eq:strain_rate}
\end{equation}                

Where,
\ind{   $T             =$  Temperature       \sinum{K}                         }
\ind{   $t             =$  Time              \sinum{hours}                     }
\ind{   $\sigma_{eff}  =$  Effective stress  \sinum{MPa}                       }
\ind{   $\phi          =$  Fast neutron flux \sinum{n/m^{2}-s}  }

The first term in Equation~\ref{eq:strain_rate} represents the primary creep. It has been observed
that following significant changes in stress or stress reversals, the primary creep is best related
to the change in effective stress and the direction of the change in hoop stress
(\cite{ref:Geelhood2013}).  In FAST, the first term in Equation~\ref{eq:strain_rate} is
calculated based on the time since the last significant stress change ($> \SInum{5}{MPa}$) using
the change in effective stress and in the direction of the change
in hoop stress.
\\
\\
Table~\ref{tab:parameters_creep_equation_sra_rxa} lists the parameters used in these equations for
SRA and RXA cladding types. These parameters are those recommended by Limb\"{a}ck and Andersson
(\cite{ref:Limback1996a}), with the exception of the ``$A$'' parameter and
the ``$f(T)$'' parameter, that were modified by PNNL.

\renewcommand{\captiontext}{Parameters for FAST-1.0 Creep Equation for SRA and RXA Cladding}
\begin{longtable}[c]{D{1.0in} D{0.5in} A{3.5in} A{1.0in}}
    \caption{\captiontext} \label{tab:parameters_creep_equation_sra_rxa}                                                    \\  \hline
        \textbf{ Parameter}     & \textbf{Units}            &   \textbf{ Values for SRA}    &   \textbf{Values for RXA}     \\  \hline
    \endfirsthead
    \caption[]{\captiontext~(continued)}                                                                                    \\  \hline
        \textbf{ Parameter}     & \textbf{Units}            &   \textbf{ Values for SRA}    &   \textbf{Values for RXA}     \\  \hline
    \endhead
        $A$                     &       \sinum{K/MPa-hr}     &        \num{1.08E9}                                                      &        \num{5.47E8}                       \\
        $E$                     &       \sinum{MPA}          &        $1.149-59.9\multi T$                                              &                                           \\
        $a_{i}$                 &       \sinum{1/MPa}        &       $650{1-0.56\left(1-\exp(\num{-1.4E-27}\Phi^{1.3})\right)}$         &                                           \\
                                &                            &       $\Phi = $ fast neutron fluence \sinum{n/cm^{2}}                    &                                           \\    
        $n$                     &       \sinum{unitless}     &       2.0                                                                &       3.5                                 \\
        $Q$                     &       \sinum{kJ/mole}      &       201                                                                &                                           \\
        $R$                     &       \sinum{kJ/mol-K}     &       \num{8.314E-3}                                                     &                                           \\
        $C_{0}$                 &       \sinum{n/m^{2}-s}    &       \num{4.0985E-24}                                                   &       \num{1.87473E-24}                   \\
                                &       \sinum{MPa^{-c2}}    &                                                                          &                                           \\         
                                &                            &                                                                          &                                           \\    
        $C_{1}$                 &       \sinum{unitless}     &       0.85                                                               &                                           \\
        $C_{2}$                 &       \sinum{unitless}     &       1.0                                                                &                                           \\
        $f(T)$                  &       \sinum{unitless}     &       0.7283 for $T< \SInum{570}{K}$                                     &       0.7994                              \\ 
                                &                            &       $7.0237+0.0136T$ for $\SInum{570}{K}<T<\SInum{625}{K}$             &                                           \\
                                &                            &       1.4763 for $T>\SInum{625}{K}$                                      &      $3.18562+\num{6.99132E-3}T$          \\
                                &                            &                                                                          &       1.1840                              \\
\end{longtable}

The effective stress in the cladding is found using the principal stresses at the mid-wall radius
using the thick wall formula as shown in
Equations~\ref{eq:principle_stress_radial}-~\ref{eq:principle_stress_hoop}.

\begin{equation}
    \label{eq:principle_stress_radial}
    \sigma_{r} = \frac{P_{i}r_{i}^{2}-P_{o}r_{o}^{2} + \frac {r_{i}^2r_{o}^2 \left(P_{o} - P_{i}\right)}{r^{2}}}{r_{o}^{2}- r_{i}^{2}}    
\end{equation}

\begin{equation}
    \label{eq:principle_stress_tangential}
    \sigma_{t} = \frac{P_{i}r_{i}^{2}-P_{o}r_{o}^{2} - \frac {r_{i}^2r_{o}^2 \left(P_{o} - P_{i}\right)}{r^{2}}}{r_{o}^{2}- r_{i}^{2}}    
\end{equation}

\begin{equation}
    \label{eq:principle_stress_hoop}
    \sigma_{l} = \frac{P_{i}r_{i}^{2}  - P_{o}r_{o}^{2} }{r_{o}^{2}- r_{i}^{2}}
\end{equation}

Where,
\ind{   $P_{o}       =$  outer pressure      }
\ind{   $r_{i}       =$  inner radius        }
\ind{   $r_{o}       =$  outer radius        }
\ind{   $r           =$  radius within tube  }
\ind{   $\sigma_{r}  =$  radial stress       }
\ind{   $\sigma_{t}  =$  tangential stress   }
\ind{   $\sigma_{l}  =$  longitudinal stress }

The effective stress, $\sigma_{eff}$, is then given by

\begin{equation}
    \label{eq:effective_stress}
    \sigma_{eff} = \sqrt{\frac{1}{2}\left(\left(\sigma_{l}-\sigma_{t}\right)^{2} 
    + \left(\sigma_{t}-\sigma_{r}\right)^{2} + \left(\sigma_{r}-\sigma_{l}\right)^{2} \right)}
\end{equation}

The correlations above are developed for SRA and RXA Zircaloy-4 and Zircaloy-2. For M5, the
correlation for RXA Zircaloy is used. For ZIRLO and Optimized ZIRLO, the correlation for SRA
Zircaloy reduced by a factor of 0.8 is used (\cite{ref:Sabol1994b}).
\\
\\
A plot of the resulting creep strain is shown as a function of time and effective stress for
representative flux and temperature values in Figure~\ref{fig:creep_strain_function_of_time}.

\begin{figure}
    \includegraphics[height=\figheight\textheight]{../media/SRA_Zircaloy_Creep_Strain}
    \caption{Cladding Creep Strain as a Function of Time and Hoop Stress for \SInum{630}{\mDF} and Flux=\SInum{10E18}{n/m^{2}-s} for SRA Zircaloy}
    \label{fig:creep_strain_function_of_time}
\end{figure}

\begin{figure}
    \includegraphics[height=\figheight\textheight]{../media/RXA_Zircaloy_Creep_Strain}
    \caption{Cladding Creep Strain as a Function of Time and Hoop Stress for \SInum{630}{\mDF} and Flux=\SInum{10E18}{n/m^{2}-s} for RXA Zircaloy}
    \label{fig:creep_strain_function_of_time}
\end{figure}

The second additional correlation required is a relationship between the rate of permanent
volumetric strain and the applied loads; that is,

\begin{equation}
    \label{eq:rate_perm_volume_strain_applied_loads}
    \dot{V}^{c} = g\left( \sigma_{m}, T, t, V_{avail} \right)
\end{equation}

Where,
\ind{   $\sigma_{m}  =$  $(\sigma_{1}+\sigma2+\sigma3)/3$ the mean stress         \sinum{MPa}   }
\ind{   $T           =$  temperature                                              \sinum{K}     }
\ind{   $t           =$  time                                                     \sinum{s}     }
\ind{   $V_{avail }  =$  measure of maximum permanent volumetric change possible                }

The permanent volumetric strain increment $dV_{c}$ is related to the creep strain increments by the
equation

\begin{equation}
    \label{eq:delta_rate_perm_volume_strain_applied_loads}
    dV^{c} = d\varepsilon_{1}^{c} +  d\varepsilon_{1}^{c}  + d\varepsilon_{1}^{c}  
\end{equation}                 

As previously noted, the FRACAS-I model is the default model available for analyzing the small
deformation of the fuel and cladding. The model considers the fuel pellets to be essentially rigid
and to deform due to thermal expansion, swelling, and densification only. Thus, in the rigid pellet
model, the displacement of the fuel is calculated independently of the deformation of the cladding.
This rigid pellet analysis is performed with the FRACAS-I subcode.

\subsubsection{Rigid Pellet Cladding Deformation Model}\label{section:rigid-pellet-cladding-deformation-model}

FRACAS-I consists of a cladding deformation model and a fuel deformation
model. If the fuel-cladding gap is closed, the fuel deformation model
will apply a driving force to the cladding deformation model. The
cladding deformation model, however, never influences the fuel
deformation model.
\\
\\
The cladding deformation model in FRACAS-I is based on the following
assumptions:

\begin{itemize}
    \item Incremental theory of plasticity.
    \item Prandtl-Reuss flow rule.
    \item Isotropic work-hardening.
    \item Thick wall cladding (thick wall approximation formula is used to calculate stress at midwall).
    \item If fuel and cladding are in contact, no axial slippage occurs at fuel cladding interface.
    \item Bending strains and stresses in cladding are negligible.
    \item Axisymmetric loading and deformation of cladding.
\end{itemize}

The fuel deformation model in FRACAS-I is based on the following
assumptions:

\begin{itemize}
    \item Thermal expansion, swelling, and densification are the only sources for fuel deformation.
    \item No resistance to expansion of fuel.
    \item No creep deformation of fuel.
    \item Isotropic fuel properties.
\end{itemize}

The cladding and fuel deformation models in FRACAS-I are described
below.

\subsubsection{Cladding Deformation Model}\label{section:cladding-deformation-model}

The rigid pellet cladding deformation subcode (FRACAS-I) consists of four sets of models, each used
independently.
\\
\\
Deformation and stresses in the cladding in the open gap regime are computed using a model which
considers a thick wall cylindrical shell with specified internal and external pressures and a
prescribed uniform temperature.
\\
\\
Calculations for the closed gap regime are made using a model which considers a cylindrical shell
with prescribed external pressure and a prescribed radial displacement of the cladding inside
surface. The prescribed displacement is obtained from the fuel expansion models (including swelling)
described later in this section. Further, since no slippage is assumed when the fuel and cladding
are in contact, the axial expansion of the fuel is transmitted directly to the cladding, and hence,
the change in axial strain in the shell is also prescribed.
\\
\\
The decision whether the fuel-cladding gap is open or closed is made by considering the relative
movement of the cladding inside surface and the fuel outside surface. At the completion of the
FRACAS-I analysis, either a new fuel-cladding gap size or a new fuel-cladding interfacial pressure
and the elastic-plastic cladding stresses and strains are obtained.
\\
\\
Two additional models are used to compute changes in yield stress with work-hardening, given a
uniaxial stress-strain curve. This stress-strain curve is obtained from the updated MATPRO
properties. The first model computes the effective total strain and new effective plastic strain,
given a value of effective stress and the effective plastic strain at the end of the last loading
increment. The second model computes the effective stress, given an increment of plastic strain and
the effective plastic strain at the end of the last loading increment. Depending on the
work-hardened value of yield stress, loading can be either elastic or plastic, and unloading is
constrained to occur elastically.  (Isotropic work-hardening is assumed in these calculations.)
These four sets of models are described below.
\\
\\
The determination of whether or not the fuel is in contact with the cladding is made by comparing
the radial displacement (delta change) of the fuel surface ($u_{r}^{fuel}$) with the radial
displacement (delta change) that would occur in the cladding ($u_{r}^{clad}$) due to the prescribed
external (coolant) pressure and the prescribed internal (fission and fill gas) pressure. The free
radial displacement of the cladding is obtained using Equation~\ref{eq:total_strains_hooks_law}.
The following expression is used to determine if fuel-cladding contact has occurred:

\begin{equation}
    \label{eq:fuel_clad_contact_criteria}
    u_{r}^{fuel} \geq u_{r}^{clad} + \delta
\end{equation}

Where,
\ind{   $\delta$  =  as-fabricated fuel-cladding gap size \sinum{m} }

If Equation~\ref{eq:fuel_clad_contact_criteria} is satisfied, the fuel is in contact with the
cladding. The loading history enters into this decision by virtue of the permanent plastic cladding
strains which are applied to the as-fabricated geometry. These plastic strains, and total effective
plastic strain, $\varepsilon^{P}$, are retained for use in subsequent calculations.
\\
\\
If the fuel and cladding displacements are such that Equation~\ref{eq:fuel_clad_contact_criteria} is
not satisfied, the fuel-cladding gap has not closed during the current step and the solution
obtained by the open gap solution is appropriate.  The current value of the fuel-cladding gap size
is then computed and is used in the temperature calculations. The plastic strain values may be
changed in the solution if additional plastic straining has occurred.
\\
\\
If Equation~\ref{eq:fuel_clad_contact_criteria} is satisfied, however, fuel and
cladding contact has occurred during the current loading increment. At the
contact interface, radial continuity requires that

\begin{equation}
    \label{eq:radial_continuity_requirement}
    u_{r}^{cladl} = u_{r}^{fuel} -  \delta
\end{equation}

while in the axial direction the assumption is made that no slippage occurs between the fuel and the
cladding. This state is referred to as ``lockup.''
\\
\\
Note that only the additional strain which occurs in the fuel after lockup has occurred is
transferred to the cladding. Thus, if $\varepsilon_{z,o}^{clad}$ is the axial strain in the cladding
just prior to contact, and $\varepsilon_{z,o}^{fuel}$ is the corresponding axial strain in the fuel,
then the no-slippage condition in the axial direction becomes

\begin{equation}
    \label{eq:no_slippage_condition_axial_direction}
    \varepsilon^{clad}_{z} - \varepsilon^{clad}_{z,0} = \varepsilon^{fuel}_{z} - \varepsilon^{fuel}_{z,0}
\end{equation}

The values of the ``prestrains'', $\varepsilon_{z,o}^{fuel}$ and $\varepsilon_{z,o}^{clad}$, are set
equal to the values of the strains that existed in the fuel and cladding at the time of
fuel-cladding gap closure and are stored and used in the cladding sequence of calculations. The
values are updated at the end of any load increment during which the fuel-cladding gap is closed.
\\
\\
After $u_{r}^{clad}$ and $\varepsilon_{z}^{clad}$ have been computed, they are used in a calculation
which considers a cylindrical shell with prescribed axial strain, external pressure, and prescribed
radial displacement of the inside surface.  After the solution is obtained, a value of the
fuel-cladding interfacial pressure is computed along with new plastic strains and stresses.
\\
\\
The open gap modeling considers a cylindrical shell loaded by both internal and external pressures.
Axisymmetric loading and deformation are assumed.  Loading is also restricted to being uniform in
the axial direction, and no bending is considered. The geometry and coordinates are shown in
Figure~\ref{fig:fuel_rod_geometry_and_coordinates}. The displacements of the midplane of the shell
are $u$ and $w$ in the radial and axial directions, respectively.

\begin{figure}
    \includegraphics[height=\figheight\textheight]{../media/image45}
    \caption{Fuel Rod Geometry and Coordinates}
    \label{fig:fuel_rod_geometry_and_coordinates}
\end{figure}

For this case, the equilibrium equations are identically satisfied by the thick wall approximation
below.

\begin{equation}
    \label{eq:thick_wall_equilibrium_stress_hoop}
    \sigma_{\theta} = \frac{r_{i}P_{i} - r_{o}P_{o}}{t}
\end{equation}

\begin{equation}
    \label{eq:thick_wall_equilibrium_stress_axial}
    \sigma_{z} = \frac{r_{i}^{2}P_{i} - r_{o}^{2}P_{o}}{ r_{o}^{2} - r_{i}^{2}}
\end{equation}

Where,
\ind{   $\sigma_{q}$   =    hoop stress                    \sinum{MPa}  }
\ind{   $\sigma_{z}$   =    axial stress                   \sinum{MPa}  }
\ind{   $r_{i}$        =    inside radius of cladding      \sinum{m}    }
\ind{   $r_{o}$        =    outside radius of cladding     \sinum{m}    }
\ind{   $P_{i}$        =    fuel rod internal gas pressure \sinum{MPa}  }
\ind{   $P_{o}$        =    coolant pressure               \sinum{MPa}  }
\ind{   $t$            =    cladding thickness             \sinum{m}    }

For membrane shell theory, the strains are related to the midplane displacements by

\begin{equation}
    \label{eq:membrane_shell_theory_hoop}
    \varepsilon_{\theta} = \frac{u}{\overline{r}}
\end{equation}

\begin{equation}
    \label{eq:membrane_shell_theory_axial}
    \varepsilon_{z} = \pdv{w}{z}
\end{equation}

Where,
\ind{   $\overline{r}$ is the radius of the midplane.} 

Strain across the thickness of the shell is allowed. In shell theory, since the radial stress can be
neglected, and since the hoop stress, $\sigma_{q}$, and axial stress, $\sigma_{z}$, are uniform
across the thickness when bending is not considered, the radial strain is due only to the Poisson
effect and is uniform across the thickness. (Normally, radial strains are not considered in a shell
theory, but plastic radial strains must be included when plastic deformations are considered.)
\\
\\
The stress-strain relations are written in incremental form as

\begin{equation}
    \label{eq:strain_relationship_hoop}
    \varepsilon_{\theta} = \frac{1}{E} \left(\sigma_{\theta} - \nu \sigma_{z}\right) + \varepsilon^{P}_{\theta} + d\varepsilon^{P}_{\theta} + \int_{T_{0}}^{T} \alpha_{\theta}dT
\end{equation}

\begin{equation}
    \label{eq:strain_relationship_axial}
    \varepsilon_{z} = \frac{1}{E} \left(\sigma_{z} - \nu \sigma_{\theta}\right) + \varepsilon^{P}_{z} + d\varepsilon^{P}_{z} + \int_{T_{0}}^{T} \alpha_{z}dT
\end{equation}

\begin{equation}
    \label{eq:strain_relationship_radial}
    \varepsilon_{r} = - \frac{\nu}{E} \left(\sigma_{\theta} - \nu \sigma_{z}\right) + \varepsilon^{P}_{r} + d\varepsilon^{P}_{r} + \int_{T_{0}}^{T} \alpha_{r}dT
\end{equation}

Where,
\ind{   $t_{o}$   =  strain-free reference temperature \sinum{K}    }
\ind{   $\alpha$  =  coefficient of thermal expansion         }
\ind{   $t$       =  current average cladding temperature \sinum{K} }
\ind{   $e$       =  modulus of elasticity                    }
\ind{   $\nu$     =  poisson's ratio                          }

The terms $\varepsilon_{\theta}^P$, $\varepsilon_{z}^P$, and $\varepsilon_{r}^P$ are the plastic
strains at the end of the last load increment, and $d\varepsilon_{\theta}^P$, $d\varepsilon_{z}^P$,
and $d\varepsilon_{r}^P$ are the additional plastic strain increments which occur due to the new
load increment.
\\
\\
The magnitude of the additional plastic strain increments is determined by the effective stress and
the Prandtl-Reuss flow rule, expressed as

\begin{equation}
    \label{eq:prandtl_reuss_effective_stress}
    \sigma_{e} = \frac{1}{\sqrt{2}} \left[ \left( \sigma_{\theta} - \sigma_{z} \right)^{2}   + \left(\sigma_{z} \right)^{2} +  \left(\sigma_{\theta} \right)^{2} \right]^{\frac{1}{2}}
\end{equation}

\begin{equation}
    \label{eq:prandtl_reuss_dep_i}
    d{\varepsilon}^{p}_{i} = \frac{3}{2} \frac{d\varepsilon^{P}}{\sigma_{e}}S_{i}\;\;\text{for}\;\; i = r, \theta, z
\end{equation}

\begin{equation}
    \label{eq:prandtl_reuss_S_i}
    S_{i} = \sigma_{i} - \frac{1}{3} \left(\sigma_{\theta} + \sigma_{z}\right)\;\;\text{for}\;\; i = r, \theta, z 
\end{equation}

The solution of the open gap case proceeds as follows. At the end of the last load increment the
plastic strain components, $\varepsilon_{\theta}^P$, $\varepsilon_{z}^P$ and $\varepsilon_{r}^P$ are
known.  Also the total effective plastic strain, $\varepsilon^{P}$, is
known.
\\
\\
The loading is now incremented with the prescribed values of $P_{i}$, $P_{o}$, and $T$. The new
stresses can be determined from Equation~\ref{eq:thick_wall_equilibrium_stress_hoop} and
Equation~\ref{eq:thick_wall_equilibrium_stress_axial}, and a new value of effective stress is
obtained from Equation~\ref{eq:prandtl_reuss_effective_stress}.
\\
\\
The increment of effective plastic strain, $d\varepsilon^{P}$, which results from the current
increment of loading, can now be determined from the uniaxial stress-strain curve at the new value
of $\sigma_{e}$, as shown in Figure~\ref{fig:calculation_effective_stress_from_dep}. (The new
elastic loading curve depends on the value of $\varepsilon^{P}_{old}$.)

\begin{figure}
    \includegraphics[height=\figheight\textheight]{../media/image63}
    \caption{Calculation of Effective Stress $\sigma_{e}$ from $d\varepsilon_{P}$}
    \label{fig:calculation_effective_stress_from_dep}
\end{figure}

Once $d\varepsilon^{P}$ is determined, the individual plastic strain
components are found from Equation~\ref{eq:prandtl_reuss_dep_i}, and the
total strain components are obtained from Equations~\ref{eq:strain_relationship_hoop}
through~\ref{eq:strain_relationship_radial}.
\\
\\
The displacement of the inside surface of the shell must be determined so that a new fuel-cladding
gap width can be computed. The radial displacement of the inside surface is given by

\begin{equation}
    \label{eq:shell_inside_surface_radial_displacement}
    u\left(r_{i}\right) = \overline{r} \varepsilon{\theta}  - \frac{t}{2} \varepsilon{r}
\end{equation}

where the first term is the radial displacement of the midplane (from
Equation~\ref{eq:membrane_shell_theory_hoop}) and $\varepsilon_{r}$ is the uniform strain across the
cladding thickness, $t$.
\\
\\
The cladding thickness is computed by the equation

\begin{equation}
    \label{eq:shell_cladding_thickness}
    t = \left(1 + \varepsilon_{r}\right)t_{o}
\end{equation}

Where
\ind{   $t_{o}$  =  as-fabricated, unstressed thickness   }

The final step performed is to add the plastic strain increments to the previous plastic strain
values; that is,

\begin{subequations}
    
    \begin{equation}
        \left(\varepsilon^{P}_{\theta}\right)_{new}  = \left(\varepsilon^{P}_{\theta}\right)_{old} + d\varepsilon^{P}_{\theta} 
    \end{equation}

    \begin{equation}
            \left(\varepsilon^{P}_{z}\right)_{new}       = \left(\varepsilon^{P}_{z}\right)_{old} + d\varepsilon^{P}_{z}           
    \end{equation}

    \begin{equation}
            \left(\varepsilon^{P}_{r}\right)_{new}       = \left(\varepsilon^{P}_{r}\right)_{old} + d\varepsilon^{P}_{r}           
    \end{equation}
    
    \begin{equation}
            \left(\varepsilon^{P}\right)_{new}           = \left(\varepsilon^{P}\right)_{old} + d\varepsilon^{P}_{r}               
    \end{equation}
    \label{eq:plastic_strains_old_plus_delta}
\end{subequations}

%\begin{equation}
%    \label{eq:plastic_strains_old_plus_delta}
%        \begin{aligned}
%            \left(\varepsilon^{P}_{\theta}\right)_{new}  = \left(\varepsilon^{P}_{\theta}\right)_{old} + d\varepsilon^{P}_{\theta} \\
%            \left(\varepsilon^{P}_{z}\right)_{new}       = \left(\varepsilon^{P}_{z}\right)_{old} + d\varepsilon^{P}_{z}           \\
%            \left(\varepsilon^{P}_{r}\right)_{new}       = \left(\varepsilon^{P}_{r}\right)_{old} + d\varepsilon^{P}_{r}           \\
%            \left(\varepsilon^{P}\right)_{new}           = \left(\varepsilon^{P}\right)_{old} + d\varepsilon^{P}_{r}               \\
%        \end{aligned}
%\end{equation}
These values are used for the next load increment.
\\
\\
Thus, all the stresses and strains can be computed directly, since in this case the stresses are
determinate. In the case of the fuel-driven cladding displacement, the stresses depend on the
displacement, and such a straightforward solution is not possible.
\\
\\
The closed gap modeling considers the problem of a cylindrical shell for which the radial
displacement of the inside surface and axial strain are prescribed. Here the stresses cannot be
computed directly since the pressure at the inside surface (the fuel-cladding interfacial pressure)
must be determined as part of the solution.
%\\
%\\
As in the open gap modeling, the displacement at the inside surface is
given by

\begin{equation}
    \label{eq:inside_surface_displacement_closed_gap}
    u\left(r_{i}\right) = u - \frac{t}{2} \varepsilon_{r}
\end{equation}

Where,
\ind{   $u$  =  radial displacement of the midplane   }

From Equation~\ref{eq:membrane_shell_theory_hoop}, $u = r\varepsilon_{\theta}$ and

\begin{equation}
    \label{eq:radial_displacement_r_i}
    u\left(r_{i}\right) = \overline{r}\varepsilon_{\theta} - \frac{t}{2} \varepsilon_{r}
\end{equation}

Thus, prescribing the displacement of the inside surface of the shell is equivalent to a
constraining relation between $\varepsilon_{\theta}$and $\varepsilon_{i}$. As before, Hooke's law is
taken in the form

\begin{equation}
    \label{eq:hooks_law_shell_contrained_hoop}
    \varepsilon_{\theta} = \frac{1}{E} \left( \sigma{\theta}-\nu\sigma_{z} \right)+\varepsilon_{z}^P+d\varepsilon_{z}^P+\int\limits_{T_0}^T \alpha_{z}dT
\end{equation}

\begin{equation}
    \label{eq:hooks_law_shell_contrained_axial}
    \varepsilon_{z} = \frac{1}{E} \left( \sigma{z}-\nu\sigma_{\theta} \right)+\varepsilon_{z}^P+d\varepsilon_{z}^P+\int\limits_{T_0}^T \alpha_{z}dT
\end{equation}

\begin{equation}
    \label{eq:hooks_law_shell_contrained_radial}
    \varepsilon_{r} = -\frac{\nu}{E} \left( \sigma_{\theta}-\sigma_{z} \right)+\varepsilon_{r}^P+d\varepsilon_{r}^P+\int\limits_{T_0}^T \alpha_{r}dT
\end{equation}

Use of Equations~\ref{eq:radial_displacement_r_i} and~\ref{eq:hooks_law_shell_contrained_radial} in
Equation~\ref{eq:hooks_law_shell_contrained_hoop} results in a relation between the stresses
$\sigma_{\theta}$ and $\sigma_{z}$, and the prescribed displacement $u(r_{i}$):

\begin{equation}
    \label{eq:relationship_hoop_axial_stress_and_displacement_ur}
    \begin{aligned}
        \frac{u(r_i)}{\bar{r}}+\frac{1}{2} \frac{1}{2 \bar{r}} & \left[ \varepsilon_{r}^P+d\varepsilon_{r}^P+\int\limits_{T_0}^T \alpha dT \right]-\left[ \varepsilon_{\theta}^P+d\varepsilon_{\theta}^P+\int\limits_{T_0}^T \alpha dT \right] = \\
        & \frac{1}{E} \left[ \left( 1+ \frac{\nu t}{2 \bar{r}} \right) \sigma_{\theta}+\nu \left( \frac{t}{2 \bar{r}}-1 \right) \sigma_{z} \right]
    \end{aligned}
\end{equation}

Equations~\ref{eq:hooks_law_shell_contrained_axial}
and~\ref{eq:relationship_hoop_axial_stress_and_displacement_ur} are now a pair of simultaneous
algebraic equations for the stresses $\sigma_{\theta}$~and $\sigma_{z}$, which may be written as

\begin{equation}
    \label{eq:simultaneous_stress_hoop_axial}
    \mqty[ A_{11}&A_{12}\\A_{21}&A_{22}]\mqty[ \sigma_{\theta} \\ \sigma_{\theta}]=\mqty[ B_{1} \\ B_{2}]
\end{equation}

Where,
\ind{   $A_{11} $  = $ 1+ \frac{\nu t}{2 \bar{r}} $                }
\ind{   $A_{12} $  = $ \nu \left( \frac{t}{2 \bar{r}} -1 \right) $ }
\ind{   $A_{21} $  = $ -\nu $                                      }
\ind{   $A_{22} $  = $ 1 $                                         }
\ind{   $B_{1}  $  = $  \left( E \frac{u(r_i)}{\bar{r}} +\frac{Et}{4 \bar{r}} \left[ \varepsilon_{r}^P+d\varepsilon_{r}^P+\int\limits_{T_0}^T \alpha dT \right] \right)-E \left[ \varepsilon_{\theta}^P+d\varepsilon_{\theta}^P+\int\limits_{T_0}^T \alpha dT \right] $}
\ind{   $B_{2}  $  = $ E \left( \varepsilon_z- E\varepsilon_{z}^P+d\varepsilon_{z}^P+\int\limits_{T_0}^T \alpha dT \right)$ }

Then the stresses can be written explicitly as

\begin{equation}
    \label{eq:hoop_stress_matrix_soln}
    \sigma_{\theta}=\frac{B_{1}A_{22}-B_{2}A_{12}}{A_{11}A_{22}-A_{12}AA_{21}}
\end{equation}

\begin{equation}
    \label{eq:axial_stress_matrix_soln}
    \sigma_{z}=\frac{B_{2}A_{11}-B_{1}A_{12}}{A_{11}A_{22}-A_{12}AA_{21}}
\end{equation}

These equations relate the stresses to $u(r_{i})$ and $\varepsilon_{z}$, which are prescribed, and
to \detheta, \dez, and \der, which are to be determined. The remaining equations which must be
satisfied are

\begin{equation}
    \label{eq:effective_hoop_stress}
    \sigma_e=\frac{1}{\sqrt{2}} \left[ (\sigma_{\theta}-\sigma_{z})^2+(\sigma_{z})^2+(\sigma_{\theta})^2 \right] ^{\frac{1}{2}}
\end{equation}

\begin{equation}
    \label{eq:plastic_strain_increment_dep}
    d \varepsilon^P=\frac{\sqrt{2}}{3} \left[ (d \varepsilon_{r}^{P}-d \varepsilon_{\theta}^{P})^2+(d \varepsilon_{\theta}^{P}-d \varepsilon_{z}^{P})^2+(d \varepsilon_{z}^{P}-d \varepsilon_{r}^{P})^2 \right] ^{\frac{1}{2}}
\end{equation}

and the Prandtl-Reuss flow equations (defined in Equation~\ref{eq:prandtl_reuss_dep_i}):

\begin{equation}
    \label{eq:delta_effective_strain_hoop}
    d \varepsilon_{\theta}^{P}= \frac{3}{2} \frac{d \varepsilon^P}{\sigma_e} \left[ \sigma_{\theta} - \frac{1}{3}(\sigma_{\theta}+\sigma_{z} )\right]
\end{equation}

\begin{equation}
    \label{eq:delta_effective_strain_axial}
    d \varepsilon_{z}^{P}= \frac{3}{2} \frac{d \varepsilon^P}{\sigma_e} \left[ \sigma_{z} - \frac{1}{3}(\sigma_{\theta}+\sigma_{z} )\right]
\end{equation}

\begin{equation}
    \label{eq:delta_effective_strain_radial}
    d \varepsilon_{r}^{P}= -d \varepsilon_{\theta}^{P}-d \varepsilon_{z}^{P}
\end{equation}
\\
\\
The effective stress, $\sigma_{e}$, and the plastic strain increment, $d\varepsilon^{P}$, must, of
course, be related by the uniaxial stress-strain law. Equations~\ref{eq:hoop_stress_matrix_soln}
through~\ref{eq:delta_effective_strain_radial} must be simultaneously satisfied for each loading
increment.
\\
\\
As discussed in Section~\ref{section:general-theory-and-method-of-solution}, a straightforward
numerical solution to these equations can be obtained using the method of successive elastic
solutions. By this method, arbitrary values are initially assumed for the increments of plastic
strain, and Equations~\ref{eq:hoop_stress_matrix_soln}
through~\ref{eq:delta_effective_strain_radial} are used to obtain improved estimates of the plastic
strain components.  The following steps are performed for each increment of load:
\\
\\
Values of \detheta, \dez , \der and are assumed. Then, $d\varepsilon^{P}$ is computed from
Equation~\ref{eq:plastic_strain_increment_dep} and the effective stress is obtained from the
stress-strain curve at the value of $d\varepsilon^{P}$.
\\
\\
From Hooke's law, still using the assumed plastic strain increments and the prescribed values of
$u(r_{i}$) and $\varepsilon_{z}$, values for the stresses can be obtained from
Equations~\ref{eq:hoop_stress_matrix_soln} and~\ref{eq:axial_stress_matrix_soln}.
\\
\\
New values for \detheta, \dez, \der and are now computed from the Prandtl-Reuss relations,

\begin{equation}
    \label{eq:new_value_dep_i_prandtl_reuss_relation}
    d \varepsilon_i^P=\frac{3}{2} \frac{d \varepsilon^P}{\sigma_e} \left[ \sigma_i- \frac{1}{3} \left( \sigma_{\theta}+\sigma_z \right) \right] \text{ for i= r, }\theta \text{, z}
\end{equation}

The old and new values of \detheta, \dez , \der and are compared and the process continued until
convergence is obtained.
 \\
 \\
Once convergence has been obtained, the fuel-cladding interfacial pressure is computed from the
following thick wall approximation equation.

\begin{equation}
    \label{eq:thick_wall_approximation_interfacial_pressure}
    P_{int}=\frac{t\sigma_{\theta}+r_{o}P_{o}}{r_i}
\end{equation}
\\
\\
When steps 1 through 5 have been accomplished, the solution is complete, provided that the
fuel-cladding interface pressure is not less than the local gas pressure.
\\
\\
However, due to unequal amounts of plastic straining in the hoop and axial directions upon
unloading, the fuel-cladding interfacial pressure as obtained in step 5 is often less than the gas
pressure even though the fuel-cladding gap has not opened. When this situation occurs, the
frictional ``locking'' (which is assumed to constrain the cladding axial deformation to equal the
fuel axial deformation) no longer exists. The axial strain and stress adjust themselves so that the
fuel-cladding interfacial pressure equals the gas pressure, at which point the axial strain is again
``locked.'' Thus, upon further unloading, the axial strain and the hoop and axial stresses
continually readjust themselves to maintain the fuel-cladding interfacial pressure equal to the gas
pressure until the fuel-cladding gap opens. Since the unloading occurs elastically, a solution for
this portion of the fuel-cladding interaction problem can be obtained directly as discussed below.
\\
\\
Since the external pressure and the fuel-cladding interfacial pressure
are known, the hoop stress is obtained from Equation~\ref{eq:thick_wall_approximation_interfacial_pressure} as

\begin{equation}
    \label{eq:hoop_stress_calc}
    \sigma_{\theta} = \frac{r_{i}P_{int}-r_{o}P_{o}}{t}
\end{equation}

From Equation~\ref{eq:radial_displacement_r_i}, the following expression
can be written:

\begin{equation}
    \label{eq:hoop_strain_calc}
    \varepsilon_{\theta} = \frac{u^{fuel}_{r} - \delta + \frac{t}{2}\varepsilon}{\overline{r}}
\end{equation}

Substitution of $\varepsilon_{\theta}$~and $\varepsilon_{r}$, as given by
Equations~\ref{eq:hooks_law_shell_contrained_hoop} and~\ref{eq:hooks_law_shell_contrained_radial},
into Equation~\ref{eq:hoop_strain_calc} results in an explicit equation for
$\sigma_{z}$:

\begin{equation} 
    \nu r_i \sigma_z=\left(r+\nu \frac{t}{2} \right) \sigma_{\theta}+rE \left( \int \alpha dT+d \varepsilon_{\theta}^P \right)-\frac{t}{2}E \left( \int \alpha dT+d \varepsilon_r^P \right)-Eu(r_i)
\end{equation}

in which $\sigma_{\theta}$ is known from Equation~\ref{eq:hoop_stress_calc}.  With $\sigma_{z}$ and
$\sigma_{\theta}$~known, the strains may be computed from Hooke's law,
Equations~\ref{eq:hooks_law_shell_contrained_hoop}
through~\ref{eq:hooks_law_shell_contrained_radial}.  This set of equations is automatically invoked
whenever $P_{int}$ is computed to be less than the local gas pressure.

As in the open gap modeling, the last step is to set the plastic strain components and total
effective strain equal to their new values by adding in the computed increments \de{i} and \de{}.
\\
\\
The stress-strain modeling is used to relate stress and plastic strain, taking into consideration
the direction of loading and the previous plastic deformation. A typical stress-strain curve is
shown in Figure~\ref{fig:idealized_stress_strain_behavior}.  This curve presents the results of a
uniaxial stress-strain experiment and may be interpreted beyond initial yield as the focus of
work-hardened yield stresses. The equation of the curve is provided by the updated MATPRO properties
at each temperature given in
Section~\ref{section:updated-matpro-cladding-mechanical-properties-models}.
\\
\\
To use this information, the usual idealization of the mechanical behavior of metals is made. Thus,
linear elastic behavior is assumed until a sharply defined yield stress is reached, after which
plastic (irrecoverable) deformation occurs. Unloading from a stress state beyond the initial yield
stress, $\sigma_{y}^{o}$ , is assumed to occur along a straight line having the elastic modulus for
its slope. When the (uniaxial) stress is removed completely, a residual plastic strain remains, and
this completely determines the subsequent yield stress. That is, when the specimen is loaded again,
loading will occur along line BA in Figure~\ref{fig:idealized_stress_strain_behavior} and no
additional deformation will occur until point A is again reached. Point A is the subsequent yield
stress. If $\sigma =f(\varepsilon)$ is the equation of the plastic portion of the stress-strain
curve (YAC), then for a given value of plastic strain, the subsequent yield stress is found by
simultaneously solving the pair of equations.

\begin{figure}
    \includegraphics[height=\figheight\textheight]{../media/image94}
    \caption{Idealized Stress Strain Behavior}
    \label{fig:idealized_stress_strain_behavior}
\end{figure}

\begin{equation} 
    \label{eq:sigma_f_e}
    \sigma=f\left(\varepsilon\right)
\end{equation} 

\begin{equation} 
    \label{eq:sigma_E_e}
    \sigma=E(\varepsilon-\varepsilon^P)
\end{equation}

Which may be written as

\begin{equation}
    \label{eq:sigma_rewritten}
    \sigma=f \left( \frac{\sigma}{E}+\varepsilon^P \right)
\end{equation}

This nonlinear equation may be solved efficiently by using an iteration
scheme:

\begin{equation} 
    \label{eq:initial_stress_iterate}
    \sigma^{m+1}=f \left( \frac{\sigma^m}{E}+\varepsilon^P \right) \text{ for $m$ = 0, 1,2...}
\end{equation} 

The initial iterate, $\sigma^{m}$, is arbitrary, and without loss of generality, is taken as 34.5
MPa. For any monotonically, increasing stress-plastic strain relation, the iteration scheme in
Equation~\ref{eq:initial_stress_iterate} will converge uniformly and absolutely.
\\
\\
The computations of the stress-strain modeling are described below. The first computes strain as a
function of plastic strain, temperature, and stress. The second computes stress as a function of
plastic strain, temperature, and plastic strain increments.
\\
\\
Values of plastic strain, $\varepsilon^{P}$, temperature, and stress are used as follows:

\begin{enumerate}
\item
  For a given temperature, $\sigma= f(\varepsilon)$ is obtained from the updated
  MATPRO properties given in Section~\ref{section:updated-matpro-cladding-mechanical-properties-models}.
\item
  The yield stress $\sigma_{y}$ for given $\varepsilon^{P}$ is obtained from Equation~\ref{eq:initial_stress_iterate}.
\item
  For a given value of stress, $\sigma$,
    \begin{itemize}
        \item if $\sigma \leq \sigma_{y}$
            \begin{equation} 
                \begin{aligned}
                    \varepsilon           & = \frac{\sigma}{E} + \varepsilon^{P} \\
                    \varepsilon_{P}^{new} & = \varepsilon_{old}^{P}
                \end{aligned}
            \end{equation} 
        \item if $\sigma > \sigma_{y}$
            \begin{equation}
                \begin{aligned}
                    \varepsilon           & = f(\sigma)                      \\
                    \varepsilon_{new}^{P} & = \varepsilon - \frac{\sigma}{E} \\
                    d\varepsilon^{P}      & = \varepsilon_{new}^{P} - \varepsilon_{old}^{P}
                \end{aligned}
            \end{equation}
    \end{itemize}
\end{enumerate}

Where $E$ is computed using the correlation in the material
properties handbook (Luscher and Geelhood 2014).

Values of plastic strain, $\varepsilon^{P}$, temperature, and
plastic strain increment, $d\varepsilon^{P}$, are used as
follows:
\begin{enumerate}
    \item For a given temperature, $\sigma = f(\varepsilon)$ is obtained from the updated MATPRO
        properties given in Section~\ref{section:updated-matpro-cladding-mechanical-properties-models}.
    \item The yield stress $\sigma_{y}$ for given $\varepsilon^{P}$ is obtained from
        Equation~\ref{eq:initial_stress_iterate}.
    \item Given $d\varepsilon^{P}$ (see Figure~\ref{fig:computing_stress}).
\end{enumerate}

\begin{equation} 
    \varepsilon^{P}_{new} = \varepsilon^{P}_{old} + d\varepsilon^{P} 
\end{equation} 

\begin{figure}
    \includegraphics[height=\figheight\textheight]{../media/image101}
    \caption{Computing Stress}
    \label{fig:computing_stress}
\end{figure}

Since $d\varepsilon^{P} > 0$, the new value of stress and strain must lie on the plastic portion of
the stress-strain curve $\sigma = f(\varepsilon) $. So, $\sigma$ and $\varepsilon$ are obtained by
performing a simultaneous solution, as before.

\subsubsection{Updated MATPRO Cladding Mechanical Properties Models}\label{section:updated-matpro-cladding-mechanical-properties-models}

The mechanical properties of fuel rod Zircaloy cladding are known to change with irradiation because
of damage induced from the fast neutron fluence. The changes are similar to cold-working the
material because dislocation tangles are created that tend to both strengthen and harden the
cladding while decreasing the ductility. In addition to the fast fluence effects, the presence of
excess hydrogen in the Zircaloy, in the form of hydrides, may also affect the mechanical properties.
\\
\\
Three MATPRO models have been modified to account for the high fast
neutron fluence levels, temperature, and strain rate. Those models are
a) the strength coefficient in CKMN, b) the strain hardening exponent in
CKMN, and c) the strain rate exponent in CKMN.

\subsubsection{Strength Coefficient, $K$}\label{section:strength-coefficient-k}

The strength coefficient, $K$, has been modified from MATPRO and is a function of temperature,
fast neutron fluence, cold work, and alloy composition. The strength coefficient has not been found
to be a function of hydrogen concentration. The fluence dependency, $K(\Phi)$, has been modified
from MATPRO to better fit the high burnup data. The models for the strength coefficients of
Zircaloy-2 and Zircaloy-4 are given below.

\begin{equation}
    \label{eq:strength_coefficient_Zirc2_4}
    K= \frac{K(T)\cdot (1+K(CW)+K(\Phi))}{K(Zry)}
\end{equation}

Where,

\begin{equation}
    \label{eq:stength_coefficient_Zirc2_4_K_T}
    K(T) = 
    \begin{cases}
            \num{1.17628E9}+\num{4.54859E5}T-\num{3.28185E3}T^{2}                           & T<\SInum{750}{K}                       \\
            \num{2.522488E6}\exp\left( \frac{\num{2.850027E6}}{T^2} \right)                  & \SInum{750}{K}<T<\SInum{1090}{K}  \\
            \num{1.841376039E8}-\num{1.434544E5}T                                           & \SInum{1090}{K}<T<\SInum{1255}{K} \\
            \num{4.330E7}-\num{6.685E4}T+\num{3.7579E1}T^2 - 7.33 \times 10^{-3}T^3         & \SInum{1255}{K}<T<\SInum{2100}{K}
    \end{cases}
\end{equation}

\begin{equation}
    \label{eq:stength_coefficient_Zirc2_4_K_CW}
    K(CW) = 0.546 CW
\end{equation}

\begin{equation}
    \label{eq:stength_coefficient_Zirc2_4_K_phi}
    K(\Phi) = 
    \begin{cases}
        (0.1464+1.464 \times 10^{-25} \phi)f(CW,T) & \Phi< \SInum{0.1E25}{n/m^2}                         \\
        2.928 \times 10^{-26} \phi                 & \SInum{0.1E25}{n/m^2}< \Phi< \SInum{2E25}{n/m^2} \\
        0.53236+2.6618 \times 10^{-27} \phi        & \SInum{2E25}{n/m^2}< \Phi<\SInum{12E25}{n/m^2}
    \end{cases}
\end{equation}

\begin{equation}
    \label{eq:stength_coefficient_Zirc2_4_f_CW_T}
    f(CW,T)=2.25\exp(-20  CW)  min \left[ 1, \exp \left( \frac{T-550}{10} \right) \right] +1
\end{equation}

Where,
\ind{   $K(Zry) $  =  1.305 for Zircaloy-2                                                   }
\ind{   $K(Zry) $  =  1 for Zircaloy-4                                                       }
\ind{   $T $       =  temperature          \sinum{K}                                               }
\ind{   $CW $      =  cold work            \sinum{unitless-ratio-of-areas} (valid from 0 to 0.75)  }
\ind{   $\Phi $    =  fast neutron fluence \sinum{n/m^2} (E \textgreater{} \SInum{1}{MeV})  }

The effective cold work and fast neutron fluence used to calculate the strength coefficient, $K$,
can be reduced by annealing if the time and/or temperature are high enough. FAST uses the MATPRO
model, CANEAL, to calculate the effective cold work and fast neutron fluence at each time step using
the following equations.

\begin{equation}
    CW_i=CW_{i-1}\exp \left[-1.504 \left(1.504(1+\num{2.2E-25} \phi+{i-1} \right) (t) \exp \left( \frac{-\num{2.33E18}}{T^6} \right) \right]
\end{equation}

\begin{equation}
    \phi_i= \frac{\num{1E20}}{\num{2.49E-6}(t) \exp \left( \frac{-\num{5.35E23}}{T^8} \right)+\frac{10^{20}}{\phi_{i-1}}}
\end{equation}

Where,
        \ind    {$CW_{i-1}$, and $CW_{i}$ = the effective cold work for strength coefficient at the start and end of the time step, respectively \sinum{unitless-ratio-of-areas}  }
        \ind{   $\phi_{I }$, and $\phi_{i-1}$      =     effective fast neutron fluence for strength coefficient at the start and end of the time step, respectively \sinum{n/m^{2}} }
        \ind{   $t$                                      =     time step size                                                                    \sinum{s}                        }
        \ind{   $T$                                      =     cladding temperature                                                              \sinum{K}                        }


\subsubsection{Strain-Hardening Exponent, $n$}\label{section:strain-hardening-exponent-n}

The strain-hardening exponent, $n$, has been modified from MATPRO to better fit the high burnup data
and is a function of temperature, fast neutron fluence, and alloy composition. The strain hardening
exponent has not been found to be a function of hydrogen concentration.  The models for the strain
hardening exponents of Zircaloy-2 and Zircaloy-4 are given below.

\begin{equation}
    \label{eq:strain_hardening_n}
    n = \frac{n(T) n(\Phi )}{n(Zry)}
\end{equation}

Where,

\begin{equation}
    n(T) = 
    \begin{cases}
        0.11405                                                                                         & T<\SInum{419.4}{K}            \\
        -\num{9.490E-2}+\num{1.165E-3}T-\num{1.992E-6} T^2 +\num{9.588E-10} T^3 & \SInum{419.4}{K}<T<\SInum{1099.0772}{K} \\
        -0.22655119+\num{2.5E-4}T                                                                 & \SInum{1099.0772}{K}<T<\SInum{1600}{K}  \\
        0.17344880                                                                                      & T>\SInum{1600}{K}
    \end{cases}
\end{equation}

\begin{equation}
    n(\Phi) = 
    \begin{cases}
            1.321+\num{0.48E-25} \phi                   & \Phi<\SInum{0.1E25}{n/m^2}\\
            1.369+\num{0.096E-25} \phi                  & \SInum{0.1E25}{n/m^2}< \Phi< \SInum{2E25}{n/m^2}\\
            1.5435+\num{0.008727E-25} \phi              & \SInum{2E25}{n/m^2}< \Phi<\SInum{7.5E25}{n/m^2}\\
            1.608953                                    & \Phi>\SInum{7.5E25}{n/m^2}
    \end{cases}
\end{equation}

Where,
\ind{   $n(Zry)$  =  1.0 for Zircaloy-4                                                    }
\ind{   $n(Zry)$  =  1.6 for Zircaloy-2                                                    }
\ind{   $T$       =  temperature          \sinum{K}                                              }
\ind{   $\Phi$    =  fast neutron fluence \sinum{n/m^{2}} (E \SInum{1}{MeV}) }

The effective fast neutron fluence used to calculate the strain-hardening exponent, n, can be
reduced by annealing if the time or temperature, or both, are high enough. FAST uses the MATPRO
model, CANEAL, to calculate the effective fast neutron fluence at each time step using the following
equation.

Where,
\ind{   $\phi_{i}$, and $\phi_{i-1}$       =     effective fast neutron fluence for strain hardening exponent at the start and end of the time step, respectively  \sinum{n/m^{2}} }
\ind{   $t$                                      =     time step size                                                                             \sinum{s}                       }
\ind{   $T$                                      =     cladding temperature                                                                       \sinum{K}                      }

\subsubsection{Strain Rate Exponent}\label{section:strain-rate-exponent}

The strain rate exponent, $m$, has been modified from MATPRO to better fit the high burnup data and
is given by a function of temperature only as described in the equation below.

\begin{equation}
    \label{eq:strain_rate_exponent_Zirc}
    m =
    \begin{cases}

        0.015                           & T<\SInum{750}{K}      \\
        \num{7.458E-4}T-0.544338        & \SInum{750}{K}<T<\SInum{800}{K} \\
        \num{3.24124E-4}T-0.20701       & T>\SInum{800}{K}

    \end{cases}
\end{equation}

Where,
\ind{   $m$   =  strain rate exponent }
\ind{   $T$   =  temperature \sinum{K}      }


The impact of the strain rate exponent on yield stress is to increase the yield strength with
increasing strain rate, but the effect is not large. For example, increasing the strain rate from
\SInum{1E-4}{1/s} to \SInum{1.0}{1/s} will increase the yield strength by about 15\percent.
\\
\\
\paragraph{Assembled Model}

Tensile strength, yield strength, and strain are calculated using the same relationships in MATPRO's
CMLIMT subroutine with slight modifications. The true ultimate strength is calculated using

\begin{equation}
    \label{eq:true_ultimate_strength}
    \sigma = K\left(\frac{\dot{\varepsilon}}{\num{1E-3}}\right)^{m}\varepsilon^{n}_{p+e}
\end{equation}


Where,
\ind{    $\sigma$                          =  true ultimate strength \sinum{MPa}                            }
\ind{    $K$                               =  strength coefficient \sinum{MPa}                              }
\ind{    $\dot{\varepsilon}$               =  strain rate \sinum{unitless}                                  }
\ind{    $m$                               =  strain rate sensitivity constant from MATPRO \sinum{unitless} }
\ind{    $\varepsilon\textsubscript{p+e}$  =  true strain at maximum load \sinum{unitless}                  }
\ind{    $n$                               =  strain hardening exponent \sinum{unitless}                    }

This is a change in the original MATPRO model in that the true strain at maximum load in the
original model was set equal to the strain hardening exponent. This change was made to better fit
the ultimate tensile strength data.
\\
\\
The CMLIMT subroutine equations predicting true yield strength and true strain at yield remain
unchanged.
\\
\\
This model is applicable over the following ranges with an uncertainty (standard deviation) on yield
and tensile strength of approximately 17\percent relative. 

\begin{itemize}
    \item   cladding temperature: 560 to \SInum{700}{K}         
    \item    oxide corrosion thickness: 0 to \SInum{100}{\mu m}         
    \item    excess hydrogen level: 0 to \SInum{650}{ppm}         
    \item    strain rate: \num{1E-4} to \SInum{1E-5}{1/s}         
    \item    fast neutron fluence: 0 to \SInum{12E25}{n/m^2}         
    \item    Zircaloy: cold work and stress relieved         
\end{itemize}

A plot of predicted vs. measured yield stress is shown in
Figure~\ref{fig:predicted_vs_measured_yield_stress_PNNL_database}.  Further data comparisons are
shown in \cite{ref:Geelhood2008a}.

\begin{figure}
    \includegraphics[height=\figheight\textheight]{../media/Zircaloy_yield_stress} 
    \caption{Predicted vs. Measured Yield Stress from the PNNL Database ($\SInum{293}{K} \leq T \leq \SInum{755}{K}$, $0 \leq \Phi \leq  \SInum{14E25}{n/m^2}$, $0 \leq H_{ex} \leq \SInum{850}{ppm}$)}
    \label{fig:predicted_vs_measured_yield_stress_PNNL_database}
\end{figure}

\subsubsection{Rigid Pellet Fuel Deformation in FRACAS-I}\label{section:rigid-pellet-fuel-deformation-in-fracas-i}

This section describes the analytical models used to compute fuel deformation in FRACAS-I. Models
are available to calculate length change and fuel radial displacement. Relocation is also considered
in FRACAS-I and is also discussed in this section. The effect of relocation is included in the
thermal response; however, no hard contact between the fuel and cladding (and therefore no
mechanical interaction) is allowed until the other fuel expansion components (swelling and thermal
expansion) recover 50\percent of the original relocated pellet radius.  Therefore, the rigid pellet
for mechanical analyses, and that also controls contact conductance, includes 50\percent of the
original relocated pellet radius as well as the other pellet expansion components.
\\
\\
The assumptions made with respect to fuel deformation in FRACAS-I are that no pellet deformation is
induced by fuel-cladding contact stress or thermal stress and that free-ring thermal expansion
applies. Each individual fuel ring is assumed to expand without restraint from any other ring, and
the total expansion is the sum of the individual expansions.
%
\subsubsection{Radial Deformation}\label{section:radial-deformation}

Radial deformation of the pellet due to thermal expansion, irradiation swelling, and densification
is calculated with a free-ring expansion model. The governing equation for this model is

\begin{equation}
    \label{eq:pellet_radial_deformation_model}
    R_{H} = \sum_{i = 1}^{N} \Delta r_{i} \left[1 + \alpha_{T_{i}} \left(T_{i} - T_{ref} \right) + \varepsilon_{i}^{s} + \varepsilon_{i}^{d}\right]
\end{equation}

Where,
\ind{   $R_{H }$                   =   hot-pellet radius                                               \sinum{m}        }
\ind{   $\alpha_{T_{i}}$           =   coefficient of thermal expansion of the i-th radial temperature \sinum{1/K}      }
\ind{   $T_{i }$                   =   average temperature of i-th radial ring                         \sinum{K}        }
\ind{   $T_{ref }$                 =   reference temperature                                           \sinum{K}        }
\ind{   $r_{i }$                   =   width of \iteration{i} radial ring                                       \sinum{m}        }
\ind{   $N$                        =   number of annular rings                                                    }
\ind{   $\varepsilon^{s}_{i}$      =   swelling strain                                                 (positive) }
\ind{   $\varepsilon^{s}_{d}$      =   densification strain                                            (negative) }

The fuel densification and solid fuel swelling models are briefly discussed. The densification
asymptotically approaches the (input) ultimate density change, typically over a local (node-average)
burnup of approximately \SInum{5}{GWd/MTU}. Solid fuel swelling is considered only as the athermal
swelling associated with solid fission product accumulation. It is linear with local (node-average)
burnup, and starts following a burnup of \SInum{6}{GWd/MTU} (delayed for swelling into as-fabricated
porosity).  It then accumulates per time step at a rate equal to
0\SInum{.062}{volume\percent/GWd/MTU} up to \SInum{80}{GWd/MTU} and 0.086 volume \percent per
\sinum{GWd/MTU} beyond \SInum{80}{GWd/MTU} \cite{ref:Luscher2014b}.
\\
\\
A gasesous swelling model is included in FAST. The model is based on data from Mogensen
(\cite{ref:Mogensen1985b}) and was developed after ramp test results suggested gaseous
swelling may influence permanent cladding hoop strain in high burnup rods. The linear strain is
given as a function of temperature over the ranges given in the following equations. These models
are phased in between 40 and \SInum{50}{GWd/MTU} by applying a factor that varies linearly between 0
and 1 at 40 and \SInum{50}{GWd/MTU}, respectively.

\begin{equation}
    \frac{\Delta l}{l} = 
    \begin{cases}
        \num{4.55E-5}T-\num{4.7E-2}     & \SInum{960}{\mDC} < T < \SInum{1370}{\mDC} \\
        -\num{4.05E-5}+\num{7.40E-2}    & \SInum{1370}{\mDC} < T < \SInum{1832}{\mDC}
    \end{cases}
\end{equation}

\subsubsection{Axial Deformation}\label{section:axial-deformation}

Axial deformation of the total fuel stack takes into account the thermal, densification, and
swelling strains at each axial node. The calculation proceeds differently for flat-ended versus
dished-pellets as described below.
\\
\\
For flat-ended pellets, the volume-averaged ring axial deformation is calculated for each axial
node, and these are summed to find the total stack deformation assuming isotropic behavior. The ring
deformations account for thermal, densification, and swelling strains specific to each ring.
\\
\\
For dished pellets, the axial deformation of the ``maximum ring'' (the ring with the maximum deformed
length) per node is found, and these ``maximum ring'' deformations are summed to find the total
deformation.  Typically, the ``maximum ring'' is the innermost ring on the dish shoulder because the
deformation of the rings within the dish does not fill the dish volume, as illustrated in
Figure~\ref{fig:interpellet_void_volume}.

\begin{figure}
    \includegraphics[height=\figheight\textheight]{../media/image9}
    \caption{Interpellet Void Volume}
    \label{fig:interpellet_void_volume}
\end{figure}

\subsubsection*{Fuel Relocation}\label{section:fuel-relocation}

Fuel pellet center temperatures measured at beginning of life (BOL) in instrumented test rods have
repeatedly been found to be lower than values predicted by thermal performance computer programs
when the predicted fuel-cladding gap in operation is calculated based only on fuel and cladding
thermal expansion (\cite{ref:Lanning1982}). It has long been concluded, based on
microscopic examination of fuel cross sections (\cite{ref:Galbraith1973};
\cite{ref:Cunningham1984}), that fuel pellet cracking promotes an outward
relocation of the pellet fragments that causes additional gap closure. This process begins at BOL
and quickly reaches equilibrium. \cite{ref:Oguma1983} characterized this approach to
equilibrium based on his analysis of BOL test rod elongation data from Halden instrumented test
assemblies.
\\
\\
The fuel pellet cracking that promotes relocation is predominantly radial; however, some
circumferential components to these crack patterns exist, and these components could alter the fuel
thermal conductivity.  Thus, cracking and relocation will to some degree increase the thermal
resistance in the pellet while reducing the thermal resistance of the pellet-cladding gap by
reducing its effective size. The relocation model implicitly includes any crack effects on heat
transfer because the model is based on fuel centerline temperature data.
\\
\\
The best estimate pellet relocation model developed for GT2R2 (\cite{ref:Cunningham1984}), has been
altered for use in FAST in conjunction with the FRACAS-I mechanical model.  This model is based on
the model developed for FRAPCON-3.5. The gap closure at beginning of life was fit to the first ramp
to power data. Due to the excellent centerline temperature predictions throughout life, the
FRAPCON-3.4 pellet relocation model beyond \SInum{5}{GWd/MTU} was retained.  Data from IFA-677.1
which contained very stable pellets that exhibited little to no densification was available showing
stack elongation (which is proportional to fuel temperature) as a function of power for ramps to
power at 0.1, 0.6, 4, and \SInum{5}{GWd/MTU} (\cite{ref:Therache2005b}).  These data
demonstrated that the increase in relocation from 0 to \SInum{5}{GWd/MTU} appears to follow a
logarithmic trend. Therefore, a logarithmic function was adopted to model the relocation between 0
and \SInum{5}{GWd/MTU}.

{\color{red}There are two possible citations for Th\`{e}ache 2005 and I am not sure which one is
correct}

The gap closure due to relocation as a fraction of the as-fabricated pellet-cladding gap is given by

\begin{equation}
    \label{eq:gap_closure_function_of_asfab_gap}
    \frac{\Delta G}{G} = 
    \begin{cases}
        0.055 & \text{burnup} < 0.0937 \sinum{GWd/MTU}  \\
        0.055 + min\left(reloc, reloc  \left(0.5795 + 2447  \ln\left(\text{burnup}\right)\right)\right)  & \text{burnup} > 0.0937 \sinum{GWD/MTU}
    \end{cases}
\end{equation}

Where,
 \ind{   $\Delta G/G$   =  fraction of as fabricated gap closure due to pellet relocation (fraction) }
 \ind{   $P$           =  local power, \sinum{kW/ft}                                                        }
 \ind{   burnup        =  local burnup, \sinum{GWd/MTU}                                                     }

and,

\begin{equation}
    \label{eq:relocation_as_function_of_power}
    reloc = 
    \begin{cases}
        0.345                      &   P > 20           \\
        0.45+\left(P-20\right)/200 & 20 \leq P \leq 40  \\
        0.445                      & P > 40
    \end{cases}
\end{equation}

A plot of this model as a function of burnup and LHGR is shown in 
Figure~\ref{fig:relocation_model}. Also shown for reference is the relocation model from FRAPCON-3.4.
\\
\\
\begin{figure}
    \includegraphics[height=\figheight\textheight]{../media/UO2_relocation_model}
    \caption{Power and Burnup Dependence of the FAST-1.0 Relocation Model with the FRAPCON-3.4 Model Shown for Reference}
    \label{fig:relocation_model}
\end{figure}

The fuel-cladding gap size used in the thermal and internal pressure calculations includes the fuel
relocation, while the fuel-cladding gap size used in the mechanical calculations allows for
50\percent of the relocation to be recovered before cladding stress/strain is driven by the fuel.
