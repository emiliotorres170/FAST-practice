\section{Transient Options}\label{section:transient-options}

The development of FAST was focused around the merger of the transient capabilities from
FRAPTRAN-2.0 coupled with the steady-state capabilities of FRAPCON-4.0. A set of ``transient'' input
options allow the user to control the time-dependent thermal/mechanical code solution and transient
fuel rod models. These options are described in Section~\ref{section:transient_block}. The transient
code solution control is based on the problem time-step size (and code output intervals) and
convergence criteria. The recommended time-step size to be used is outlined in
Table~\ref{tab:recommended_time_step_sizes_for_various_transients}.

\begin{ThreePartTable}
    \begin{TableNotes}
        \footnotesize
    \item[$*$] 
        Will default to using the Steady-State solution.
    \end{TableNotes}
    \begin{longtable}[c]{c c c}
   \caption{Recommended Time Step Sizes for Various Transients}
   \label{tab:recommended_time_step_sizes_for_various_transients}\\
   \hline
    Transient/Accident           &          Period of Transient/Accident            &       time Step, s \\\hline
   \endfirsthead
   \caption[]{Recommended Time Step Sizes for Various Transients (continued)}\\ \hline
    Transient/Accident           &          Period of Transient/Accident            &       time Step, s    \\\hline
   \endhead
   \insertTableNotes
   \endlastfoot
        Steady-state equilibrium                &                     & 0.0 \tnote{$*$}     \\

        Large break loss of coolant             & Blowdown            & 0.2                 \\
        (LBLOCA)                                & Reflood             & 0.5                 \\

        Small break loss of coolant             & Prior to SCRAM      & 0.2                 \\
        (SBLOCA)                                & Adiabatic heatup    & 2.0                 \\
                                                & Quenching           & 0.5                 \\

        Reactivity-initiated accident  (RIA)    & During power pulse  & $\num{01.0E-5}$ \\
        Anticipated transient with SCRAM (ATWS) &                     & 0.2                  \\ 
                                                               
  \end{longtable}
\end{ThreePartTable}

The model options in the \textit{transient} input block define the high
temperature cladding oxidation model, the high temperature cladding
deformation (ballooning) model, and the time-step size control.
