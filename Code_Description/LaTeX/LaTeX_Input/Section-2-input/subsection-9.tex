\section{Heat Conduction and Temperature Solution}\label{section:heat-conduction-and-temperature-solution}

Once values for the heat generation, gap conductance, and cladding surface temperature have been
obtained, the complete temperature distribution in the fuel and cladding is obtained by applying the
law for heat conduction in solids in one dimension.

\subsection{One-Dimensional Radial Steady-State Heat Conduction}\label{section:one-dimensional-radial-steady-state-heat-conduction}

\begin{equation}
    \label{eq:one_d_heat_conduction_radial_ss}
    \int_{S}^{}{k(\overline{x},T)\overline{\nabla}Td\overline{s}} = \int_{V}^{}S(\overline{x})dV
\end{equation}

Where,
\ind{   $T$             =  Temperature           \sinum{K}                      }
\ind{   $\overline{x}$  =  Spatial coordinate    \sinum{m}                      }
\ind{   $S$             =  Heat source           \sinum{W/m^{3}} }
\ind{   $k$             =  Thermal conductivity  \sinum{W/m-K}                  }

\subsection{One-Dimensional Radial Transient Heat Conduction}\label{section:one-dimensional-radial-transient-heat-conduction}

Heat conduction in the radial direction in both the fuel and cladding is
described by Equation~\ref{eq:one_d_heat_conduction_radial_tr}.

\begin{equation}
    \label{eq:one_d_heat_conduction_radial_tr}
    \int_{V}^{}{\rho(\overline{x},T) c_{p}(\overline{x},T)\frac{\partial T}{\partial t}dV} = \int_{S}^{}{k(\overline{x},T)\overline{\nabla}Td\overline{s}} + \int_{V}^{}S(\overline{x})dV
\end{equation}

Where,
\ind{   $T$             =  Temperature          \sinum{K}                       }
\ind{   $\overline{x}$  =  Spatial coordinate   \sinum{m}                       }
\ind{   $t$             =  Time                 \sinum{s}                       }
\ind{   $S$             =  Heat source          \sinum{W/m^{3}}  }
\ind{   $c_{p}$         =  Specific heat        \sinum{J/kg-K}                  }
\ind{   $\rho$          =  Density              \sinum{kg/m^{3}} }
\ind{   $k$             =  Thermal conductivity \sinum{W/m-K}                   }

The first integral calculates the enthalpy change of an arbitrary infinitesimal volume, $V$, of
material, the second the heat transfer through the surface, $S$, of the volume, and the third the
heat generation within the volume. The parameters \(c_{p}\)and \(k\) are temperature dependent. The
fuel thermal conductivity is also burnup dependent. The following boundary conditions in
Equations~\ref{eq:one_d_heat_conduction_radial_BC_1} and~\ref{eq:one_d_heat_conduction_radial_BC_2}
are used with Equation~\ref{eq:one_d_heat_conduction_radial_tr}:

\begin{equation}
    \label{eq:one_d_heat_conduction_radial_BC_1}
    \pdv{T}{t}\Bigr|_{r=0} = 0
\end{equation}

\begin{equation}
    \label{eq:one_d_heat_conduction_radial_BC_2}
    T\Bigr|_{r=0} = T_{s}
\end{equation}

Where
\ind{   $r$       =  Radial position                    \sinum{m} }
\ind{   $r_{0}$   =  Outer radius of fuel               \sinum{m} }
\ind{   $T_{s}$   =  Fuel rod outer surface temperature \sinum{K} }

Equation~\ref{eq:one_d_heat_conduction_radial_tr} is numerically solved by using an implicit finite
difference approximation. The solution method is taken from the HEAT-1 code
(\cite{ref:Wagner1963}). The solution method accounts for temperature- and time-dependent
thermal properties; transient spatially varying heat generation; and melting and freezing of the
fuel and cladding.
\\
\\
With Figure~\ref{fig:fd_geometry_terms} as a reference for geometry terms, the finite
difference approximation for heat conduction is shown in Equation~\ref{eq:fd_heat_conduction}.

\begin{equation}
    \label{eq:fd_heat_conduction}
    \begin{aligned}
    \frac{\left(T^{m+1}_{n} + T^{m}_{n}\right) \left(c_{ln}h_{ln}^{V} + c_{rn}h_{rn}^{V} \right)}{\Delta t} = &
    -\left( T_{n}^{m + \frac{1}{2}} - T_{n - 1}^{m + \frac{1}{2}} \right)k_{ln}h_{ln}^{S} + \left( T_{n + 1}^{m + \frac{1}{2}} - T_{n}^{m + \frac{1}{2}} \right)k_{rn}h_{rn}^{S}\\
        &+ Q_{ln}h_{ln}^{V} + Q_{rn}h_{rn}^{V}
    \end{aligned}
\end{equation}

Where,
\ind{   $T_{n}^{m + 1}$              =     Temperature at radial node $n$ and time point $m$+1                                                           \sinum{K}                              }
\ind{   $T_{n}^{m + \frac{1}{2}}$    =     0.5 ($T_{n}^{m}$ + $T_{n}^{m+1}$)                                                                                                              }
\ind{   $\mathrm{\Delta}t$           =     Time step                                                                                                     \sinum{s}                              }
\ind{   $c_{ln}$                    =     Volumetric heat capacity on left side of node $n$                                                             \sinum{J/m^{3}-K} }
\ind{   $c_{rn}$              =     Volumetric heat capacity on right side of node  $n$                                                           \sinum{J/m^{3}-K} }
\ind{   $k_{ln}$                    =     Thermal conductivity at left side of node $n$                                                                 \sinum{W/m-K}                    }
\ind{   $k_{rn}$              =     Thermal conductivity at right side of node $n$                                                                \sinum{W/m-K}                    }
\ind{   $h_{ln}^{V}$                =     Volume weight of mesh spacing on left side of radial node $n$                                                 \sinum{m^{2}}  =     $\pi{\mathrm{\Delta}r}_{ln}\left( r_{n} - \frac{{\mathrm{\Delta}r}_{ln}}{4} \right)$                                                       }
\ind{   $h_{rn}^{V}$          =     Volume weight on right side of node \emph{n} =     $\pi{\mathrm{\Delta}r}_{rn}\left( r_{n} + \frac{{\mathrm{\Delta}r}_{rn}}{4} \right)$ \sinum{m^2}                                           }
\ind{   $h_{ln}^{S}$                =     Surface weight on left side of node \emph{n} =     $\frac{2\pi}{{\mathrm{\Delta}r}_{ln}}\left( r_{n} - \frac{{\mathrm{\Delta}r}_{ln}}{2} \right)$                                             }
\ind{   $Q_{ln}$                    =     Heat generation per unit volume for mesh spacing on left side of radial node \emph{n}                         \sinum{W/m^{3}}         }
\ind{   $Q_{rn}$              =     Heat generation per unit volume for mesh spacing on right side of radial node \emph{n}                        \sinum{W/m^{3}}         }

\begin{figure}
    \includegraphics[height=\figheight\textheight]{../media/image143}
    \caption{Description of Geometry Terms in Finite Difference Equations for 1-D Radial Heat Conduction}
    \label{fig:fd_geometry_terms}
\end{figure}

If a phase change from solid to liquid, or liquid to solid, occurs at radial node \(n\),
Equation~\ref{eq:fd_heat_conduction} is modified as shown in
Equation~\ref{eq:fd_heat_conduction_w_phase_change} to account for the storage or release of the
heat of fusion while the temperature remains equal to the melting temperature.

\begin{equation}
    \label{eq:fd_heat_conduction_w_phase_change}
    \rho H\left( h_{\ln}^{V} + h_{rn}^{V} \right)\dv{\alpha_{n}^{m + \frac{1}{2}}}{t} =
   -\left( T_{L} - T_{n - 1}^{m + \frac{1}{2}} \right)k_{\ln}h_{\ln}^{S}\  + \ \left( T_{n + 1}^{m + \frac{1}{2}} - T_{L} \right)k_{rn}h_{rn}^{S} + Q_{\ln}^{m + \frac{1}{2}}h_{\ln}^{V} + Q_{rn}^{m + \frac{1}{2}}h_{rn}^{V}
\end{equation}

Where,
\ind{   $\dv{\alpha_{n}^{m + \frac{1}{2}}}{t}$  =  Rate of change of volume fraction of material melted in the two half-mesh spaces on either side of radial node $n$ during the midpoint of the time step                   \sinum{1/s} }
\ind{   $H$                                     =  Heat of fusion of the material                                                        \sinum{J/kg}                  }
\ind{   $T_{L}$                                 =  Melting temperature of the material                                                   \sinum{K}                     }

The phase change from solid to liquid is complete when

\begin{equation}
    \label{eq:phase_change_complete}
    \sum_{m = M_{1}}^{M_{2}}{\frac{d\alpha_{n}^{m + \frac{1}{2}}}{\text{dt}}\mathrm{\Delta}t^{m}} = 1
\end{equation}

Where,
\ind{   $M_{1}$                 =  Number of time step at which melting started }
\ind{   $M_{2}$                 =  Number of time step at which melting ends    }
\ind{   $\mathrm{\Delta}t^{m}$  =  Size of $m$-th time step \sinum{s}               }

The finite difference approximations at each radial node are combined together to form one
tri-diagonal matrix equation. The equation has the form

\begin{equation}
    \label{eq:fd_heat_conduction_tridiagonal_matrix}
    \begin{bmatrix}
        b_{1}   &   c_{1}   &   0       &  0        &   \dots     &   0         \\
        a_{2}   &   b_{2}   &   c_{2}   &  0        &   \dots     &   0         \\
        0       &   a_{3}   &   b_{3}   &  c_{3}    &   \dots     &   0         \\
        \vdots  &   \hdots  &   \ddots  &  \ddots   &   \ddots    &   \vdots    \\
        \vdots  &   \hdots  &   0       &  a_{N-1}  &   b_{N-1}   &   c_{N-1}   \\
        0       &   \hdots  &   \vdots  &   0       &   a_{N}     &     b_{N}   \\  
    \end{bmatrix}
    \begin{bmatrix}
        T_{1}^{m+1}         \\
        T_{2}^{m+1}         \\
        T_{3}^{m+1}         \\
        \vdots              \\
        T_{N-1}^{m+1}       \\
        T_{N}^{m+1}         \\
    \end{bmatrix}
    =
    \begin{bmatrix}
        d_{1}               \\
        d_{2}               \\
        d_{3}               \\
        \vdots              \\
        d_{N-1}             \\
        d_{N}               \\
    \end{bmatrix}
\end{equation}

Equation~\ref{eq:fd_heat_conduction_tridiagonal_matrix} is solved by Gaussian elimination for the
radial node temperatures. Because the off-diagonal elements are negative and the sum of the diagonal
elements is greater than the sum of the off-diagonal elements, little roundoff error occurs as a
result of using Gaussian elimination. When the forward reduction step of Gaussian elimination has
been applied, the last equation in the transformed equation is:

\begin{equation}
    \label{eq:fd_heat_conduction_transformed_equation}
    AT_{N}^{m + 1} + B = q_{N}^{m + 1}
\end{equation}

Where,
\ind{   $T_{N}^{m + 1}$    =    Cladding surface temperature \sinum{K}                      }
\ind{   $q_{N}^{m + 1}$    =    Cladding surface heat flux   \sinum{W/m^{2}} }
\ind{   $A,B$              =    Coefficients that are defined in Section~\ref{chapter:calculation_cladding_tsurf}}

Equation~\ref{eq:fd_heat_conduction_transformed_equation} is combined with the correlation for
convective heat transfer to solve for the cladding surface temperature, as previously shown in
Figure~\ref{fig:fd_geometry_terms}.
\\
\\
The description of the calculations for the temperature distribution in the fuel and cladding is
complete at this point. The calculation of the temperature of the gas in the fuel rod plenum is then
needed to complete the solution for the fuel rod temperature distribution. This calculation is
performed by a separate model and is described in Section~\ref{section:fuel-rod-thermal-response}.
