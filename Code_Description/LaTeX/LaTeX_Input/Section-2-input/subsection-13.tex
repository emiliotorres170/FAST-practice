\section{Coolant Conditions}\label{section:coolant-conditions}

If the user chooses to model the coolant as water, the fuel rod cooling model calculates the amount
of heat transfer from the fuel rod to the surrounding coolant. The model calculates the heat
transfer coefficient, surface heat flux, and temperature at the cladding surface. These variables
are determined by the simultaneous solution of two independent equations for cladding surface heat
flux and surface temperature.
\\
\\
One of the equations is the appropriate correlation for convective heat transfer from the fuel rod
surface. This correlation relates surface heat flux to surface temperature and coolant conditions.
Different correlations are required for different heat transfer modes, such as nucleate or film
boiling. The relation of the surface heat flux to the surface temperature for the various heat
transfer modes is shown in Figure~\ref{fig:water_SHF_temp}. Logic for selecting the appropriate mode
and the correlations available for each mode are shown in Table~\ref{tab:Water_HTMode_Selection}.
The correlations are described in Section~\ref{chapter:HTC_and_models}.
\\
\\
The second independent equation containing surface temperature and surface heat flux as the only
unknown variables is derived from the finite difference equation for heat conduction at the mesh
bordering the fuel rod surface. A typical plot of this equation during the nucleate boiling mode of
heat transfer is also shown in Figure~\ref{fig:water_SHF_temp}. The intersection of the plot of this
equation and that of the heat transfer correlations determines the surface heat flux and
temperature. The derivation of this equation and the simultaneous solution for surface temperature
and surface heat flux are described in Appendix C. Neither of the two equations solved
simultaneously contains past iteration values so that numerical instabilities at the onset of
nucleate boiling are avoided. A separate set of heat transfer correlations is used to calculate fuel
rod cooling during the reflooding portion of a LOCA.  During this period, liquid cooling water is
injected into the lower plenum and the liquid level gradually rises over time to cover the fuel
rods. This complex heat transfer process is modeled by a set of empirical relations derived from
experiments performed in the FLECHT facility (\cite{ref:Cadek1972}). A description
of these models is presented in \colorbox{yellow}{Appendix D}.
\\
\\
The user specifies the inlet condition at the bottom of the fuel rod.  The enthalpy rise across the
height of the coolant channel is initially calculated based on the known surface heat flux from the
previous timestep. The values obtained from the bottom and top of the coolant channel (Axial nodes
$j-1$ and $j$) are averaged together to obtain a ``cell-centered'' average coolant condition used as
the boundary condition in the conduction solution.

\begin{figure}
    \includegraphics[height=\figheight\textheight]{../media/image152}
    \caption{Relation of Surface Heat Flux to Surface Temperature}
    \label{fig:water_SHF_temp}
\end{figure}

\begin{ThreePartTable}
    \begin{TableNotes}
        \footnotesize
        \item[a] 
		    The symbols used are                                              \\ 
            $   Q_i      $  =   Surface flux in the $i-th$ heat transfer mode \\
            $   X        $  =   Coolant Quality                               \\
            $   Q_{crit} $  =   Critical heat flux                            \\
            $   G        $  =   Mass flux \sinum{lbm/hr-ft^{2}})              \\
            $   T_{w}    $  =   Cladding surface temperature                  \\
            $   P        $  =   Coolant Pressure \sinum{psia}                 \\
            $   T_{sat}  $  =   Saturation temperature of coolant             \\
            $   T_{bulk} $  =   Local bulk temperature of coolant             \\
        \item[b] 
            Parameters limits describing the range of the heat transfer apply
            to the default correlation for each mode. The correlation to be used
            is specified in the input.
    \end{TableNotes}

    \begin{longtable}[c]{L K J J}
        \caption{Water Heat Transfer Mode Selection and Correlation}   \\
        \label{tab:Water_HTMode_Selection}                             \\ \hline
        Heat Transfer Mode                                &     Range \tnote{a}                                                                         &   Default Correlation \tnote{b}                                                                                          &   Optional Heat Transfer Correlation(s)                                              \\   \hline
        \endfirsthead
        \caption[]{Water Heat Transfer Mode Selection and Correlation} \\ \hline
        Heat Transfer Mode                                &     Range \tnote{a}                                                                         &   Default Correlation \tnote{b}                                                                                          &   Optional Heat Transfer Correlation(s)                                              \\   \hline
        \endhead
        \insertTableNotes
        \endlastfoot
        Forced convection to subcooled liquid (Mode 1)    &  \(T_{w} < T_{sat}\)  \(Q_{2} < Q_{1} < Q_{crit})\)                                         & Dittus-Boelter (\cite{ref:Dittus1930a}) for turbulent flow; constant $Nu = 7.86$ for laminar flow (\cite{ref:Sparrow1961}) &  -                                                                                     \\   \hline
        Subcooled nucleate boiling (Mode 2)               &  \(Q_{1} < Q_{2} < Q_{crit}\) \(T_{b} > T_{sat}\) \(T_{w} > T_{sat}\)  & Jens-Lottes                                                                                                                      &  Chen (\cite{ref:Chen1963}) Thom (\cite{ref:Thom1965})                                                \\   \hline
        Saturated nucleate boiling (Mode 3)               &  \(Q_{1} < Q_{2} < Q_{crit}\) \(T_{b} = T_{sat}\) \(T_{w} > T_{sat}\)  & Jens-Lottes                                                                                                                      &  Chen (\cite{ref:Chen1963}) Thom (\cite{ref:Thom1965})                                                \\   \hline
        Post-CHF transition boiling (Mode 4)              &  \(Q_{2} > Q_{crit}\) \(Q_{4} > Q_{5}\) \(G > 200,000\)                              & Modified Tong-Young (Tong and Young 1974)                                                                          &  Bjornard-Griffith (\cite{ref:Bjornard1977}) Modified Condie-Bengston (\cite{ref:INL1978})            \\   \hline
        Post-CHF film boiling (Mode 5)                    &  \(Q_{2} > Q_{crit}\) \(Q_{5} > Q_{4}\) \(Q_{5} > Q_{6}\) \(G > 200,000\)            & Groeneveld 5.9 (\cite{ref:Groeneveld1973}, \cite{ref:Groeneveld1978}; \cite{ref:Groeneveld1976})                   &  Bishop-Sandberg-Tong (\cite{ref:Bishop1965}) Groeneveld-Delorme (\cite{ref:Groeneveld1976})          \\   \hline
        Post-CHF boiling for low flow conditions (Mode 7) &  \(Q_{2} > Q_{crit}\) \(Q_{6} > Q_{5}\) \(G < 200,000\)                              & Bromley (\cite{ref:Bromley1950})                                                                                   &  -                                                                                                    \\   \hline
        Forced convection to superheated steam (Mode 8)   &  \(X > 1\)                                                                           & Dittus-Boelter (\cite{ref:Dittus1930a})                                                                            &  -                                                                                                    \\   \hline
    \end{longtable}
\end{ThreePartTable}
