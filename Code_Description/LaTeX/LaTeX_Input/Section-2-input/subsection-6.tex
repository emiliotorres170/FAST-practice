\section{Rod Internal Pressure and Fission Gas Release}\label{section:rod-internal-pressure-and-fission-gas-release}

The pressure of the gas in the fuel rod must be known in order to calculate the deformation of the
cladding and the transfer of heat across the fuel-cladding gap. The pressure is a function of the
gas temperature, composition, void volume, and quantity of gas. Because the temperature is spatially
non-uniform, the fuel rod must be divided into several smaller volumes so that the temperature in
each small volume can be assumed to be uniform. In particular, the fuel rod is divided into a plenum
volume and several fuel-cladding gap and fuel void volumes. The temperature of each volume is given
by the thermal models described in Section~\ref{section:fuel-rod-thermal-response}, the size of the
volume by the deformation models described in Section~\ref{section:fuel-rod-mechanical-response},
and the quantity of gases by the fission gas release models described later in
Section~\ref{section:rod-internal-pressure-and-fission-gas-release}.
\\
\\
The internal gas pressure can be calculated either by a static pressure model (which assumes that
all volumes inside the fuel rod equilibrate in pressure instantaneously) or by a transient pressure
model which takes into account the viscous flow of the gas in the fuel rod. The static pressure
model is the default model. The transient model is an input option. Unless the fuel-cladding gap is
small ($\leq \SInum{25}{\mu m}$), the static and transient models give identical results.
\\
\\
The static fuel rod internal gas pressure model is based on the
following assumptions:

\begin{enumerate}
    \item Perfect gas law holds ($PV = NRT$).
    \item Gas pressure is constant throughout the fuel rod.
    \item Gas in the fuel cracks is at the average fuel temperature.
\end{enumerate}

The transient fuel rod internal gas pressure model is based on the
following assumptions:

\begin{enumerate} \def\labelenumi{\arabic{enumi}.}
    \item Perfect gas law holds ($PV = NRT$).
    \item Gas flow past the fuel column is a quasi-steady process.
    \item Gas flow is compressible and laminar.
    \item Gas flow past the fuel column can be analyzed as Poiseuille flow (that is, by force balance
        only).
    \item Gas expansion in the plenum and ballooning zone is an isothermal process.
    \item The entire fuel-cladding gap can be represented as one volume containing gas at a uniform
        pressure.
    \item The flow distance is equal to the distance from the plenum to the centroid of the
        fuel-cladding gap.
    \item The minimum cross-sectional area of flow is equivalent to an annulus with inner radius equal
        to that of the fuel pellet radius and a radial thickness of $\SInum{25}{\mu m}$.
\end{enumerate}
%-------------------------------------------------------------------------%
% Static Gas Pressure                                                     %
%-------------------------------------------------------------------------%
\subsection{Fuel Rod Internal Gas Pressure}
\subsubsection{Static Gas Pressure}\label{section:static-gas-pressure}

Fuel rod internal gas pressure is calculated from the application of the perfect gas law to a
multiple volume region. The volumes accounted for in FAST-1.0 include the hot plenum volume, gap,
annulus, crack, dish, porosity, roughness, and pellet-pellet interface volumes specific to each
node. Thus, the equation for rod internal pressure is

\begin{equation}
    \label{eq:rod_internal_pressure}
    P = \frac{MR}{\frac{V_{p}}{T_{p}} + \sum_{n = 1}^{N}\left\lbrack \frac{V_{g}}{T_{g}} + \frac{V_{ch}}{T_{ch}} + \frac{V_{cr}}{T_{cr}} + \frac{V_{dsh}}{T_{dsh}} + \frac{V_{por}}{T_{por}} + \frac{V_{rfc}}{T_{rfc}} + \frac{V_{i}}{T_{i}} \right\rbrack}
\end{equation}

Where,
\ind{    $ P $       =          Rod internal pressure                                                         \sinum{Pa}                   }
\ind{    $ M $       =          Total \# of moles of gas                                                                             }
\ind{    $ R $       =          Universal gas constant, \SInum{8.34}{J/mole-K}                                                                }
\ind{    $ N $       =          Number of axial nodes into which fuel rod is divided for numerical solution                          }
\ind{    $ n $       =          Axial node number                                                                                    }
\ind{    $ V $       =          Volume                                                                        \sinum{m^{3}} }
\ind{    $ T $       =          Temperature                                                                   \sinum{K}                    }

The following void volumes are accounted for in the pressure solution:

\renewcommand{\captiontext}{Void Volumes}
\begin{longtable}[c]{D{2.0cm}D{4.0cm}D{9.0cm}}
    \caption{\captiontext} \label{tab:void-volume-pressure-solution} \\   \hline
        \textbf{Symbol}     & \textbf{Location}     & \textbf{Temperature Description}  \\  \hline
    \endfirsthead
    \caption{\captiontext (continued)}  \\   \hline
        \textbf{Symbol}     & \textbf{Location}     & \textbf{Temperature Description}  \\  \hline
    \endhead
        $ p $     &  Plenum                         &  As described in Section~\ref{section:plenum-volume}, dependent on the upper          \\
                  &                                 &  cladding and fuel temperatures                                               \\
        $ g $     &  Gap                            &  Average of the fuel surface and cladding ID temperatures                     \\                                               
        $ ch $    &  Central hole                   &  Nodal fuel centerline temperature                                            \\
        $ cr $    &  Fuel cracks                    &  Average of fuel surface and temperature at the restructured fuel radius      \\
        $ dsh $   &  Fuel dish                      &  Fuel volume average temperature                                              \\
        $ cham $  &  Fuel chamfer                   &                                                                               \\
        $ por $   &  Fuel open-porosity             &  Fuel volume average temperature                                              \\
        $ rfc $   &  Fuel \& Clad surface roughness &  Temperature at fuel surface and cladding ID                                  \\
        $ i $     &  Interface volumes              &  Average between fuel volume average and                                      \\
\end{longtable}

\notes{Note: In the FAST-1.0 Output file, a table appears that presents the
fractions of total volume represented by the plenum, gap, cracks,
dishes, annulus, open porosity, and roughness, and the rod-averaged
temperatures associated with these various volume-fractions. These are
not the node-specific values that appear in the above equation, but are
he results of the sum of each axial node for each volume.} 
\\
\\
The gas pressure calculation, therefore, requires information on the gas
inventory, void volumes, and the void temperatures, which is provided by
the following supportive models.
%-------------------------------------------------------------------------%
% Transient internal Gas flow
%-------------------------------------------------------------------------%
\subsubsection{Transient Internal Gas Flow} \label{section:transient-internal-gas-flow}

Transient flow of gas between the plenum and fuel-cladding gap is
calculated by the Poiseuille equation for viscous flow along an annulus
according to Equation~\ref{eq:transient_internal_gas_mass_flow_rate}. Assumptions inherent in the equation 
are ideal gas, laminar flow, and density based on linear average pressure:

\begin{equation}
    \label{eq:transient_internal_gas_mass_flow_rate}
    \dot{m} = \frac{\pi \left( P_{P}^{2} - P_{S}^{2} \right)}{R\mu \sum_{i = I_{s}}^{I_{p}}\frac{l_{i}T_{i}H}{D_{g}{D_{h}}^{3}}}
\end{equation}
%\cleardoublepage
Where,
\ind{   $ \dot{m} $       =  Mass flow rate                                                               \sinum{g-moles/s}                                                                    }
\ind{   $ \mu$            =  Gas viscosity at temperature $T_{a} $                                        \sinum{N-s/m^{2}}                                                     }
\ind{   $ T_{a} $         =  Volume-averaged temperature of gas in gas                                    (fuel-cladding) gap \sinum{K}                                                        }
\ind{   $ T_{i}$          =  Gas temperature at node $i $                                                 \sinum{K}                                                                            }
\ind{   $ l_{i}$          =  Axial length of node $i $                                                    \sinum{m}                                                                            }
\ind{   $ I_{P} $         =  Number of top axial node                                                                                                                                    }
\ind{   $ I_{s} $         =  Number of axial node closest to centroid of gas gap                          (see Figure~\ref{fig:internal_pressure_distribution_transient_gas_flow_model}) }
\ind{   $ H $             =  Hagen number                                                                 (see Equation~\ref{eq:hagen_number})                                           }
\ind{   $ P_{P} $         =  Fuel rod plenum gas pressure                                                 \sinum{N/m^{2}}                                                       }
\ind{   $ P_{S} $         =  Fuel-cladding gap gas pressure                                               \sinum{N/m^{2}}                                                       }
\ind{   $ R $             =  Universal gas constant                                                       \sinum{N-m/K-g-moles}                                                                }
\ind{   $ D_{g} $         =  Mean diameter of fuel-cladding gap                                           \sinum{m}                                                                            }
\ind{   $ D_{h}$          =   Hydraulic diameter of fuel-cladding gap = 2$t_{\text{gi}} $ for a small gap \sinum{m}                                                                            }
\ind{   $ t_{gi}$  =  Fuel-cladding radial gap thickness at node $i $                              \sinum{m}                                                                            }
    

The Hagen number is calculated by:

\begin{equation}
    \label{eq:hagen_number}
    H = 22 + \frac{0.24558}{2t_{gi} -\num{7.87E- 4}}
\end{equation}

Where, 
    \ind{   \(t_{gi}\) is in \sinum{in}.    }


A plot of the relation between Hagen number and gap width given by Equation~\ref{eq:hagen_number} is
shown in Figure~\ref{fig:hagen_number}.  For gaps smaller than \SInum{25}{\mu m}, the function is
cut off at value of 1177. To calculate the fuel-cladding gap pressure, a modified form of
Equation~\ref{eq:rod_internal_pressure} is used. The plenum term is deleted and the moles of gas in
the fuel-cladding gap is substituted in place of the moles of gas in the fuel rod.

\begin{figure}
    \includegraphics[height=\figheight\textheight]{../media/image137}
    \caption{Internal Pressure Distribution with the Transient Gas Flow Model}
    \label{fig:internal_pressure_distribution_transient_gas_flow_model}
\end{figure}

\begin{figure}
 \includegraphics[height=\figheight\textheight]{../media/Hagen_number_vs_gap_thickness}
 \caption{Hagen Number Versus Width of Fuel-Cladding Gap}
 \label{fig:hagen_number}
\end{figure}

\subsection{Fuel Rod Void Volume} \label{section:fuel-rod-void-volume}

Void volumes computed by FAST-1.0 include the plenum, pellet dishes and chamfers, the fuel-cladding
gap, the fuel cracks, the fuel open porosity, pellet-pellet interface, the fuel and cladding surface
roughness volumes, and (if it exists) the fuel central hole volume. These void volumes are
calculated as indicated below.

\subsubsection{Plenum Volume} \label{section:plenum-volume}

The plenum volume is calculated from geometry considerations and irradiation induced effects on both
the fuel and cladding. In the plenum, the cladding undergoes creep and thermal expansion, while the
fuel undergoes thermal expansion, swelling and densification. The axial effects of both the fuel and
cladding deformation are accounted for in determining the reduction (or possibly expansion) of the
plenum free volume.  In addition, the volume of the hold-down spring is considered, as well as
changes in the spring geometry due to thermal expansion.

\subsubsection{Pellet Dish and Chamfer Volumes} \label{section:pellet-dish-and-chamfer-volumes}

The volume between pellets is calculated and included as part of the overall volume in the internal
gas pressure model. The inter-pellet volume is calculated at each time step based on hot-pellet
geometries.  Figure~\ref{fig:interpellet_void_volume} shows 1) a cold-pellet interface configuration
for the case where the pellets are dished and 2)~an exaggerated hot-pellet interface configuration.
The void volume available for internal fill gas is defined by the cross-hatched areas ($a$ and $b$
in Figure~\ref{fig:interpellet_void_volume}).  The dish volume is that portion of the hot
inter-pellet volume that is within the dishes, excluding the volume of any central hole. The chamfer
volume is included in the portion of the hot inter-pellet volume
that is outside the dishes.

\subsubsection{Fuel-Cladding Gap Volume} \label{section:fuel-cladding-gap-volume}

The fuel-cladding gap volume is calculated by considering the area between two concentric cylinders.
The outer cylinder is assumed to have a diameter equal to the diameter of the cladding inside
surface based on plastic deformation. The inside cylinder is assumed to have a diameter equal to the
diameter of the relocated fuel pellets.

\subsubsection{Fuel Crack Volume} \label{section:fuel-crack-volume}

As the fuel expands, extensive cracking occurs due to the high thermally
induced stresses, resulting in a relocated fuel surface. The crack
volume is computed in Equation~\ref{eq:fuel_crack_volume}.

\begin{equation}
    \label{eq:fuel_crack_volume}
    V_{c} = V_{g} - V_{eg} - V_{tx}
\end{equation}

Where,
\ind{   $ V_{c} $   =  Fuel crack volume per unit length \sinum{m^{2}}                                                         }
\ind{   $ V_{g} $   =  The volume per unit length within the thermally expanded cladding \sinum{m^{2}}                         }
\ind{   $ V_{eg} $  =  Fuel volume per unit length defined by expanded radial nodes, including the thermal expansion swelling, and densification \sinum{m^{2}}                                                               }
\ind{   $ V_{tx} $  =  The computed fuel-cladding gap volume per unit length based on the relocated fuel surface \sinum{m^{2}} }

\subsubsection{Open Porosity Volume} \label{section:open-porosity-volume}

A portion of the initial fabrication porosity is open to free gas flow, which is given by the
following expressions:

\begin{equation}
    V_{por} = 
    \begin{cases}
        \num{2.77E-4}-3.8181\Gden-\num{1.43E-8}\Gden^{2}+\num{2.497E-10}\Gden^{3}, & \Gden \leq 91.25     \\
        \num{1.97E-8}\left(94.0-\Gden\right),                                              & 91.25 < \Gden < 94.0 \\
        0.0 ,                                                                                 & \Gden \geq 94.0
    \end{cases}
\end{equation}

Where,
\ind{   $ V_{por} $  =  Porosity volume per unit length \sinum{m^{2}}           }
\ind{   $ G_{den}$   =  $DEN - 1.25 $                                                    }
\ind{   $ DEN $      =  Fuel density                    (\percent of theoretical density) }

\notes{Note: It should be noted that most commercially fabricated fuel today has little open
porosity.} 

\subsubsection{Pellet-Pellet Interface Volume} \label{section:pellet-pellet-interface-volume}

The pellet-pellet interface volume is calculated as the difference between the hot inter-pellet
volume and the dish volume.

\subsubsection{Surface Roughness Volume} \label{section:surface-roughness-volume}

The roughness of the surface of the fuel and cladding results in a small void volume accounted for
by

\begin{equation}
    \label{eq:surface_roughness_volume}
    V_{rough} = \frac{\pi\left( {r_{p}}^{2} - \left( r_{p} - rough_{p} \right)^{2} \right) + \pi\left( \left( r_{c} + rough_{c} \right)^{2} - {r_{c}}^{2} \right)}{V/L}
\end{equation}

Where,
\ind{   $ V_{rough} $  =  Fraction of pellet volume in roughness                        }
\ind{   $ r_{p} $      =  Outer radius of pellet                 \sinum{m}                    }
\ind{   $ r_{c} $      =  Inner radius of cladding               \sinum{m}                    }
\ind{   $ rough_{p} $  =  Fuel surface roughness                 \sinum{m}                    }
\ind{   $ rough_{c} $  =  Cladding surface roughness             \sinum{m}                    }
\ind{   $ V $          =  Volume of pellet stack                 \sinum{m^{3}} }
\ind{   $ L $          =  Length of fuel stack                   \sinum{m}                    }

The gas pressure response resulting from the above models feeds back into the mechanical and
temperature response models in the iteration scheme.

\subsubsection{Central Hole Volume} \label{section:central-hole-volume}

The central hole volume is calculated by considering the area of the central hole (if present), the
length of the axial node, and the length of the central hole.

\subsection{Fission Gas Production} \label{section:fission-gas-production}

Given production rates for the major diffusing gases, the burnup-dependent total fission gas
generated at axial elevation $ z $ is calculated as

\begin{equation}
    \label{eq:fission_gas_production_rate}
    GPT\left( z \right) = \frac{BU\left( z \right)VF\left( z \right)}{100A_{v}}\left( PR_{He} + PR_{Kr} + PR_{Xe} \right)
\end{equation}

Where,
\ind{    $ GPT\left( z \right) $  =  Total fission gas produced at Elevation $ z $ \sinum{moles} }
\ind{    $ BU\left( z \right)$    =  Burnup at elevation $ z $ \sinum{fission/cm^{3}} }
\ind{    $ VF\left( z \right) $   =  Fuel volume                                   \sinum{cm^{3}}         }
\ind{    $ A_{v} $                =  Avogadro's number                                                             }
\ind{    $ PR $                   =  Fission gas production rate                   \sinum{atoms/100-fissions}            }

All the fission gas produced, however, is not released. A portion is trapped in the fuel and a
portion is released to the fuel-cladding gap volume. Only the released portion is used to calculate
the rod internal gas pressure. The gas release fraction is calculated as discussed in the following
sections.

\subsection{Steady-State Gas Release} \label{section:steady-state-gas-release}

Gas release models in FAST-1.0 account for not only fission gas release (krypton, xenon, and helium)
but also nitrogen release. The nitrogen is released from the fuel lattice, where it is trapped
during the fuel fabrication process. There are four fission gas release model options: ANS-5.4
(1982) (\cite{ref:Rausch1979}) ANS-5.4(2011); the modified Forsberg and Massih
model (\cite{ref:Forsberg1985}), modified at PNNL; and the FRAPFGR model
developed at PNNL. All four of these release models are based on earlier formulations for diffusion
from a sphere by \cite{ref:Booth1957} and are discussed below.
\\
\\
The Massih model is recommended and is set as the default model. The ANS-5.4 models are useful for
calculating the release of short-lived radioactive gas nuclides but are known to provide very
conservative values for release. The FRAPFGR model is useful for initializing the transient gas
release model. However, neither the ANS-5.4 model nor the FRAPFGR model predicts stable fission gas
release as well as the Massih model does, thus it is recommended that the Massih model be used for
best-estimate calculation of stable fission gas release.

\subsubsection{ANS-5.4 (1982) Gas Release Model} \label{section:ans-5.4-1982-gas-release-model}

The ANS-5.4 (ANS, 1982) fractional fission gas release is calculated as a function of time and
radial fuel temperature and axial burnup. The fuel is divided into radial and axial nodes according
to the old 1982 American Nuclear Society (ANS) standard. A user requirement is that the time step
sizes be such that the burnup increments do not exceed \SInum{2}{GWd/MTU}.
\\
\\
The modeling is divided into two main sections, one for release of stable isotopes and the other for
release of short-lived isotopes. There are high- and low-temperature models for both the stable and
radioactive fission products. The release is calculated using both the high-temperature and the
low-temperature models, and the larger release value is used. Time steps should not exceed 50 days.
\\
\\
The ANS-5.4 fission gas release model (\cite{ref:ANS1982}) is incorporated as specified by the
standard and will not be described in this document. A revised ANS-5.4 fission gas release model has
been recently approved as a standard (\cite{ref:ANS2011}). The 1982 model is not currently an
approved standard and provides a very conservative prediction of release in the FAST-1.0 code, while
the revised model provides a less conservative prediction even at the 95/95 upper bound. The 1982
model is retained in FAST-1.0 for compliance with various regulations. The new ANS-5.4 standard
(\cite{ref:ANS2011}) is also available and is described below.

\subsubsection{ANS-5.4 (2011) Gas Release Model} \label{section:ans-5.4-2011-gas-release-model}

The new ANS-5.4 standard was approved in 2011 and it provides a methodology for determining the
radioactive releases from fuel rods, and to determine radiological consequences of postulated
accidents. The model is based on the assumption that no significant power transient will occur, such
as reactivity insertion accidents (RIAs). This model includes volatile and gaseous fission products
of primary importance such as krypton, xenon, iodine, and cesium. The largest contributor to the
equivalent dose to individuals is generally I-131, which is included in the model. The radioactive
gaseous and volatile fission products are divided into two categories: 

\begin{enumerate}
    \item Short-lived radioactive nuclides with a half-life $ < \SInum{1}{year}$ 
    \item Long-lived radioactive nuclides with half-life $ > \SInum{1}{year}$. 
\end{enumerate}

This distinction is particularly important when considering diffusion processes that proceed slowly
as compared to the decay time for the nuclides under consideration.  The model presented in the ANS
5.4 2011 standard is applicable to short lived nuclides; a further distinction is applied in the
standard for nuclides with a half-life smaller than six hours, and nuclides with half-lives greater
than six hours but smaller than sixty days.
\\
\\
The first incarnation of the ANS-5.4 standard was first implemented in 1982 and it was originally
based on the Booth diffusion model. The model coefficients were determined from the measured release
data for xenon and krypton. Because of the lack of data for I-131, the diffusion coefficient for
this nuclide was assumed to be a factor of seven higher than the one used for xenon and krypton.
However, in the last twenty-five years, fuel experiments in test reactors have been performed and
measured data have been used to validate the standard at higher burnups. Based on this data it was
also concluded that the prediction for I-131 was overly conservative.
\\
\\
The fission gas release model from ANS 5.4 2011 (\cite{ref:ANS2011}) implements the model described
in the standard and it calculates the release-to-birth ratio ($R/B$), or the so-called ``gap
release'' for short-lived and long-lived nuclides, as defined by the standard. The short lived and
long lived nuclides considered by the model are listed in
Tables~\ref{tab:short_decay_constants_ans_fgr} and Table~\ref{tab:long_decay_constants_ans_fgr},
respectively, together with their precursor coefficients and decay constants for radioactive
nuclides. The nuclides are categorized as short lived if their half-life is less than six hours,
while they are considered long lived, if their half-life is greater than six hours but less than
sixty days. It should be noted that Table 1 of (\cite{ref:ANS2011}) does not contain all the
nuclides mentioned in the text of the document. Due to this issue, it was necessary to obtain the
complete nuclide list and associated physical parameters from (\cite{ref:Turnbull2010}).

\renewcommand{\captiontext}{Decay Constants and Precursor Coefficients for short lived Noble Gases
and Iodines with half-life $<\SInum{6}{h}$}
\begin{longtable}[c]{KA{4.0cm}K}     
    \caption{\captiontext} \label{tab:short_decay_constants_ans_fgr}    \\ \hline
        \textbf{Nuclide} & \textbf{Precursor Coef. ($\alpha$)} &             \textbf{Decay Constant} \\ \hline
    \endfirsthead
    \caption{\captiontext (continued)}    \\ \hline
        \textbf{Nuclide} & \textbf{Precursor Coef. ($\alpha$)} &             \textbf{Decay Constant} \\ \hline
    \endhead
        \element{Xe}{}-135m          & 23.5                                &             \num{7.55E-4}\\
        \element{Xe}{}-137           & 1.07                                &             \num{3.02E-3}\\
        \element{Xe}{}-138           & 1.0                                 &             \num{8.19E-4}\\
        \element{Xe}{}-139           & 1.0                                 &             \num{1.75E-2}\\
        \element{Kr}{}-85m           & 1.31                                &             \num{4.30E-5}\\
        \element{Kr}{}-87            & 1.25                                &             \num{1.52E-4}\\
        \element{Kr}{}-88            & 1.03                                &             \num{6.78E-5}\\
        \element{Kr}{}-89            & 1.21                                &             \num{3.35E-3}\\
        \element{Kr}{}-90            & 1.11                                &             \num{2.15E-2}\\
        \element{I}{}-132            & 137                                 &             \num{8.44E-5}\\
        \element{I}{}-134            & 4.4                                 &             \num{2.20E-4}\\ 
\end{longtable}

\renewcommand{\captiontext}{Decay Constants and Precursor Coefficients for long lived Noble Gases and Iodines with $ \SInum{6}{h} < \text{half-life} < \SInum{60}{days}$}
\begin{longtable}[c]{K A{4.0cm} K}     
    \caption{\captiontext}    \label{tab:long_decay_constants_ans_fgr}    \\ \hline
            \textbf{Nuclide} & \textbf{Precursor Coef. ($\alpha$)} &             \textbf{Decay Constant} \\ \hline
    \endfirsthead
    \caption{\captiontext (continued)}   \\ \hline
            \textbf{Nuclide} & \textbf{Precursor Coef. ($\alpha$)} &             \textbf{Decay Constant} \\ \hline
    \endhead
        \element{Xe}{}-133       & 1.25                                &             \num{1.53E-6}\\
        \element{Xe}{}-135       & 1.85                                &             \num{2.21E-5}\\
        \element{I}{}-131        & 1.0                                 &             \num{9.98E-7}\\
        \element{I}{}-133        & 1.21                                &             \num{9.26E-6}\\ 
\end{longtable}

\notes{Note: that ANS-2011 FGR model requires at least 11 axial meshes in order to accurately
predict the fission gas release.}

\subsubsection{Modified Forsberg-Massih Model} \label{section:modified-forsberg-massih-model}

The original Forsberg-Massih model begins with a solution of the gas
diffusion equation for constant production and properties in a spherical
grain:

\begin{equation}
    \label{eq:gas_diffusion_forsberg_massih}
    \frac{dC}{dt} = D\left( t \right)\Delta_{r}C\left( r,t \right) + \beta\left( t \right)
\end{equation}

{\color{red}If $C$ is a $func(r,t)$ should these be partial derivatives ?}

Where,
\ind{   $ C $              =      Gas concentration                                                   }
\ind{   $ \beta $          =      Gas production                                                      }
\ind{   $\Delta_{r}$       =      \(\frac{d^{2}}{dr^{2}} + \frac{2}{r}\left( \frac{d}{dr} \right)\)   }
\ind{   $ D $              =      Diffusion constant                                                  }
\ind{   $ t $              =      Time                                                                }

With boundary conditions,
\ind{   $C\left( r,0 \right)$    =      0  }
\ind{   $C\left( a,t \right)$    =      0  }

Forsberg and Massih attempt to solve the equation for the case where there is re-solution of gas on
the grain surface, which changes the outer boundary condition to

\begin{equation}
    \label{eq:forsberg_massih_outer_bc}
    C\left( a,t \right) = \frac{b\left( t \right)\lambda N\left( t \right)}{2D}
\end{equation}


Where,
        
\ind{   \(N\)         =  Surface gas concentration }
\ind{   \(\lambda\)   =  Resolution layer depth    }
\ind{   \(a\)         =  Hypothetical grain radius }
\ind{   \(b\)         =  Resolution rate           }

They make use of a four-term approximation to the integration kernel,
\(K\), where

\begin{equation}
    \label{eq:forsberg_massih_integration_kernal}
    \int_{0}^{a}{4\pi r^{2}C\left( r,t \right)dr} = \int_{0}^{\tau}{K\left( \tau - \tau_{0} \right)\beta_{e}\left( \tau_{0} \right){d\tau_{0}}}
\end{equation}

\begin{equation}
    \label{eq:forsberg_massih_beta}
    \beta_{e} = \frac{\beta}{D}
\end{equation}

\begin{equation}
    \label{eq:forsberg_massih_tau}
    \tau = Dt
\end{equation}

\begin{equation}
    \label{eq:forsberg_massih_K}
    K = \frac{8a^{3}}{\pi}\sum_{n = 1}^{\infty}\frac{e^{\left( \frac{- n^{2}\pi^{2}\tau}{a^{2}} \right)}}{n^{2}}
\end{equation}

\subsubsection{Low-Temperature Fission Gas Release Model at High Burnup}\label{section:low-temperature-fission-gas-release-model-at-high-burnup}

The modified Forsberg-Massih model is used to calculate fission gas release unless the
low-temperature fission gas release model predicts a higher value for fission gas release. The
low-temperature fission gas release model is defined as

\begin{equation}
    \label{eq:forsberg_massih_low_temp_fgr}
    F = \num{7E- 5}BU + C
\end{equation}

Where,
\ind{   \(F\)   =  Fission gas release fraction }
\ind{   \(BU\)  =  Local burnup in \sinum{GWd/MTU}      }

and

$$
    C = \begin{cases}
            0,                     & BU \leq \SInum{40}{GWd/MTU}  \\
            0.01 \frac{BU-40}{10}, & BU>\SInum{40}{GWd/MTU} \text{ and } F\leq 0.05
        \end{cases}
$$

\subsubsection{Grain Boundary Accumulation and Re-Solution}\label{section:grain-boundary-accumulation-and-re-solution}

The final solution for a given time step, without re-solution and with constant production rate
during the step, can be written as

\begin{equation}
    \label{eq:forsberg_massih_delta_G_n}
    \Delta G_{n} = f_{n}G_{n}\left(\tau_{1}\right) + A_{n}\int_{\tau_1}^{\tau_2}e^{\frac{-B_{n}\left(\tau_2-\tau_0\right)}{a^2}}q\left(\tau_{0}\right)d\tau_0
\end{equation}

\begin{equation}
    \label{eq:forsberg_massih_delta_G}
    \Delta G = \sum_{n = 1}^{4} \Delta G_n
\end{equation}

Where,
\ind{    \(\mathrm{\Delta}G\) = change in gas concentration in fuel grain }

\begin{equation}
    \label{eq:forsberg_massih_G_b}
    \mathrm{\Delta}G_{B} = \sum_{}^{}{f_{n}G_{n}\left( \tau_{1} \right)} + A_{n}\int_{\tau_{1}}^{\tau_{2}}{\text{funct}\left( \tau_{2} - \tau_{0} \right)q\left( \tau_{0} \right){d\tau_{0}}}
\end{equation}

Where,
\ind{   \(\mathrm{\Delta}G_{B}\) = change in gas concentration on grain boundaries}

\begin{equation}
    \label{eq:forsberg_massih_f_n}
    f_{n} = e^{\left\lbrack \frac{- B_{n}\left( \tau_{2} - \tau_{0} \right)}{a^{2}} \right\rbrack} - 1
\end{equation}

\ind{   \(f_{n}\) = fission gas production fraction remaining in the grain from the previous time
        step.}

Where \(q\) is determined from

\begin{equation}
    \label{eq:forsberg_massih_q}
    a^{2}q\left\lbrack - \sum_{n = 1}^{4}\left( \frac{f_{n}A_{n}}{B_{n}} \right) + func\left( \Delta\tau \right) \right\rbrack = \beta\Delta t
\end{equation}

With the following relationship:

\begin{equation}
    \label{eq:forsberg_massih_funct}
    func\left( \Delta \tau \right) = \int_{\tau_{1}}^{\tau_{2}}funct\left(\tau_2 - \tau_0\right) d \tau = 
    \begin{cases}
        \frac{6}{\sqrt{\pi}}\sqrt{\frac{\tau_2-\tau_0}{a^{2}}}-3\frac{\tau_2-\tau_0}{a^2}, & \; \tau < 0.1 \\
        1-\frac{6}{\pi^2}e^{-\pi^2\frac{\left(\tau_2-\tau_0\right)}{a^2}},                 & \; \tau \geq 0.1
    \end{cases}
\end{equation}

Where,
\ind{   \(A_{n}\),\(\ B_{n}\) = Constants given by Massih}
\ind{   \(K_{2} = \frac{a}{3} - \frac{K}{4\pi a^{2}}\)}
\ind{   \(K_{3} = \frac{3}{a}K_{2}\)}
\ind{   \(1 + K_{3} = \sum_{n = 1}^{4}{A_{n}e^{\left( - \frac{B_{n}\tau}{a^{2}} \right)}}\)}

In modifying the original model, it was chosen to introduce
re-solutioning by defining the partition of the gas arriving at the
boundary each time step as follows:
\begin{equation}
    \Delta \text{Resolved Gas} = \Delta G_{B}  \frac{F}{1+F}
\end{equation}

\begin{equation}
    \Delta G_{B} =\frac{ \Delta G_{B} }{1+F}
\end{equation}

With the factor, \(F\), defined as

\begin{equation}
    F = M\left\lbrack \frac{\num{1.84E-14}r_{\text{grn}}}{3D} \right\rbrack
\end{equation}

Where,
\ind{   \(M\)           =    An empirical multiplier on the term in brackets that is the original Massih equation for the resolution rate                                            (\(M = 300\))            }
\ind{   \(r_{grn}\)     =    Grain radius \sinum{m} }
\ind{   \(D\)           =    Diffusion constant \sinum{m^{2}/s} }
\notes{NOTE: Although \(F\) is unitless in Massih's derivation, it does not
represent the fraction of retained gas.}

\subsubsection{Diffusion Constant}\label{section:diffusion-constant}

The diffusion constant (\(D\)) in the original Forsberg-Massih model is
defined over three temperature ranges, as follows:

\begin{equation}
    \label{eq:massih_diffusion_constant}
    D = 
    \begin{cases}
        \num{1.51E-17}\exp\left(\frac{- 9508}{T}\right),          & T < \SInum{1381}{K}               \\
        \num{2.14E-13}\exp\left(\frac{- 22884}{T}\right),         & \SInum{1381}{K} \leq T \leq \SInum{1650}{K} \\
        \num{1.09E-17}\exp\left(\frac{- 6614}{min(T,1850)}\right), & T > \SInum{1650}{K}              \\
    \end{cases}
\end{equation}

In principle, typically only the mid-range diffusion constant from
Equation~\ref{eq:massih_diffusion_constant} is generally used. The activation energy term
(\(\frac{Q}{R} = 22884 (1.15)\)). If the modified constant is less than the low-range Massih
diffusion constant, the latter is used.
\\
\\
A burnup enhancement factor multiplies the mid-range diffusion constant and has the form shown in
Equation~\ref{eq:massih_midrange_enhancement_factor}
\begin{equation}
    \label{eq:massih_midrange_enhancement_factor}
    \text{burnup factor} = 100^{\left( \frac{BU - 21}{40} \right)}
\end{equation}

Where
\ind{   \(BU\)  =  burnup, \sinum{GWd/MTU}}

The enhancement factor has a maximum value of 20,000. A factor of 12 is applied to the
burnup-enhanced diffusion constant as a final step.

\subsubsection{Gas Release}\label{gas-release}

The gas is accumulated at the grain boundary until a saturation concentration is achieved, at which
time the grain boundary gas is released. The saturation area density of gas is given by

\begin{equation}
    \label{eq:massih_gas_saturation_density}
    N_{s} = \left\lbrack \frac{4rF\left( \theta \right)V_{c}}{3K_{B}T\sin^{2}\left( \theta \right)} \right\rbrack\left( \frac{2\gamma}{r} + P_{ext} \right)
\end{equation}

Where,
\ind{   $N_{s}$    =  Saturation concentration                    \sinum{atoms/m^{2}} }
\ind{   $\theta$   =  Dihedral half-angle = 50$^{\circ}$                                       }
\ind{   $K_{B}$    =  Boltzman constant                                                        }
\ind{   $\gamma$   =  Surface tension = 0.6                       \sinum{J/m^{2}}     }
\ind{   $V_{c}$    =  Critical area coverage fraction = 0.25                                   }
\ind{   $r$        =  Bubble radius = 0.5                         \sinum{\mu m}                     }
\ind{   $P_{ext}$  =  External pressure on bubbles = gas pressure \sinum{Pa}                         }

And

\begin{equation}
    \label{eq:massih_F_theta_gas_release}
    F\left( \theta \right) = 1 - 1.5\cos\left( \theta \right) + 0.5\cos^{3}\left( \theta \right)\
\end{equation}

The final modification to the original model was to release both the grain boundary and the
re-solved gas whenever the saturation condition is achieved and the grain boundary gas is released.
\\
\\
To summarize, optimized parameters have been applied based on comparisons to selected steady-state
and transient data:

\begin{itemize}
    \item The activation energy \(\frac{Q}{R} = 1.15(22884) = 29060 \) (High temperature diffusion).
    \item The resolution parameter $= 300\left(\num{1.84E-14}\right) = \num{1.47E-12}$.
    \item Burnup enhancement factor on diffusion constant $= 100(BU-21)/40$.
    \item Multiplier on the diffusion constant = 12.0 (applied after all other modifications).
\end{itemize}

\subsubsection{FRAPFGR Model}\label{frapfgr-model}

The FRAPFGR model has been developed at PNNL to initialize the transient release model in FRAPTRAN
that is used to calculate fission gas release during fast transients such as a reactivity initiated
accident. Because of this, it is important that the FRAPFGR model predict not only the steady state
gas release, but also the amount of gas that remains within the grains and the amount of gas that is
currently residing on the grain boundaries for each axial and radial node. The grain boundary gas is
released during a fast transient due to cracking along the grain boundaries. To do this, gas release
data as well as electron probe microanalysis (EPMA) and X-ray fluorescence (XRF) data have been used
to validate that the model can accurately predict these parameters.
\\
\\
The basic layout of the FRAPFGR model is similar to the modified Massih model with the following
differences.

\subsubsection{Grain Growth Model}\label{grain-growth-model}

The FRAPFGR model accepts an input grain size that can be specified in the input. The default value
for this is \SInum{10}{\mu m} using the mean linear intercept (MLI) method. The FRAPFGR model
uses a grain growth model proposed by \cite{ref:Khorushii1999} given by

\begin{equation}
    \label{eq:FRAPFGR_grain_growth}
    \frac{da}{dt} = K\left( \frac{1}{a} - \frac{1}{a_{\max}} - \frac{1}{a_{ir}} \right)
\end{equation}

Where
\ind{   \(\frac{da}{dt}\)  =  Grain radius growth rate                                                                 \sinum{\mu-m/hour} }
\ind{   $K$            =  $\num{5.24E7}\exp\l\exp\left((\frac{- 32100}{T}\right)$                                                                     }
\ind{   \(T\)          =  Temperature                                                                              \sinum{K}           }
\ind{   \(a\)          =  Grain size                                                                               \sinum{\mu m}      }
\ind{   $a_{max}$      =  $\num{2.23E3}\left(\frac{- 7620}{T}\right)$                                                                      }
\ind{   $a_{ir}$       =  $\left[ \frac{50}{\dot{F}} \times \frac{1400}{T} \right] 326.5\exp\left(\frac{- 5620}{T}\right)$                     }
\ind{   \(\dot{F}\)    =  Fission rate \sinum{MW/tU}       }

Equation~\ref{eq:FRAPFGR_grain_growth} is solved by dividing the current time step into \SInum{1}{hour}
time steps and solving assuming constant rates within each sub-step.

\subsubsection{High Burnup Rim Thickness and Porosity}\label{high-burnup-rim-thickness-and-porosity}

The high burnup rim that is observed in the outer edge of high burnup pellets can be characterized
in terms of sub-micron grains and high porosity. These two items are modeled in the FRAPFGR model.
The size of the high burnup rim has been measured by optical microscopy (\cite{ref:Manzel2002}) and
is modeled using the equation

\begin{equation}
    \label{eq:high_burnup_rim_thickness}
    t_{rim} = \num{1.439E-6}BU^{4.427}
\end{equation}

Where,
\ind{   \(t_{rim}\)  =  Thickness of high burnup rim \sinum{\mu m} }
\ind{   \(BU\)       =  Pellet average burnup \sinum{GWd/MTU}       }

Figure~\ref{fig:hbu_rim_model} shows how the high burnup structure is modeled in FRAPFGR.  The
calculated value of \(t_{rim}\) sets a thickness on the pellet surface that is entirely restructured
grains. The grain size (MLI) for these grains is set at \SInum{0.15}{\mu m}. The next region, which has a
width also set by \(t_{rim}\), is composed of a mixture of restructured grains and non-restructured
grains. The fraction of restructured grains decreases linearly to zero across this thickness of this
region. If the temperature in a given axial node is greater than \SInum{1000}{\mDC} then no
restructured grains are assumed to form.

\begin{figure}
    \includegraphics[height=\figheight\textheight]{../media/hbu_rim_model}
    \caption{Modeling of the Pellet High Burnup Rim Structure}
    \label{fig:hbu_rim_model}
\end{figure}

In addition to the restructured grains, there is also a porosity increase within the high burnup
rim. The porosity is modeled based on a fit to observations on high burnup fuel
(\cite{ref:Spino1996}; \cite{ref:Une2001}; and \cite{ref:Manzel2000}). This model is given by

\begin{equation}
    P = \begin{cases}
        0                                          &  BU_{local} < \SInum{57}{GWd/MTU}  \\
        11.283 \ln \left( BU_{local}\right)-45.621 & BU_{local} \geq \SInum{57}{GWd/MTU}
    \end{cases}
\end{equation}

Where,
\ind{   \(P\)           =  Porosity increase in HBU rim structure (fraction of fuel theoretical density)  }
\ind{   \(BU_{local}\)  =  Local radial node burnup, \sinum{GWd/MTU}                                              }

This porosity is subtracted off the input theoretical density, which is used to calculate the
production in each radial node. Therefore, as the porosity in the rim increases, the power
production in the outer radial nodes is slightly decreased due to increase porosity.

\subsubsection{Diffusion Constant}\label{diffusion-constant-1}

The diffusion constant used in FRAPFGR is given by Equation~\ref{eq:FRAPFGR_diffusion_constant}.

\begin{equation}
    \label{eq:FRAPFGR_diffusion_constant}
    D \left( T \right) =
    \begin{cases}
        \num{1.51E- 17}\exp\left(- \frac{9508}{max(T,675)}\right)   & T < \SInum{1381}{K}                   \\
        \num{2.14E- 13}\exp\left(- \frac{22884}{T}\right)           & \SInum{1381}{K} \leq T < \SInum{1650}{K} \\
        \num{7.14433E- 10}\exp\left(-\frac{34879}{min(T,1850)}\right)    & \SInum{1650}{K} \leq T
    \end{cases}
\end{equation}

Where,
\ind{     \(D\)  =  Diffusion constant \sinum{m^{2}/s} }
\ind{     \(T\)  =  Temperature \sinum{K}               }

For non-restructured grains, Equation~\ref{eq:FRAPFGR_diffusion_constant_nonrestructured_grains} is used, up to a maximum
adjustment of 49.81. For restructure grains, there is no burnup
adjustment.

\begin{equation}
    \label{eq:FRAPFGR_diffusion_constant_nonrestructured_grains}
    D\left( T,BU \right) = D\left( T \right)\left(1 \times 10^{\frac{max\left( (BU - 21,)0 \right)}{40}} + 10 \times \frac{min\left( BU,12 \right)}{12} \right)
\end{equation}

Where,

\ind{   \(D\left( T,BU \right)\)  =  Diffusion constant adjusted for burnup \sinum{m^2/s}                                                         }
\ind{   \(D\left( T \right)\)     =  Temperature dependent diffusion constant given by Equation~\ref{eq:FRAPFGR_diffusion_constant} \sinum{m^2/s} }
\ind{   $BU$                      =  Local radial node burnup \sinum{GWd/MTU}                                                                         }

The diffusion constant is also modified for the effects of low power using an error function as
shown in Equation~\ref{eq:FRAPFGR_diffusion_constant_low_power}.

\begin{equation}
    \label{eq:FRAPFGR_diffusion_constant_low_power}
    D\left( T,BU,Power \right) = \frac{D\left( T,BU \right)}{2.5 - 1.5erf\left( Power - 3 \right)}
\end{equation}

Where,
\ind{    \(D\left( T,BU,Power \right)\)  =  Diffusion constant adjusted for burnup and power \sinum{m^{2}/s}                                                                   }
\ind{    \(D\left( T,BU \right)\)        =  Diffusion constant adjusted for burnup given by Equation~\ref{eq:FRAPFGR_diffusion_constant_nonrestructured_grains} \sinum{m^{2}/s} }
\ind{    \(Power\)                       =  Local radial node power \sinum{kW/ft}                                                                                                 }

\subsubsection{Gas Release}\label{section:gas-release-1}

Gas release calculations are performed separately for restructured grains and non-restructured
grains. For those nodes that contain both restructured and non-restructured grains, the releases
from each are combined based on the relative amount of each type of grain.
\\
\\
For the restructured grains, it is assumed that, because the grains are so small, all the gas
produced in the grain will diffuse out to the grain boundary. Therefore, the only gas that will
remain in these grains at the end of the time step is the gas that is re-solved back into the
grains.
\\
\\
The gas re-solved back into the grain is given by the resolution factor from Massih
(\cite{ref:Forsberg1985a}) . The gas that is in the grain for a given time step, \(i\), is given by

\begin{equation}
    \label{eq:FRAPFGR_gas_in_grain}
    GG_{i} = GB_{i - 1}\frac{f}{1 + f}
\end{equation}

Where,
\ind{   \(GG_{i}\)  =  Gas in grains \sinum{moles/m^{3}}           }
\ind{   \(GB_{i}\)  =  Gas on grain boundaries \sinum{moles/m^{3}} }

\begin{equation}
    \label{eq:FRAPFGR_f}
    f = \frac{\num{1.84E-14}a}{3D}
\end{equation}

Where,
\ind{   \(a\)  =  Grain radius       \SInum{0.075E-6}{m} (for restructured grains) }
\ind{   \(D\)  =  Diffusion constant \sinum{m^{2}/s}                                     }

For the non-restructured grains, the same formulas as those in Massih are used to calculate
diffusion from the grains except that the release is reduced to account for resolution during the
calculation of release.  The following terms are changed as follows:
\\
\\
From Equation~\ref{eq:forsberg_massih_G_b}, the following term is changed:

\begin{equation}
    \label{eq:FRAPFGR_mod_G_b}
    \sum_{}^{}{f_{n}G_{n}\left( \tau_{1} \right)} \rightarrow \frac{\sum_{}^{}{f_{n}G_{n}\left( \tau_{1} \right)}}{resolterm}
\end{equation}

From Equation~\ref{eq:forsberg_massih_q}, the following term is changed:

\begin{equation}
    \label{eq:FRAPFGR_mod_q}
    \left\lbrack - \sum_{n = 1}^{4}\left( \frac{f_{n}A_{n}}{B_{n}} \right) + funct\left( \mathrm{\Delta}\tau \right) \right\rbrack \rightarrow \frac{\left\lbrack - \sum_{n = 1}^{4}\left( \frac{f_{n}A_{n}}{B_{n}} \right) + funct\left( \mathrm{\Delta}\tau \right) \right\rbrack}{resolterm}
\end{equation}

Where,
\ind{   \(T\)  =  Temperature \sinum{K} }

and

\begin{equation}
    \label{eq:FRAPFGR_resolterm}
    resolterm   = max 
        \begin{cases}
            0.14009 e^{0.00282T} & T < \SInum{1528.77}{K}    \\
            22.976-0.0082T       & T \leq \SInum{1528.77}{K}
        \end{cases}
\end{equation}

In order for gas to be released from the grain boundaries, the saturation concentration must be
reached. The saturation concentration is given by

\begin{equation}
    \label{eq:FRAPFGR_saturation_concentration}
    g_{s} = \frac{3N_{s}}{2a}
\end{equation}

Where,

\ind{   \(g_{s}\)  =  Grain boundary saturation concentration                                          \sinum{moles/m^{3}} }
\ind{   \(N_{s}\)  =  Saturation area density given in Equation~\ref{eq:massih_gas_saturation_density} \sinum{moles/m^{2}} }
\ind{   \(a\)      =  Grain radius \sinum{m}             }

When the grain boundary gas concentration for a given radial node exceeds the saturation value for
the first time, all the gas on the grain boundary except 65\percent of the saturation value is
released.  From then on for that radial node, any gas above 65\percent of the saturation values is
released.
\\
\\
As discussed, for radial nodes that contain some restructured grains and some non-restructured
grains, the released gas is calculated as

\begin{equation}
    \label{eq:FRAPFGR_released_gas}
    {\Delta rel}_{tot} = {\Delta rel}_{1}\left( restructure^{2} \right) + {\Delta rel}_{2}\left( 1 - restructure^{2} \right)
\end{equation}

Where,
\ind{   \({\Delta rel}_{tot}\)  =  Total release from a radial node                       \sinum{moles/m^{3}} }
\ind{   \({\Delta rel}_{1}\)    =  Release from restructured grains in a radial node      \sinum{moles/m^{3}} }
\ind{   \({\Delta rel}_{2}\)    =  Release from non- restructured grains in a radial node \sinum{moles/m^{3}} }
\ind{   \(restructure\)         =  Fraction of restructured grains in radial node                         }

As with the MASSIH model, an athermal release term of 1\percent for every \SInum{10}{GWd/MTU} beyond
\SInum{40}{GWd/MTU} is added on if the predicted release is less than 5\percent to account for the
observed gas release from rods with very low power at high burnup.

\subsection{Transient Gas Release} \label{section:transient-gas-release}

The transient release of fission gas is highly dependent on the location of the gas in the fuel
pellet, both radially, and in each radial node the location (in the grains versus on the grain
boundaries) of the gas.  Because of this, the transient gas release model in FAST-1.0 may only be
used if initialized with the {FRAPFGR} model.  This model has been developed specifically to predict
the location of fission gas within the pellets. This transient release model is a burst release (not
diffusion release) model as described below:

\begin{itemize}

    \item All grain boundary gas for a given radial node is released when the temperature exceeds
        \SInum{2000}{\mDF} (\SInum{1093}{\mDC}).

    \item All gas in the restructured grains (matrix) of the high burnup structure for a given radial
        node is released when the temperature exceeds \SInum{3300}{\mDF} (\SInum{1816}{\mDC}).

    \item 5\percent of the gas in the un-restructured grains (matrix) for a given radial node is
        released when the temperature exceeds \SInum{3300}{\mDF} (\SInum{1816}{\mDC}).

\end{itemize}

This release model was developed to predict the measured release data from RIA experimental tests in
CABRI and NSRR. (See data comparisons in \colorbox{yellow}{Geelhood and Luscher (2014b)}.)

{\color{red}Need to add Geelhood and Luscher 2014b to the references}

\subsection{Nitrogen Release} \label{section:nitrogen-release}

The release of nitrogen initially present in fuel material from fabrication occurs as a result of a
diffusion transport mechanism.  The release of nitrogen affects the rod internal pressure and the
gas conductivity. The model proposed by \cite{Booth1957} is used, given the following
assumptions:

\begin{itemize}

    \item The initial concentration of diffusing substance, $C$, is uniform throughout a sphere of
        radius, $a$.
    \item Transport of material does not occur from the external phase (gaseous nitrogen) back into
        the initial carrier medium.

\end{itemize}

The governing equation is

\begin{equation}
    \label{eq:nitrogen_release}
    r\frac{\partial C}{\partial t} = D\left( \frac{\partial^{2}}{\partial r^{2}}\left( C r \right) \right)
\end{equation}

Where,
\ind{    \(r\)  =  Radial location                        \sinum{m}         }
\ind{    \(C\)  =  Concentration of diffusing substance               }
\ind{    \(t\)  =  Time                                   \sinum{s}         }
\ind{    \(D\)  =  Diffusion coefficient                  \sinum{m^{2}/s} }

With the following boundary conditions:
\begin{equation}
    \label{eq:nitrogen_release_bcs}
    C = \begin{cases}
        0  & \text{at $r = a$} \\
        C  & \text{at $t = 0$}
    \end{cases}
\end{equation}

By applying a series solution method, the fractional release of the diffusing substance (nitrogen)
can be approximated based on the value of
\(B\):

\begin{equation}
    \label{eq:nitrogen_release_b}
    B = \pi^{2}\frac{D_{N_{2}}}{a^{2}}t
\end{equation}

Where,
\ind{    \(\frac{D_{N_{2}}}{a^{2}}\)  =  Temperature-dependent diffusion coefficient for nitrogen divided by the effective diffusion radius squared \sinum{1/s} }
\ind{     \(t\)                       =  Time from the start of diffusion                                      \sinum{s}        }

The fraction of nitrogen released (\(F_{N_{2}}\)) as of time (\(t\)) is shown in
Equation~\ref{eq:nitrogen_release_fraction} equals

\begin{equation}
    \label{eq:nitrogen_release_fraction}
    F_{N_{2}} = \begin{cases}
                6\sqrt{\left[ \frac{D_{N_{2}}}{a^{2}}\frac{t}{\pi} \right]} - 3\frac{D_{N_{2}}}{a^{2}}t & B \leq 1  \\
                1 - \frac{6e^{- B}}{\pi^{2}}                                                            & B > 1
    \end{cases}
\end{equation}

From the experimental data of \cite{ref:Ferrari1963} and \cite{ref:Ferrari1964}

\begin{equation}
    \label{eq:nitrogen_release_d_a}
    \frac{D_{N_{2}}}{a^{2}} = 1.73\exp\left( \frac{33400}{1.9869T} \right)
\end{equation}

Where,
\ind{    \(T\)  = Temperature \sinum{K} }

\subsection{Helium Production and Release} \label{section:helium-production-and-release}

\subsubsection{Steady-state Production}\label{section:steady-state-production}

Helium is produced at different rates in \element{UO}{2} and MOX.  The release of helium affects the
rod internal pressure and the gas conductivity.
\\
\\
For \element{UO}{2}, helium production is given by

\begin{equation}
    \label{eq:he_prod_uo2}
    \text{He}_{\text{prod}} = \num{1.297E-18}Q\times t\times SA\times PR
\end{equation}

Where,
\ind{   $\text{He}_{\text{prod}}$  =  Helium produced for a given axial node \sinum{moles}              }
\ind{   $Q$                        =  Surface heat flux                      \sinum{W/m^2}          }
\ind{   $t$                        =  Time                                   \sinum{s}                  }
\ind{   $SA$                       =  Axial node surface area                \sinum{m^2}            }
\ind{   $PR$                       =  Fission gas production rate            \sinum{atoms/100-fissions} }

For MOX, a formula has been developed as a function of \element{Pu}{} concentration
(\element{Pu}{}) and burnup (\(BU\), in \sinum{GWd/MTU}):

\begin{equation}
    \label{eq:he_prod_mox}
    \text{He}_{\text{prod}} = \left( A_{1}\melement{Pu}{} + A_{2} \right)BU^{2} + \left( B_{1}\melement{Pu}{} + B_{2} \right)BU
\end{equation}

Table~\ref{tab:fitting_parameters_helium_mox} shows the fitting parameters that should be used for
reactor-grade plutonium and weapons-grade plutonium.

\renewcommand{\captiontext}{Fitting Parameters for Helium in MOX}

\begin{longtable}[c]{A{2.0cm}A{4.0cm}A{4.0cm}}
    \caption{\captiontext}  \label{tab:fitting_parameters_helium_mox}   \\  \hline
        \textbf{Parameter}              & \textbf{Reactor-Grade Plutonium}   & \textbf{Weapons-Grade Plutonium}       \\ \hline    
    \endfirsthead
    \caption{\captiontext (continued)}  \\  \hline
        \textbf{Parameter}              & \textbf{Reactor-Grade Plutonium}   & \textbf{Weapons-Grade Plutonium}       \\ \hline    
    \endhead
            $A_{1}$         & $\num{1.5350E-4}$      & $-\num{2.4360E-4}$       \\
            $A_{2}$         & $\num{2.1490E-3}$    & $\num{3.6059E-3}$        \\
            $B_{1}$         & $-\num{2.9080E-3}$   & $\num{3.3790E-3}$        \\
            $B_{2}$         & $\num{9.7223E-2}$    & $\num{5.3658E-2}$        \\
\end{longtable}

The above equations calculate the amount of helium produced as a function of time. In order to
calculate the helium released to the void volume, an approach similar to the approach for nitrogen
release is used. By applying a series solution method, the fractional release of the diffusing
substance (helium) can be approximated based on the value of \(B\):

\begin{equation}
    \label{eq:helium_release_b}
    B = \pi^{2}\frac{D_{\text{He}}}{a^{2}}t
\end{equation}

Where,
\ind{   \(\frac{D_{\text{He}}}{a^{2}}\)  =  Temperature dependent diffusion coefficient for helium divided by the effective diffusion radius squared \sinum{1/s} }
\ind{   \(t\)                            =  Time from the start of diffusion \sinum{s}                         }

If \(t \leq \frac{1}{\pi^{2}\frac{D_{\text{He}}}{a^{2}}}\) then the fraction of helium released,
\(F_{\text{He}}\), as of time, \(t\), equals

\begin{equation}
    \label{eq:he_release_fraction_1}
    F_{\text{He}} = 4\sqrt{\left\lbrack \frac{D_{\text{He}}}{a^{2}}\frac{t}{\pi} \right\rbrack} - \frac{3D_{\text{He}}}{2a^{2}}t
\end{equation}

If this fraction is greater than 0.57, then, when \(B < 1\), the fraction of helium released as a
function of time, \(t\), equals

\begin{equation}
    \label{eq:he_release_fraction_2}
    F_{\text{He}} = 1 + \frac{0.607927e^{- B} - 0.653644}{B}
\end{equation}

and, when \(B > 1\),

\begin{equation}
    \label{eq:he_release_fraction_3}
    F_{\text{He}} = 1
\end{equation}

When \(t > \frac{1}{\pi^{2}\frac{D_{\text{He}}}{a^{2}}}\), the fraction of helium released is equal
to that shown in Equations~\ref{eq:he_release_fraction_2} and~\ref{eq:he_release_fraction_3} . The
diffusion coefficient of Helium divided by the effective radius squared is shown in
Equation~\ref{eq:he_release_d_a}.

\begin{equation}
    \label{eq:he_release_d_a}
    \frac{D_{\text{He}}}{a^{2}} = 
        \begin{cases}
            \num{0.452847E-10},                                                                               & T \leq \SInum{873}{K} \\
            \num{0.28E-5}\exp \left( \frac{\num{4.0E4}}{1.986}\left( \frac{1}{1673} - \frac{1}{T} \right) \right) & T > \SInum{873}{K} 
        \end{cases}
\end{equation}

\subsubsection{Integral Fuel Burnable Absorber Production} \label{section:integral-fuel-burnable-absorber-production}

Some fuel designs use a thin layer of \element{ZrB}{2} applied to the surface of the pellets to act
as an integral fuel burnable absorber (IFBA). The use of such coatings produces a large amount of
helium. The following empirical correlation was fit to results from Monte Carlo N-Particle (MCNP), a
neutron transport code, for helium production from IFBA liners.

\begin{equation}
    \label{eq:he_ifba_production}
    \text{He}_{\text{prod}} = - \left( A_{1}IFBA + A_{2} \right){B10}^{2} + \left( B_{1}IFBA + B_{2} \right)B10
\end{equation}

Where,
\ind{    \(\text{He}_{\text{prod}}\)      =   Helium production \sinum{atoms-He/cm^{3}-s}                    }
\ind{     \(\text{IFBA}\)                 =   Percent of fuel rods in a core containing IFBA liners \sinum{\percent} (valid only between 10 and 50\percent)            }
\ind{     \(B10\)                         =   Boron-10 enrichment \sinum{\percent} (valid from 0 to 90\percent) }
\ind{     \(A_{1}\)                       =   $\num{6.23309E-9}$                                    }
\ind{     \(A_{2}\)                       =   $\num{7.02006E-7}$                                    }
\ind{     \(B_{1}\)                       =   $-\num{1.35675E-7}$                                   }
\ind{     \(A_{2}\)                       =   $\num{3.1506E-4}$                                     }

It can be seen from Equation~\ref{eq:he_ifba_production} that the helium production rate is a
function of the number of IFBA rods in a core and the boron-10 enrichment. Helium is produced as the
boron-10 burns out until there is no more boron-10 in the liner. The rate of boron-10 depletion is
equal to the helium production rate. The depletion of boron-10 is calculated in the code and the
remaining boron-10 enrichment, \(B10\) in Equation~\ref{eq:he_ifba_production}, at the end of the
time step is used to calculate the helium production for the next time step. It is assumed in the
code that all helium produced in the \element{ZrB}{2} coatings is released directly to the rod-free
volume.

