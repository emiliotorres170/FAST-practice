\section{Corrosion and Hydrogen Pickup}\label{section:corrosion-and-hydrogen-pickup}

There are two types of corrosion models in FAST-1.0: corrosion under normal LWR operating conditions
and under high temperature steam-oxidation.

\subsection{LWR Waterside Corrosion Models}\label{section:lwr-waterside-corrosion-models}

\subsubsection{PWR Conditions}\label{section:pwr-conditions}

For Zircaloy-4 under pressurized-water reactor (PWR) conditions, a cubic rate law for
corrosion-layer thickness as a function of time is applied until a transition thickness is attained
\cite{ref:Garzarolli1982}:

\begin{equation}
    \label{eq:cubic_rate_law_function_of_time}
    \frac{d\left( s^{3} \right)}{dt} = \frac{A}{s^{2}}\exp\left( - \frac{Q_{1}}{RT_{1}} \right)
\end{equation}

In FAST-1.0, this equation is integrated without regard to the feedback between oxide layer
thickness and oxide metal interface temperature to obtain

\begin{equation}
    \label{eq:cubic_rate_law_integrated}
    s_{i + 1} = \left( 3A\exp\left( - \frac{Q_{1}}{RT_{1}} \right)\left( t_{i + 1} - t_{i} \right) + {s_{i}}^{3} \right)^{\frac{1}{3}}
\end{equation}

Where,
\ind{   $i$,$\ i + 1$    =  Refers to beginning ($i$) and end ($i + 1$) of current time step                 }
\ind{   $s$              =  Oxide thickness                                                       \sinum{m}            }
\ind{   $A$              =  $\num{6.3E9}$                                                    \sinum{m^{3}/day}  }
\ind{   $Q_{1}$          =  See Table~\ref{tab:enthalpy_transition_thickness_PWR}                                }
\ind{   $R$              =  1.98                                                                  \sinum{cal/mol-K}    }
\ind{   $T_{1}$          =  Metal-oxide interface temperature                                     \sinum{K}            }
\ind{   $t$              =  Time                                                                  \sinum{days}         }

After the transition thickness is attained, a flux-dependent linear rate law is applied, with the
rate constant being an Arrhenius function of oxide-metal interface temperature:

\begin{equation}
    \label{eq:flux_dependent_linear_rate_law}
    \frac{ds}{dt} = \left( C_{0} + U\left( M\Phi \right)^{P} \right)\exp\left( - \frac{Q_{2}}{RT_{1}} \right)
\end{equation}

Because there is significant feedback between oxide-layer thickness and oxide-metal interface
temperature, the oxide thickness is converted to weight gain, and the approximate integral solution
from \cite{ref:Garzarolli1982} is used. This solution has the form

\begin{equation}
    \label{eq:integral_solution_weight_gain}
    \Delta w_{i + 1} = \Delta w_{i} + \frac{R{T_{0}}^{2}\lambda}{\gamma Q_{2}q^{\prime \prime}}{\ln\left\lbrack 1 - \frac{\gamma Q_{2}q^{\prime \prime }}{R{T_{0}}^{2}\lambda}k_{0}e^{\frac{- Q_{2}}{RT_{0}}}e^{\frac{\gamma Q_{2}q^{\prime \prime }\Delta w_{i}}{R{T_{0}}^{2}\lambda}}\left( t_{i + 1} - t_{i} \right) \right\rbrack}^{- 1}
\end{equation}

{\color{red}Consider expressing the above as,}
\begin{equation}
    \label{eq:weight_gain_above_transition_thickness_ZIRLO_OptZIRLO}
    \Delta w_{i + 1} = \Delta w_{i} + \frac{R{T_{0}}^{2}\lambda}{2\gamma Q_{2}q^{\prime \prime}}{\ln\left\lbrack 1 - \frac{\gamma Q_{2}q^{\prime \prime}}{R{T_{0}}^{2}\lambda}k_{0}\exp\left( \frac{- Q_{2}}{RT_{0}}+\frac{\gamma Q_{2}q^{\prime \prime}2\Delta w_{i}}{R{T_{0}}^{2}\lambda}\right)\left( t_{i + 1} - t_{i} \right) \right\rbrack}^{- 1}
\end{equation}

Weight gain can be converted to thickness using the following formula:

\begin{equation}
    \label{eq:weight_gain_oxide_thickness}
    s = \frac{\Delta w\gamma}{100}
\end{equation}


Where,
\ind{   $i$,$ i + 1$  =  Refers to beginning ($i$) and end ($i + 1$) of current time step                                                         }
\ind{   $s$             =  Oxide thickness                                                           \sinum{m}                                                }
\ind{   $\Delta w$      =  Weight gain                                                               \sinum{g/cm^{2}}                          }
\ind{   $R$             =  1.98                                                                      \sinum{cal/mol-K}                                        }
\ind{   $T_{0}$         =  Oxide-to-water interface temperature                                      \sinum{K}                                                }
\ind{   $\lambda$       =  Oxide thermal conductivity                                                \sinum{W/cm-K}                                           }
\ind{   $\gamma$        =  0.6789                                                                    \sinum{cm^{3}/g}                          }
\ind{   $Q_{2}$         =  See Table~\ref{tab:enthalpy_transition_thickness_PWR}                                                                        }
\ind{   $q^{\prime \prime }$        =  Heat flux                                                    \sinum{W/cm^{2}}                          }
\ind{   $k_{0}$           =  $11863 + \num{3.5E4}\left( \num{1.91E- 15}\Phi \right)^{0.24}$                                                     }
\ind{   $\Phi$          =  Fast neutron flux (E$>\SInum{1}{MeV}$) \sinum{n/cm^{2}/s} }
\ind{   $t$             =  Time                                                                      \sinum{days}                                             }

\renewcommand{\captiontext}{Enthalpy and Transition Thickness under PWR Conditions.  (\cite{ref:Garzarolli1982})} 
\begin{longtable}[c]{A{4.0cm}A{2.0cm}A{2.0cm}A{5.0cm}}
    \caption{\captiontext}  \label{tab:enthalpy_transition_thickness_PWR}    \\ \hline
            \textbf{Cladding Type}                       &   \textbf{$Q_{1}$ \sinum{cal/mol}}       &  \textbf{$Q_{2}$   \sinum{cal/mol}}     &   \textbf{Transition Thickness \sinum{\mu m}}  \\     \hline
    \endfirsthead
    \caption[]{\captiontext (continued)}  \\ \hline
            \textbf{Cladding Type}                       &   \textbf{$Q_{1}$ \sinum{cal/mol}}       &  \textbf{$Q_{2}$   \sinum{cal/mol}}     &   \textbf{Transition Thickness \sinum{\mu m}}  \\     \hline
    \endhead
            Zirc-4                                       &   32289                   & 27354                     &   2        \\
            M5                                           &   27446                   & 29816                     &   7        \\
            ZIRLO\TM                                     &   32289                   & 27080                     &   2        \\
            Optimized ZIRLO\TM                           &   32289                   & 27354                     &   2        \\
\end{longtable}

Above the transition thickness, if the oxide thickness is less than \SInum{80}{\mu m} then use \(2\Delta
w_{i}\) in Equation~\ref{eq:weight_gain_oxide_thickness} and then divide the second term of
Equation~\ref{eq:integral_solution_weight_gain} by 2 as shown in
Equation~\ref{eq:weight_gain_above_transition_thickness_ZIRLO_OptZIRLO}.

\begin{equation}
    \label{eq:weight_gain_above_transition_thickness_ZIRLO_OptZIRLO}
    \Delta w_{i + 1} = \Delta w_{i} + \frac{R{T_{0}}^{2}\lambda}{2\gamma Q_{2}q^{\prime \prime}}{\ln\left\lbrack 1 - \frac{\gamma Q_{2}q^{\prime \prime}}{R{T_{0}}^{2}\lambda}k_{0}e^{\frac{- Q_{2}}{RT_{0}}}e^{\frac{\gamma Q_{2}q^{\prime \prime}2\Delta w_{i}}{R{T_{0}}^{2}\lambda}}\left( t_{i + 1} - t_{i} \right) \right\rbrack}^{- 1}
\end{equation}

{\color{red}Consider the following}
\begin{equation}
    \label{eq:weight_gain_above_transition_thickness_ZIRLO_OptZIRLO}
    \Delta w_{i + 1} = \Delta w_{i} + \frac{R{T_{0}}^{2}\lambda}{2\gamma Q_{2}q^{\prime \prime}}{\ln\left\lbrack 1 - \frac{\gamma Q_{2}q^{\prime \prime}}{R{T_{0}}^{2}\lambda}k_{0}\exp \left( \frac{- Q_{2}}{RT_{0}} + \frac{2\gamma Q_{2}q^{\prime \prime}\Delta w_{i}}{R{T_{0}}^{2}\lambda}\right) \left( t_{i + 1} - t_{i} \right) \right\rbrack}^{- 1}
\end{equation}

\subsubsection{BWR Conditions}\label{section:bwr-conditions}

For Zircaloy-2 under boiling-water reactor (BWR) conditions, a flux-dependent linear rate law is
applied, with the rate constant being an Arrhenius function of oxide-metal interface temperature:

\begin{equation}
    \label{eq:Zirc2_ds_dt}
    \frac{ds}{dt} = K\exp\left( - \frac{Q}{RT_{1}} \right) \left\lbrack 1 + Cq^{\prime \prime }\exp \left( \frac{Q}{RT_{1}} \right) \right\rbrack
\end{equation}

Because there is significant feedback between oxide-layer thickness and oxide-metal interface
temperature, the oxide thickness is converted to weight gain (as shown in
Equation~\ref{eq:weight_gain_oxide_thickness}), and the approximate integral solution from
(\cite{ref:Garzarolli1982}) is used. This solution has the form

\begin{equation}
    \label{eq:Zirc2_weight_gain}
    \Delta w_{i + 1} = \Delta w_{i} + \frac{R{T_{0}}^{2}\lambda}{\gamma Q q^{\prime \prime}}{\ln\left\lbrack 1 - \frac{\gamma Q q^{\prime \prime}}{R{T_{0}}^{2}\lambda}ke^{\frac{- Q}{RT_{0}}}e^{\frac{\gamma Q q^{\prime \prime}\Delta w_{i}}{R{T_{0}}^{2}\lambda}}\left( t_{i + 1} - t_{i} \right) \right\rbrack}^{- 1} + Ck\left( t_{i + 1} - t_{i} \right)q^{\prime \prime}
\end{equation}

{\color{red}Consider writing the following}
\begin{equation}
    \label{eq:Zirc2_weight_gain}
    \Delta w_{i + 1} = \Delta w_{i} + \frac{R{T_{0}}^{2}\lambda}{\gamma Q q^{\prime \prime}}{\ln\left\lbrack 1 - \frac{\gamma Q q^{\prime \prime}}{R{T_{0}}^{2}\lambda}k \exp \left(\frac{- Q}{RT_{0}} + \frac{\gamma Q q^{\prime \prime}\Delta w_{i}}{R{T_{0}}^{2}\lambda}\right) \left( t_{i + 1} - t_{i} \right) \right\rbrack}^{- 1} + Ck\left( t_{i + 1} - t_{i} \right)q^{\prime \prime}
\end{equation}

Where,
\ind{   $i$,$\ i + 1$  =  Refers to beginning ($i$) and end ($i + 1$) of current time step                                 }
\ind{   $s$              =  Oxide thickness                                                         \sinum{m}                             }
\ind{   $\Delta w$       =  Weight gain                                                             \sinum{g/cm^{2}}       }
\ind{   $R$              =  1.98                                                                    \sinum{cal/mol-K}                     }
\ind{   $T_{0}$          =  Oxide-to-water interface temperature                                    \sinum{K}                             }
\ind{   $\lambda$        =  Oxide thermal conductivity                                              \sinum{W/cm-K}                        }
\ind{   $\gamma$         =  0.6789                                                                  \sinum{cm^{3}/g}       }
\ind{   $Q$              =  27350                                                                   \sinum{cal/mol}                       }
\ind{   $q^{\prime \prime }$         =  Heat flux                                                   \sinum{W/cm^{2}}       }
\ind{   $k$              =  11800                                                                   \sinum{g/cm^{2}-day} }
\ind{   $\Phi$           =  Fast neutron flux (E\textgreater{}1 MeV) \sinum{n/cm^{2}-s}     }
\ind{   $C$              =  $\num{2.5E-16}$                                                         \sinum{m^{2}/W}        }
\ind{   $t$              =  Time                                                                    \sinum{days}                          }

To achieve numerical stability, the rate equation is integrated across each time step and applied to
calculated corrosion layer increments per time step, which are accumulated to calculate cumulative
layer thickness as a function of axial position (axial node) along the fuel rod.

\subsection{Hydrogen Pickup Fraction}\label{section:hydrogen-pickup-fraction}

The fraction of the hydrogen liberated by the metal-water corrosion reaction that is absorbed
locally by the cladding is called the pickup fraction. For PWR conditions, a constant hydrogen
pickup fraction has been found to be applicable. Table~\ref{tab:hydrogen_pickup_fraction_pwr} lists
the hydrogen pickup fraction for the claddings available in FAST.

\renewcommand{\captiontext}{Hydrogen pickup fraction under PWR conditions. (\cite{ref:Geelhood2011a})} 
\begin{longtable}[c]{A{6.0cm} A{6.0cm}}
    \caption{\captiontext}    \label{tab:hydrogen_pickup_fraction_pwr}                         \\ \hline
        \textbf{Cladding Type}              &       \textbf{Pickup Fraction}  \\  \hline
    \endfirsthead
    \caption[]{\captiontext (continued)}                         \\ \hline
        \textbf{Cladding Type}              &       \textbf{Pickup Fraction}  \\  \hline
    \endhead
        Zirc-4                              &       0.15             \\
        M5                                  &       0.10             \\
        ZIRLO\TM                            &       0.175            \\
        Optimized ZIRLO\TM                  &       0.175            \\
\end{longtable}

For BWR conditions, a constant hydrogen pickup fraction does not fit the observed hydrogen
concentration data. Therefore, FAST-1.0 uses a burnup-dependent hydrogen concentration model
(\cite{ref:Geelhood2008c}). For Zircaloy-2 prior to 1998 (when the vendors did not have tight
control over concentration and second-phase precipitate particle size),
Equation~\ref{eq:Zirc2_pre_1998_hydrogen} is used.

\begin{equation}
    \label{eq:Zirc2_pre_1998_hydrogen}
        H_{conc} = 
            \begin{cases}
                47.8 \exp \left(- \frac{1.3}{1 + BU}\right) + 0.316BU               & BU < \SInum{50}{GWd/MTU} \\
                28.9 + \exp \left( 0.117\left( BU - 20 \right) \right)              & BU \geq \SInum{50}{GWd/MTU} 
            \end{cases}
\end{equation}

For modern Zircaloy-2 since 1998 (when the vendors have had tight control over concentration and
second phase precipitate particle size), Equation~\ref{eq:Zirc2_post_1998_hydrogen} is used.

\begin{equation}
    \label{eq:Zirc2_post_1998_hydrogen}
    H_{conc} = 22.8 + \exp \left( 0.117\left( BU - 20 \right) \right)
\end{equation}

Where,
\ind{   $H_{conc}$           =    Hydrogen concentration \sinum{ppm}  }
\ind{   $BU$                 =    Local burnup \sinum{GWd/MTU}        }

\subsection{High-Temperature Corrosion}\label{section:high-temperature-corrosion}

In FAST, the initial oxide thickness from in-reactor behavior is calculated using the previously
described correlations, or input to the code. During a reduced-cooling, high-temperature transient,
such as LOCA, the cladding temperature can become very hot relative to steady-state operations.
Under these circumstances, a large corrosion layer could form in a matter of seconds to minutes.
This corrosion layer both erodes the structural thickness of the cladding and produces excess energy
term resulting from the exothermic Zirconium-Water reaction. FAST contains two high-temperature
corrosion models, the Cathcart/Pawel (\cite{ref:Cathcart1977}) model and the
Baker/Just (\cite{ref:Baker1962}) model, that are selected using the input
variable $modmw$. In addition, the option exists to not model high-temperature corrosion. Guidance
on model selection is given in the input instructions shown in \colorbox{yellow}{Appendix A}.
\\
\\
The Cathcart/Pawel (default) model is activated in FAST-1.0 when the cladding temperature exceeds
\SInum{1073}{K} (\SInum{800}{\mDC}). The Baker/Just model is activated in FAST-1.0 when the cladding
temperature exceeds \SInum{1000}{K} (\SInum{727}{\mDC}). A derivation of these models and discussion
of extrapolation to lower temperatures than the models were originally validated for is provided in
\colorbox{yellow}{Appendix G}. These models are described below.
\\
\\
Both the Cathcart-Pawel and Baker-Just models are of the following form:

\begin{equation}
    \label{eq:cp_bj_base_form}
    \frac{dK}{dt} = \frac{1}{K}A\exp\left(\frac{- B}{RT}\right)
\end{equation}

Where,
\ind{   $K$              =  Oxide thickness \sinum{m} }
\ind{   $t$              =  Time            \sinum{s} }
\ind{   $T$              =  Temperature     \sinum{K} }
\ind{   $A$,$B$,$R$  =  Constants           }

This equation can be integrated and rearranged to the following form:

\begin{equation}
    \label{eq:cp_bj_K2}
    K_{2} = \sqrt{K_{1}^{2} + 2A  \exp\left(\frac{- B}{RT}\right)  \Delta t}
\end{equation}

Where
\ind{   $K_{1}$  =  Oxide thickness at beginning of time step \sinum{m} }
\ind{   $K_{2}$  =  Oxide thickness at end of time step       \sinum{m} }

Table~\ref{tab:constants_cp_bj_models} shows the parameters that are used for the Cathcart-Pawel and
Baker-Just models.
 
\renewcommand{\captiontext}{Constants for Cathcart-Pawel and Baker-Just Models} 
\begin{longtable}[c]{A{2.0cm}A{4.0cm}A{4.0cm}}
    \caption{\captiontext}  \label{tab:constants_cp_bj_models}  \\ \hline
        \textbf{Constant}                           & \textbf{Cathcart-Pawel}                   & \textbf{Baker-Just}                                   \\  \hline
    \endfirsthead
    \caption[]{\captiontext (continued)}   \\ \hline
        \textbf{Constant}                           & \textbf{Cathcart-Pawel}                   & \textbf{Baker-Just}                                   \\  \hline
    \endhead
        \(A\)                                       & $\SInum{1.126E-6}{m^{2}/s}$               & $\SInum{9.415E-5}{m^{2}/s}$                               \\
        \(B\)                                       & $\SInum{1.502E5}{J/mol}$                  & $\SInum{4.550E4}{cal/mol}$                 \\
        \(R\)                                       & $\SInum{8.314}{J/mol-K}$                  & $\SInum{1.987}{cal/mol-K}$                              \\
\end{longtable}

For the Cathcart-Pawel model, the user can specify that the weight gain be calculated assuming
perfect stoichiometry of the oxide, or by assuming a stoichiometric gradient \textit{iStoicGrad}. It
can be seen from Equation~\ref{eq:cp_bj_base_form} that the rate of oxidation is inversely
proportional to the oxide layer thickness. In FAST there are two ways of treating the initial oxide
thickness layer that are selected using the input variable, ProtectiveOxide. If $ProtectiveOxide =
0$, the initial oxide from steady state is included with the high temperature oxidation, so the rate
of oxidation for a previously oxidized rod is lower than for a rod with no oxide. If
$ProtectiveOxide = 1$, the initial oxide from steady-state is not included with the high-temperature
oxidation, so the rate of oxidation for a previously oxidized rod is the same as the rate for a rod
with no oxide.
\\
\\
FAST calculates the oxidation of the outer rod surface and, if the inner cladding surface is in
contact with steam (i.e., the rod has burst), the oxidation of the inner rod surface. From these
inner diameter (ID) and outer diameter (OD) oxide layer thicknesses, FAST calculates the
oxygen-stabilized alpha layer, the oxygen uptake, the metal water reactor energy, and equivalent
cladding reacted (ECR). For the stress calculations, FAST reduces the wall thickness based on the
thinning from the oxide layer growth. No strength is attributed to the oxide layer.  FAST-1.0
includes an option to calculate ID oxidation regardless of rupture above a specified burnup
\textit{BuOxide}.

