%-------------------------------------------------------------------------%
% Solution Scheme
%-------------------------------------------------------------------------%
\section{Solution Scheme}\addcontentsline{\arabic{section}-toc}{subsection}{Solution Scheme}

FAST-1.0 has two solution schemes for solving the conduction equation in order to obtain the
temperature profile; one is for the steady-state (time-independent) solution and the other is for
the transient (time-dependent) solution.
%-------------------------------------------------------------------------%
% Steady State
%-------------------------------------------------------------------------%
\subsection{Steady State} \label{section:steady-state}

FAST code iteratively calculates the interrelated effects of fuel and cladding temperature, rod
internal gas pressure, fuel and cladding deformation, release of fission product gases, fuel
swelling and densification, cladding thermal expansion and irradiation-induced growth, cladding
corrosion and hydriding, and crud deposition for a given buildup rate as functions of time and
fuel-rod-specific power.
\\
\\
\noindent
The calculated procedure is illustrated in Figure~\ref{fig:fast-flowchart}, with a simplified
flowchart of FAST. A detailed flowchart is provided in
Section~\ref{chapter:calculation-cladding-tsurf}. The calculation begins by processing input data.
Next, the initial fuel rod state is determined through a self-initialization calculation. Time is
advanced according to the input-specified time-step size, a steady-state solution is performed, and
the new fuel rod state is determined. The new fuel rod state provides the initial state conditions
for the next time step. The calculations are cycled in this manner for the user-specified number of
time steps.
\\
\\
The solution for each time step consists of 

\begin{enumerate}

    \item Calculating the temperature of the fuel and the cladding 

    \item Calculating fuel and cladding deformation 
        
    \item Calculating the fission product generation and release, void volume, and fuel rod internal
        gas pressure 
        
\end{enumerate}

Each calculation is made in a separate subcode. As shown in Figure~\ref{fig:fast-flowchart}, the
fuel rod response for each time step is determined by repeated cycling through two nested loops of
iterative calculations until the fuel-cladding gap temperature difference and internal gas pressure
converge.
\\
\\
For the FRACAS-I (\cite{ref:Bohn1977}) mechanical model, the fuel temperature and
deformation are alternately calculated in the inner loop. On the first cycle through this loop for
each time step, the gap conductance is computed using the fuel-cladding gap size from the previous
time step. Then the fuel rod temperature distribution is computed. This temperature distribution
feeds the deformation calculation by influencing the fuel and cladding thermal expansions and the
cladding stress-strain relation. An updated fuel-cladding gap size is calculated and used in the gap
conductance calculation on the next cycle through the inner loop. The cyclic process through the
inner loop is repeated until two successive cycles calculate essentially the same temperature
distribution.
\\
\\
The outer loop of calculations is cycled in a manner similar to that of the inner loop, but with the
amount of internal gas being determined during each iteration. The calculation alternates between
the fuel rod void volume-gas pressure calculation and the fuel rod temperature-deformation
calculation. On the first cycle through the outer loop for each time step, the gas pressure from the
previous time step is used. For each cycle through the outer loop, the number of gas moles is
calculated and the updated gas pressure computed and fed back to the deformation and temperature
calculations (the inner loop). The calculations are cycled until two successive cycles calculate
essentially the same gas pressure, and then a new power-time step is begun.

\begin{figure}
    \includegraphics[height=\figheight\textheight]{../media/image1}
    \caption{Simplified FAST flowchart}
    \label{fig:fast-flowchart}
\end{figure}
%-------------------------------------------------------------------------%
% Transient
%-------------------------------------------------------------------------%
\subsection{Transient} \label{section:transient}
The transient solution to the conduction equation is similar to the steady-state solution described
above in Section~\ref{section:steady-state} with the addition of the time-dependent terms (density,
specific heat and time).  The code iteratively calculate the interrelated effects of fuel and
cladding temperature, fuel rod plenum temperature, fuel and cladding deformation, and rod internal
gas pressure. Charts of the overall flow of the computations are shown in
Figures~\ref{fig:FAST-flowchart-part1} through~\ref{fig:FAST-flowchart-part3}. The input
requirements and initialization procedure are shown in Figure~\ref{fig:FAST-flowchart-part1}; the
temperature, mechanical response, and pressure calculations are shown in
Figure~\ref{fig:FAST-flowchart-part2}; and the cladding oxidation, local cladding ballooning, and
fission gas release calculations are shown in Figure~\ref{fig:FAST-flowchart-part3}.
\\
\\
As shown in Figure~\ref{fig:FAST-flowchart-part2}, the temperature, mechanical response, and
internal gas pressure calculations are performed iteratively so that all significant interactions
are taken into account.  For example, the deformation of the cladding affects the fuel rod internal
gas pressure because the internal volume of the rod is changed. The deformation of the cladding also
affects the temperature of the fuel and cladding because the flow of heat from the fuel to the
cladding is dependent on the fuel-cladding gap width and interface pressure when the gap is closed.

\begin{figure}[H]
    \includegraphics[height=\figheight\textheight]{../media/image2}
    \caption{Flowchart of FAST (Part 1)}
    \label{fig:FAST-flowchart-part1}
\end{figure}

\begin{figure}
    \centering
    \includegraphics[height=\figheight\textheight]{../media/image3}
    \caption{Flowchart of FAST (Part 2)}
    \label{fig:FAST-flowchart-part2}
\end{figure}

\begin{figure}
    \centering
    \includegraphics[height=\figheight\textheight]{../media/image4}
    \caption{Flowchart of FAST (Part 3)}
    \label{fig:FAST-flowchart-part3}
\end{figure}

These and all other interactions are accounted for by repeatedly cycling through two nested loops of
calculations until convergence is achieved. In the outside loop, the fuel rod temperature and
mechanical response are alternately calculated. On the first cycle through this loop, the gap
conductance is calculated using the fuel-cladding gap size from the previous time step.
\\
\\
Then the fuel rod temperature distribution is calculated. This temperature
distribution then feeds into the mechanical response calculations and
influences such variables as the fuel and cladding thermal expansions and
the cladding stress-strain relation. A new fuel-cladding gap is calculated
which is used in the gap conductance calculation on the next cycle of
calculations. The calculations are cycled until two successive cycles
compute the same temperature distribution within the convergence criteria.
\\
\\
The inner loop of calculations, shown in Figure~\ref{fig:FAST-flowchart-part2}, is cycled in a
manner similar to that used for the outer loop, but with the internal gas pressure being the
variable determined by iteration. The fuel rod mechanical response and gas pressure are alternately
determined. The temperature distribution remains the same during the inner loop of calculations. On
the first cycle through this loop, the mechanical response is calculated using the previous time
step gas pressure. Variables that influence the gas pressure solution, such as fuel-cladding gap
width and plenum volume, are calculated. Then the gas pressure calculation is made, and an updated
cladding internal gas pressure is fed back to the mechanical response calculations. The calculations
are cycled until two successive cycles result in the same gas pressure within the convergence
criteria.
\\
\\
After the two loops of calculations have converged, cladding oxidation, local cladding ballooning,
and fission gas release are calculated. These calculations are performed only once per time step.
