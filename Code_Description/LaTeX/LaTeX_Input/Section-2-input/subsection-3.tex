\section{Mesh Layout} \label{section:mesh-layout}
%-------------------------------------------------------------------------%
% Cylindrical Geometry 
%-------------------------------------------------------------------------%
\subsection{Cylindrical Geometry} \label{section:cylindrical_geometry}

The default geometry modeled by FAST is an axi-symmetric (2-D $r$,$z$) solid
right cylinder. The mesh is generated internally by FAST during code
initialization, determined by the number of mesh points and geometry, which are
defined in the input file. There are three meshes that are generated by the
code for calculating different values:
\begin{itemize}

    \item Thermal mesh -- Calculates the thermal and mechanical solutions

    \item Fission gas mesh -- Calculates the fission gas diffusion and release

    \item Void volume mesh -- Calculates the void volume and rod internal pressure

\end{itemize}

\subsection{Thermal Mesh}\label{section:thermal-mesh}

The thermal mesh is composed of several different layers of materials, including fuel, gas,
cladding, oxide and crud layers, as shown in Figure~\ref{fig:mat-node-placement}.
\\
\\
\colorbox{yellow}{We should update Figure~\ref{fig:mat-node-placement}, really looks bad.}
\begin{figure}
    \includegraphics[height=\figheight\textheight]{../media/image6}
    \caption{Schematic of fuel rod materials and node placement}
    \label{fig:mat-node-placement}
\end{figure}

The input variables that define the dimensions of each material are shown in
Table~\ref{tab:radial-geometry}. Not all materials are allowed as inputs (oxide thicknesses) due to
these initially being non-existent during fabrication and forming once in reactor. The code
initially establishes these as small meshes (\SInum{1E-09}{m}) with the same properties as the base
cladding material until the layer forms.

\renewcommand{\captiontext}{Input variables defining radial geometry dimensions used in thermal mesh.}
\begin{longtable}[c]{GG}  
    \caption{\captiontext} \label{tab:radial-geometry}                  \\ \hline 
        \textbf{Variable}       & \textbf{Description}                  \\ \hline
    \endfirsthead
    \caption{\captiontext~(continued)}                                  \\ \hline 
        \textbf{Variable}       & \textbf{Description}                  \\ \hline
    \endhead
            $ rc $              & Fuel inner radius                     \\
            $ thkgap $          & Gas-gap thickness                     \\
            $ thkcld $          & Cladding thickness                    \\
            $ dco $             & Cladding Outer Diameter               \\
            $ thkcoat $         & Cladding coating thickness            \\        
            $ crdt $            & Crud thickness                        \\
\end{longtable}

The \textit{frpcn} block in the input file contains the variables used to define the number of
radial boundaries for each material region. The default number of boundaries for each material are
reproduced in Table~\ref{tab:input-number-radial-material-boundaries}.  A minimum of two boundaries
(left and right) are required to model each material.

\renewcommand{\captiontext}{Input variables defining number of radial material boundaries}
\begin{longtable}[c]{GGZ}  
    \caption{\captiontext} \label{tab:input-number-radial-material-boundaries}                          \\ \hline 
        \textbf{Variable}       & \textbf{Material Boundary}            & \textbf{Default Value}        \\ \hline
    \endfirsthead
    \caption[]{\captiontext~(continued)}                                                                \\ \hline 
        \textbf{Variable}       & \textbf{Material Boundary}            & \textbf{Default Value}        \\ \hline
    \endhead
            $ nr $              & Fuel                     & 17             \\    
            $-$                 & Gas-gap                  & 2              \\    
            $-$                 & Cladding ID Oxide        & 2              \\    
            $ ncmesh $          & Cladding                 & 2              \\    
            $ noxide $          & Cladding OD Oxide        & 2              \\    
            $ ncrud $           & Crud                     & 2              \\    
\end{longtable}

The total number of radial nodes in the conduction solution is defined as the sum of all radial
nodes in the fuel, gap, ID oxide, cladding, OD oxide, and crud layers, as shown in
Equation~\ref{eq:fast_num_radial_nodes}.

\begin{equation}
    N_{Nodes} = N_{Fuel} + N_{Gap} + N_{ID-Oxide} + N_{Clad} + N_{OD-Oxide} + N_{Crud}
    \label{eq:fast_num_radial_nodes}
\end{equation}

Due to adjoining material boundaries, the number of boundaries for each material is reduced by 1
(except for the fuel, as it is considered as the starting boundary and has an independently defined
left and right boundary). Substituting in the user-defined values,

\begin{equation}
    \label{eq:fat_num_boundaries}
    N_{Nodes} = nr + \left( 2 - 1 \right) + \left( 2 - 1 \right) + (ncmesh - 1) + (noxide - 1) + (ncrud - 1)
\end{equation}

By default,

\begin{equation}
    \label{eq:default_num_radial_nodes}
    N_{Nodes} = 17 + 1 + 1 + 1 + 1 + 1 = 22
\end{equation}

Where, \(N_{Nodes}\) is the same as the variable $nmesh$ (used internally by FAST) shown in
Figure~\ref{fig:mat-node-placement}.
\\
\\
The placement of the fuel radial nodes (\(r_{f,i}\)) is shown in
Equation~\ref{eq:fuel_node_placement}.  More boundaries are placed at the pellet outer edge than in
the center of the pellet to capture the effects of edge power peaking with high burnup fuel.

\begin{equation}
    \label{eq:fuel_node_placement}
    r_{f,i} = \left( 1 - \left\lbrack \frac{i - 1}{nr - 1} \right\rbrack^{3} \right)\left( r_{p} - r_{c} \right) + r_{c}
\end{equation}
Where,
\ind{   \(r_{f,i}\)             =            Location of fuel radial node \(i\)                  } 
\ind{   \(r_{p}\)               =            Fuel pellet radius, \sinum{m}                             }
\ind{   \(r_{c}\)               =            Fuel central hole (if exists) radius, \sinum{m}           }
\ind{   \(r_{p}\)               =            Number of radial boundaries in the fuel (\colorbox{yellow}{See B.2})   }

The placement of the remaining boundaries assumes an equal radial distance spacing. For example, the
spacing of each cladding mesh is calculated as:

\begin{equation}
    \label{eq:clad_mesh_spacing}
    \mathrm{\Delta}r_{clad,i} = \frac{r_{clad\_ outer} - r_{clad\_ inner}}{ncmesh - 1}
\end{equation}

Resulting in each cladding node placement:

\begin{equation}
    \label{eq:clad_node_placement}
    r_{clad,i} = r_{clad,i - 1} + {\mathrm{\Delta}r}_{clad,i}
\end{equation}

\notes{Note: This radial mesh placement is repeated for each axial node.  However, there is the
ability for the user to specify different dimensions at each axial node (See
Section~\ref{section:geometry_block}).}
\\
\\
The placement of each axial boundary can be either equally-spaced (default) or defined lengths. The
variables related to the axial mesh formation are shown in Table~\ref{tab:axial_boundary_inputs}.

\renewcommand{\captiontext}{Input variables defining axial material boundaries and geometry}
\begin{longtable}[c]{GGG}
    \caption{\captiontext} \label{tab:axial_boundary_inputs}                                        \\      \hline
        \textbf{Variable}       & \textbf{Boundary}                     & \textbf{Default Value}    \\      \hline
     \endfirsthead
     \caption[]{\captiontext~(continued)}                                                           \\      \hline
        \textbf{Variable}       & \textbf{Boundary}                     & \textbf{Default Value}    \\      \hline
     \endhead
        \(na\)                  &   Axial mesh placement                &   9               \\    \hline
        \textbf{Variable}       &   \textbf{Description}                &                   \\    \hline
        \(totl\)                &   Fuel column axial length            &                   \\        
        \(deltaz\)              &   Length of each axial node           &                   \\        
        \(cpl\)                 &   Plenum axial length                 &                   \\        

\end{longtable}

If equally-spaced axial lengths are used, the length of each axial node (\(\mathrm{\Delta}z_{i}\))
is calculated using Equations~\ref{eq:axial_node_length_equal_spacing_fuel}
and~\ref{eq:axial_node_length_equal_spacing_plenum}.

\begin{equation}
    \label{eq:axial_node_length_equal_spacing_fuel}
    \text{Fuel: } \Delta z_{i,i=1,na} = \frac{totl}{na}
\end{equation}

\begin{equation}
    \label{eq:axial_node_length_equal_spacing_plenum}
    \text{Plenum: } \Delta z_{na+1} =  cpl
\end{equation}

If variable axial spacing is used, the length of each axial node is
specified in the input file and calculated using Equations~\ref{eq:axial_node_length_variable_spacing_fuel}
and~\ref{eq:axial_node_length_variable_spacing_plenum}.

\begin{equation}
    \label{eq:axial_node_length_variable_spacing_fuel}
    \text{Fuel: } \Delta z_{i,i=1,na} = deltaz\left(i\right)
\end{equation}
\begin{equation}
    \label{eq:axial_node_length_variable_spacing_plenum}
    \text{Plenum: } \Delta z_{na+1} =  cpl
\end{equation}

\notes{Note: $\sum_{i=1}^{na} deltaz\left(i\right) = totl$} 

A schematic of the discretization is shown in Figure~\ref{fig:2d_cylindrical_geometry}.

\begin{figure}
    \includegraphics[height=\figheight\textheight]{../media/image176}
    \caption{Axial and radial discretization of user-defined fuel rod geometry}
    \label{fig:2d_cylindrical_geometry}
\end{figure}
%-------------------------------------------------------------------------%
% Void Volume Mesh                                                        %
%-------------------------------------------------------------------------%
\subsection{Void Volume Mesh} \label{section:void-volume-mesh}
The mesh that is used for void volume calculations is discretized similar to the thermal mesh,
except that the void volume mesh is focused only on the fuel and gas-gap (i.e., areas where voids
are present) and accounts for the presence of dish and chamfers on the fuel pellet. The radial
locations are defined the same as Equation~\ref{eq:fuel_node_placement}. The difference is in the
height of each radial node, accounting for the presence of the dish/chamfer. A schematic of a dish
and chamfer is shown in Figure~\ref{fig:dish-and-chamfer}, and a description of the input geometry
terms is shown in Table~\ref{tab:dish_chamfer_input_descriptions}.

\begin{figure}
    \includegraphics[height=\figheight\textheight]{../media/image177}
    \caption{Layout of user-defined inputs for fuel pellet dish and chamfer}
    \label{fig:dish-and-chamfer}
\end{figure}

\renewcommand{\captiontext}{Variable used to define dish and chamfer geometry}
 \begin{longtable}[c]{A{3.0cm}A{11.0cm}}
     \caption{\captiontext} \label{tab:dish_chamfer_input_descriptions}                     \\  \hline
        \textbf{Variable}           &   \textbf{Description}                                \\  \hline
     \endfirsthead
     \caption{\captiontext~(continued)}                                                     \\  \hline
        \textbf{Variable}           &   \textbf{Description}                                \\  \hline
     \endhead
        \(hdish\)                   &   Height (depth) of pellet dish, assumed to be a spherical indentation     \\
        \(dishsd \)                 &   Pellet end-dish shoulder width (outer radius of pellet - radius of dish) \\
        \(chmfrh\)                  &   Chamfer height                                                           \\
        \(chmfrw\)                  &   Chamfer width                                                            \\
 \end{longtable}

When no dishes or chamfers are present:

\begin{figure}
    \includegraphics[height=\figheight\textheight]{../media/image9}
    \caption{A schematic of the impact of dishes and chamfers on void volume \colorbox{yellow}{We
    should consider re-making}}
    \label{fig:dish-chamfer-schematic-void-volume}
\end{figure}
%-------------------------------------------------------------------------%
% Void Volume Mesh                                                        %
%-------------------------------------------------------------------------%
\subsection{Fission Gas Release Mesh} \label{section:fission-gas-release-mesh}

The fission gas release calculation typically requires a finer level of discretization on the fuel
pellet than the thermal mesh (Note: there is no mesh on any other materials). The mesh is generated
with equal area rings (\(A_{fgr}\)) using Equation~\ref{eq:FGR_mesh}. The user-defined input for the
number of radial rings is \(ngasr\), compared to the number of radial boundaries, \(nr\), in the
thermal mesh. Note that the number of radial boundaries is equal to the number of rings + 1, so the
number of boundaries used in the fission gas release mesh is
\(ngasr+1\).

\begin{equation}
    A_{fgr,i} = \frac{\pi\left( {r_{p}}^{2} - {r_{c}}^{2} \right)}{ngasr}
    \label{eq:FGR_mesh}
\end{equation}

By using equal area rings, the radius of each ring is calculated using
Equation~\ref{eq:FGR_mesh_ring_radius}. Rather than the radius be taken at the boundary (i.e.,
starting at \(r_{c}\) and ending at \(r_{p}\)), the radius is calculated as the area-average radius
of each radial ring.

\begin{equation}
    r_{i + 1} = \sqrt{{r_{i}}^{2} + \frac{A_{fgr,i}}{2\pi}}
    \label{eq:FGR_mesh_ring_radius}
\end{equation}

The fission gas mesh uses temperatures and radial power distribution from the thermal mesh by
performing interpolation based on radial location.
