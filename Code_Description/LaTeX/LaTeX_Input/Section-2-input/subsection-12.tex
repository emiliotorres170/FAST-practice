\section{Cladding Failure Models}\label{section:cladding-failure-models}

FAST has two principal models that are used to predict when cladding failure happens. The first
failure model is applicable mainly to RIA events where deformation is due to pellet cladding
mechanical interaction (fuel expansion into the cladding) and the temperature of the cladding is
relatively low ($<\SInum{700}{K}$). The second failure model is applicable mainly to LOCA events,
but may also be experienced under certain RIAs, where deformation is due to a high-pressure
difference across the cladding (high internal gas pressure with low external rod pressure) coupled
with high cladding temperature ($<\SInum{700}{K}$). In either case, when the code predicts cladding
failure, the internal gas is assumed to become steam and the cladding inner surface will oxidize for
axial nodes \SInum{1.5}{in.} above and below the failed node.

\subsection{Low-Temperature PCMI Cladding Failure Model}\label{section:low-temperature-pcmi-cladding-failure-model}

At low temperature, where PCMI is the driving force for cladding deformation, a model based on
uniform plastic elongation from irradiated cladding (\cite{ref:Geelhood2008a}) is
used as the failure criteria. This model is a function of temperature and hydrogen concentration and
is described below.

\begin{equation}
    \label{eq:uniform_elongation_excess_hydrogen_Zirc}
    UE =  
    \begin{cases}
        min\left( UE_{0},UE_{Hex} \right)       & H_{ex} < \SInum{650}{ppm}   \\
        5\percent                               & H_{ex} \geq \SInum{650}{ppm} 
    \end{cases}
\end{equation}
Where,
\ind{   $UE$        =  Uniform plastic elongation \sinum{\percent} }
\ind{   $UE_{0}$    =  2.2\percent                           }
and
\begin{equation}
    UE_{Hex} =
    \begin{cases}
        UE_{0},                     &  H_{ex} = \SInum{0}{ppm} \\
        A{H_{ex}}^{- p}             &  H_{ex} > \SInum{0}{ppm} 
    \end{cases}
\end{equation}

\begin{equation}
    A = 
    \begin{cases}
        1211e^{- 0.00927T}  &  T < \SInum{700}{K} \\
        1.840803            &  T \geq \SInum{700}{K} \\
    \end{cases}
\end{equation}

\begin{equation}
    p = 
    \begin{cases}
        1.355231 - 0.001783T   &  T < \SInum{700}{K} \\
        0.107131               &  T \geq \SInum{700}{K} 
    \end{cases}
\end{equation}
        
\begin{equation}
    H_{ex} = max(0,H_{Tot} - H_{Sol}) 
\end{equation}

\begin{equation}
    H_{Sol} = \num{1.2E5}\exp\left( \frac{- 8550}{1.985887T} \right)
\end{equation}

Where,
\ind{   $H_{Tot}$    =   Total hydrogen in cladding \sinum{ppm}  }
\ind{   $T$          =   Temperature                \sinum{K}    }

If the predicted plastic hoop stress for any axial node exceeds the model prediction of uniform
elongation based on the hydrogen concentration and average cladding temperature at that axial node,
the code assumes the cladding has failed at that node. The cladding average temperature is taken as
the average of each of the cladding radial node temperatures. A plot of predicted minus measured
uniform plastic elongation data provided in
Figure~\ref{fig:predicted_minus_measured_uniform_elongation_Zirc} as a function of excess hydrogen
($H_{ex}$) demonstrates that the uniform elongation model provides a reasonable fit as a function of
excess hydrogen level (hydrogen above the solubility limit) in the cladding. Further comparisons to
the uniform elongation data are provided in \cite{ref:Geelhood2008a}. It is noted
that this failure model was not adjusted to fit RIA data and does a good job predicting failure and
non-failure in RIA tests \colorbox{yellow}{(Geelhood and Luscher 2014b)}.

{\color{red}Need the reference to Geelhood and Luscher 2014b} 

\begin{figure}
 \includegraphics[height=\figheight\textheight]{../media/Zircaloy_uniform_elongation_PNNL_database}
 \caption{Predicted Minus Measured Uniform Elongation from Irradiated Samples from the PNNL Database
    as a Function of Excess Hydrogen (\(\SInum{293}{K} \leq T \leq \SInum{795}{K}\) and \(0 \leq \Phi \leq \SInum{14.0E25}{n/m^{2}}\))}
    \label{fig:predicted_minus_measured_uniform_elongation_Zirc}
\end{figure}

\subsection{High-Temperature Cladding Ballooning Failure Model}\label{section:high-temperature-cladding-ballooning-failure-model}

In the case of a LOCA, the cladding can fail by ballooning and burst.  The BALON2 model is used to
model the ballooning in the cladding.  FRAPTRAN contains empirical stress and strain limits that it
uses to predict when cladding failure will occur. These limits are discussed in detail in
Section~\ref{section:cladding-ballooning-model}.
