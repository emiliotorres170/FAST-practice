%-------------------------------------------------------------------------%
% 2.4 Plenum Thermal Response                                             %
%-------------------------------------------------------------------------%
\section{Plenum Thermal Response}\label{section:fuel-rod-thermal-response}

\subsection{Plenum Temperature Equations}\label{section:plenum-temperature-equations}

The plenum thermal model calculates the energy exchange between the plenum gas and structural
components, which consists of the hold-down spring, end fuel pellet, and cladding. Energy exchange
between the gas and structural components occurs by natural convection, conduction, and radiation. A
schematic of these energy exchange mechanisms is shown in
Figure~\ref{fig:energy_flow_plenum_model_energy_exchange}. The spring is modeled by two nodes of
equal mass (a center node and a surface node) as shown in
Figure~\ref{fig:energy_flow_plenum_model_spring_2_nodes}. The cladding is modeled by three nodes
(two surface nodes and one center node) as shown in Figure~\ref{fig:cladding_nodalization}. The
center node has twice the mass of the surface nodes. This nodalization scheme results in a set of
six energy equations from which the plenum thermal response can be calculated. The transient energy
equations for the gas, spring, and cladding are as follows (the nomenclature used in the equations
is defined in Table~\ref{tab:nomenclature_plenum_model}):

\begin{figure}
    \includegraphics[height=\figheighttwo{0.2}\textheight]{../media/image10}
    \caption{Energy Flow in Plenum Model-Spring Model with Two Nodes \colorbox{yellow}{We should consider re-making}}
    \label{fig:energy_flow_plenum_model_spring_2_nodes}
\end{figure}

\begin{figure}
    \includegraphics[height=\figheight\textheight]{../media/image11}
    \caption{Energy Flow in Plenum Model -- Energy Exchange Mechanisms \colorbox{yellow}{We should consider re-making}}
    \label{fig:energy_flow_plenum_model_energy_exchange}
\end{figure}

\begin{figure}
    \includegraphics[height=\figheight\textheight]{../media/image12}
    \caption{Cladding Nodalization \colorbox{yellow}{We should consider re-making}}
    \label{fig:cladding_nodalization}
\end{figure}

\begin{figure}
    \includegraphics[height=\figheight\textheight]{../media/image13}
    \caption{Geometrical Relationship Between the Cladding and Spring \colorbox{yellow}{We should consider re-making}}
    \label{fig:plenum_temp_geometry_clad_spring}
\end{figure}

\begin{figure}
    \includegraphics[height=\figheight\textheight]{../media/image14}
    \caption{Flowchart of Plenum Temperature Calculation \colorbox{yellow}{We should consider re-making}}
    \label{fig:plenum_temp_flowchart}
\end{figure}

\begin{enumerate}
  \item Plenum gas
    \begin{equation}
        \label{eq:plenum_temp_calc_plenum_gas}
        V_{g}\rho_{g}C_{g}\frac{\partial T_{g}}{\partial t} = A_{ep}h_{ep}\left( T_{ep} - T_{g} \right) + A_{cl}h_{cl}\left( T_{cli} - T_{g} \right) + A_{SS}h_{S}\left( T_{SS} - T_{g} \right)
    \end{equation}

  \item Spring center node
    \begin{equation}
        \label{eq:plenum_temp_calc_spring_center_node}
        V_{sc}\rho_{s}C_{s}\frac{\partial T_{SS}}{\partial t} = \overline{q}V_{sc} + \frac{A_{SC}k_{S}\left( T_{SS} - T_{SC} \right)}{R_{SS}}
    \end{equation}

  \item Spring surface node
    \begin{equation}
        \label{eq:plenum_temp_calc_spring_surface_node}
        \begin{aligned}
            V_{ss}C_{s}\rho_{s}\frac{\partial T_{SS}}{\partial t} = \overline{q}V_{ss} + A_{SC}k_{S}\left( T_{SC} - T_{SS} \right) + A_{SS}h_{rads}\left( T_{cli} - T_{SS} \right) \\
            + A_{SS}h_{S}\left( T_{g} - T_{SS} \right) + A_{SS}h_{cons}\left( T_{cli} - T_{SS} \right)
        \end{aligned}
    \end{equation}

where \(h_{cons}\) is the conductance between the spring and cladding. The
conductance is used only when a stagnant gas condition exists, that
is, when the natural convection heat transfer coefficient for the
spring (\(h_{s}\)) is zero.

  \item Cladding interior node
    \begin{equation}
        \label{eq:plenum_temp_calc_clad_interior_node}
        \begin{aligned}
             V_{cli}\rho_{cl}C_{cl}\frac{\partial T_{cli}}{\partial t} = & \overline{q}V_{cli} + A_{cl}h_{radc}\left( T_{SS} - T_{cli} \right) +  A_{cl}h_{cl}\left( T_{g} - T_{cli} \right) \\
             & + A_{cl}h_{conc}\left( T_{SS} - T_{cli} \right) + \frac{A_{cl}k_{cl}}{\frac{\mathrm{\Delta}r}{2}}\left( T_{clc} - T_{cli} \right)
        \end{aligned}
    \end{equation}

  \item Cladding central node
    \begin{equation}
        \label{eq:plenum_temp_calc_clad_central_node}
        V_{\text{clc}}\rho_{\text{cl}}C_{\text{cl}}\frac{\partial T_{\text{clc}}}{\partial t} = \overline{q}V_{\text{clc}} + \frac{A_{\text{cl}}k_{\text{cl}}}{\frac{\mathrm{\Delta}r}{2}}\left( T_{\text{cli}} - T_{\text{clc}} \right) + \frac{A_{\text{cl}}k_{\text{cl}}}{\frac{\mathrm{\Delta}r}{2}}\left( T_{\text{clo}} - T_{\text{clc}} \right)
    \end{equation}

  \item Cladding exterior node
    \begin{equation}
        \label{eq:plenum_temp_calc_clad_exterior_node}
        T_{\text{clo}} = T_{\text{cool}}
    \end{equation}

\end{enumerate}
%-------------------------------------------------------------------------%
% Long table
%-------------------------------------------------------------------------%
\input{LaTeX_Input/Section-2-input/table/nomenclature_plenum_thermal_model}
For steady-state analysis, the time derivatives of temperature on the left side of
Equations~\ref{eq:plenum_temp_calc_plenum_gas} through~\ref{eq:plenum_temp_calc_clad_central_node}
are set equal to zero and the temperature distribution in the spring and cladding is assumed to be
uniform.
\\
\\
To obtain a set of algebraic equations, Equations~\ref{eq:plenum_temp_calc_plenum_gas}
through~\ref{eq:plenum_temp_calc_clad_exterior_node} are written in the Crank-Nicolson
(\cite{ref:Crank1974}) implicit finite difference form. This formulation results in a set of six
equations and six unknowns. The details of the finite difference formulation of
Equations~\ref{eq:plenum_temp_calc_plenum_gas} through~\ref{eq:plenum_temp_calc_clad_exterior_node}
and the logic of the plenum temperature model are given in \colorbox{yellow}{Appendix E.}
%-------------------------------------------------------------------------%
% Heat Conduction Coefficients
%-------------------------------------------------------------------------%
\subsection{Heat Conduction Coefficients}\label{section:heat-conduction-coefficients}
Heat transfer between the plenum gas and the structural components occurs by natural convection,
conduction, and radiation. The required heat transfer coefficients for these three modes are
described in the following section.

\subsubsection{Natural Convection Heat Transfer Coefficients}\label{section:natural-convection-heat-transfer-coefficients}
Energy exchange by natural convection occurs between the plenum gas and the top of the fuel pellet
stack, the spring, and the cladding. Heat transfer coefficients \(h_{ep}\), \(h_{s}\), and
\(h_{cl}\), in the equations above, model this energy exchange.  To calculate these heat transfer
coefficients, the top of the fuel stack is assumed to be a flat plate, the spring is assumed to be a
horizontal cylinder, and the cladding is assumed to be a vertical surface. Both laminar and
turbulent natural convection are assumed to occur.  Correlations for the heat transfer coefficients
for these types of heat transfer are obtained from \cite{ref:Kreith1964a} and
\cite{ref:McAdams1954b}.
\\
\\
The flat plate natural convection coefficients used for the end pellet surface heat transfer are
given below.
\begin{enumerate}
    \item
      For laminar conditions on a heated surface:
        \begin{equation}
            \label{eq:plate_natural_convection_laminar_heated_surface}
            h_{ep} = 0.54k_{g}\frac{\left( GrPr \right)^{0.25}}{ID}
        \end{equation}
    \item
        For turbulent conditions (Grashof Number, \(Gr>\num{2.0E7}\)), on a heated surface:
        \begin{equation}
            \label{eq:plate_natural_convection_turbulent_heated_surface}
            h_{ep} = 0.14k_{g}\frac{\left( GrPr \right)^{0.33}}{ID}
        \end{equation}
    \item
      For laminar conditions on a cooled surface:
        \begin{equation}
            \label{eq:plate_natural_convection_laminar_cooled_surface}
            h_{ep} = 0.27k_{g}\frac{\left( GrPr \right)^{0.25}}{ID}
        \end{equation}
\end{enumerate}

The following natural convection coefficients for horizontal cylinders are used for the film
coefficient for the spring.
\begin{enumerate}
    \item
      For laminar conditions:
        \begin{equation}
            \label{eq:cylinder_natural_convection_laminar}
            h_{s} = 0.53k_{g}\frac{\left( GrPr \right)^{0.25}}{DIAS}
        \end{equation}
    \item
        For turbulent conditions ($\num{1.0E9} \leq Gr \leq \num{1.0E12}$):
        \begin{equation}
            \label{eq:cylinder_natural_convection_tubulent}
            h_{s} = 0.18\left( T_{g} - T_{SS} \right)^{0.33}
        \end{equation}
\end{enumerate}

The vertical surface natural convection coefficients used for the
cladding interior surface are given below.
\begin{enumerate}
    \item
      For laminar conditions:
        \begin{equation}
            \label{eq:vertical_natural_convection_laminar}
            h_{cl} = 0.55k_{g}\frac{\left( GrPr \right)^{0.25}}{L_{p}}
        \end{equation}
    
    \item
        For turbulent conditions ($Gr > \num{1E9}$):
        \begin{equation}
            \label{eq:vertical_natural_convection_turbulent}
            h_{cl} = 0.021k_{g}\frac{\left( GrPr \right)^{0.4}}{L_{p}}
        \end{equation}
\end{enumerate}

These natural convection correlations were derived for flat plates, horizontal cylinders, and
vertical surfaces in an infinite gas volume.  Heat transfer coefficients calculated using these
correlations are expected to be higher than those actually existing within the confined space of the
plenum. Until additional plenum temperature experimental data are available, these coefficients are
believed to provide an acceptable estimate of the true value.
%-------------------------------------------------------------------------%
% Conduction Heat Transfer Coefficients
%-------------------------------------------------------------------------%
\subsubsection{Conduction Heat Transfer Coefficients}\label{conduction-heat-transfer-coefficients}

Conduction of energy between the spring and cladding is represented by the heat transfer
coefficients \(h_{cons}\) and \(h_{conc}\) in
Equations~\ref{eq:plenum_temp_calc_spring_surface_node}
and~\ref{eq:plenum_temp_calc_clad_interior_node}. These coefficients are both calculated when
stagnant gas conditions exist. The conduction coefficients are calculated based on the spring and
cladding geometries shown in Figure~\ref{fig:plenum_temp_geometry_clad_spring} and the following
assumptions:
\begin{itemize}

    \item The cladding and spring surface temperatures are uniform.

    \item The cladding and spring surface temperatures are uniform.
    
    \item Energy is conducted only in the direction perpendicular to the cladding wall (heat flow is
        one-dimensional).
        
\end{itemize}

Based on these assumptions, and the geometry given in
Figure~\ref{fig:plenum_temp_geometry_clad_spring}, the energy (\(q\)) conducted from an elemental
surface area of the spring (\(L_{S}R_{S}d \theta\)) to the cladding is

\begin{equation}
    \label{eq:cladding_energy_surface_spring}
    dq = \frac{k_{g}\left( T_{SS} - T_{cli} \right)\left( L_{S}R_{S}\sin \theta d\theta \right)}{d + R_{S} - R_{S}\sin \theta }
\end{equation}

Where,
    \ind{   $\theta$ is the azimuthal coordinate.}

By integration of Equation~\ref{eq:cladding_energy_surface_spring} over the surface area of the
spring facing the cladding, the total flow of energy is given by

\begin{equation}
    \label{eq:plenum_total_energy_flow}
    q = \frac{k_{g}A_{SS}}{\pi}\left( T_{SS} - T_{cli} \right)
        \left\{ - \frac{\pi}{2R_{S}} + \frac{2}{R_{S}}\left[\frac{\frac{1}{1 - {R_{S}}^{2}}}{\left( \delta + 2R_{S} \right)^{2}} + \operatorname{}\left( \frac{\tan\frac{\theta}{2} - \frac{R_{S}}{\delta + 2R_{S}}}{1 - \frac{{R_{S}}^{2}}{\left( \delta + 2R_{S} \right)^{2}}} \right) \right] \right\} \Bigg \vert_{\theta = 0}^{\theta = \frac{\pi}{2}}
\end{equation}

The two conduction heat transfer coefficients, $h_{cons}$ and $h_{conc}$, are calculated by
Equations~\ref{eq:plenum_hcons} and ~\ref{eq:plenum_hconc}.
\begin{equation}
    \label{eq:plenum_hcons}
    h_{cons} = \frac{q\left( T_{SS} - T_{cli} \right)}{A_{SS}}
\end{equation}

\begin{equation}
    \label{eq:plenum_hconc}
    h_{conc} = h_{cons}\frac{A_{SS}}{A_{cl}}
\end{equation}

When natural convection heat transfer exists (\(h_{cl}\) or \(h_{s}\) greater than 0.0), energy is
assumed to flow to the gas from the spring and then from the gas to the cladding wall, or vice
versa.  Under these conditions, \(h_{cons}\) and \(h_{conc}\) are set equal to zero. Therefore,
\(h_{cons}\) and \(h_{conc}\) are used only when the temperature is uniform throughout the plenum.
Future plenum data or analytical analysis may indicate that natural convection flow between the
spring and cladding does not exist, in which case non-zero conduction coefficients should be used at
all times.
%-------------------------------------------------------------------------%
% Conduction Heat Transfer Coefficients
%-------------------------------------------------------------------------%
\subsubsection{Radiation Heat Transfer Coefficients}\label{radiation-heat-transfer-coefficients}

Transport of energy by radiation between the spring and cladding is included in the plenum model by
use of the heat transfer coefficients \(h_{rads}\) and \(h_{radc}\) in
Equations~\ref{eq:plenum_temp_calc_spring_surface_node}
and~\ref{eq:plenum_temp_calc_clad_interior_node}.  These coefficients are derived from the radiant
energy exchange equation for two gray bodies in thermal equilibrium (\cite{ref:Kreith1964a})
as follows:

\begin{equation}
    \label{eq:plenum_net_rate_heat_flow_radiation}
    q_{1 \rightarrow 2} = A_{1}{\overline{F}}_{1 \rightarrow 2}\sigma\left( {T_{1}}^{4} - {T_{2}}^{4} \right)
\end{equation}

Where, 
    \ind{   \(q_{1 \rightarrow 2}\) is the net rate of heat flow by radiation between bodies \(1\) 
    and \(2\).}

The gray body factor, \({\overline{F}}_{1 \rightarrow 2}\), is related to the geometrical view
factor, \(F_{1 \rightarrow 2}\), from body \(1\) to body \(2\) by

\begin{equation}
    \label{eq:plenum_gray_body_view_factor}
    A_{1}{\overline{F}}_{1 \rightarrow 2} = \frac{1}{\frac{1 - \varepsilon_{1}}{A_{1}\varepsilon_{1}} + \frac{1}{A_{1}F_{1 \rightarrow 2}} + \frac{1 - \varepsilon_{2}}{A_{2}\varepsilon_{2}}}
\end{equation}

Using Equations~\ref{eq:plenum_net_rate_heat_flow_radiation}
and~\ref{eq:plenum_gray_body_view_factor} and approximating the geometric view factor from the
cladding to the spring, \(F_{cl - s}\), by

\begin{equation}
    \label{eq:plenum_view_factor_cladding_spring}
    F_{cl - s} = \frac{A_{SS}}{2A_{cl}} + \frac{\left( 2A_{cl} - A_{SS} \right)A_{SS}}{4{A_{cl}}^{2}}
\end{equation}

The net radiation energy exchange between the cladding and spring may be written as

\begin{equation}
    \label{eq:plenum_net_radiation_cladding_spring}
    q_{cl - s} = A_{cl}{\overline{F}}_{cl - s}\left( {T_{cli}}^{4} - {T_{SS}}^{4} \right)
\end{equation}

The radiation heat transfer coefficients, $h_{radc}$ and $h_{rads}$, are calculated by
Equations~\ref{eq:plenum_radiation_hradc} and~\ref{eq:plenum_radiation_hrads}.

\begin{equation}
    \label{eq:plenum_radiation_hradc}
    h_{radc} = \frac{q_{cl - s}\left( T_{cli} - T_{SS} \right)}{A_{cl}}
\end{equation}

\begin{equation}
    \label{eq:plenum_radiation_hrads}
    h_{rads} = h_{radc}\frac{A_{cl}}{A_{SS}}
\end{equation}
%-------------------------------------------------------------------------%
% Gamma Heating of the Spring and Cladding                                %
%-------------------------------------------------------------------------%
\subsubsection{Gamma Heating of the Spring and Cladding}\label{gamma-heating-of-the-spring-and-cladding}

The volumetric power generation term, \(q\), used in Equation~\ref{eq:plenum_temp_calc_spring_center_node}
through~\ref{eq:plenum_temp_calc_clad_central_node}, represents the gamma radiation heating of the spring and
cladding. A simple relationship is used to calculate \(q\). The
relationship used is derived from the gamma flux attenuation equation:

\begin{equation}
    \label{eq:gamma_flux_attenuation}
    -dI\left( x \right) = \Sigma_{\gamma}I\left( x \right)dx
\end{equation}

Where,
\ind{   $I\left( x \right)  =    $  Gamma flux                                                            }
\ind{   $\Sigma_{\gamma}    =    $  Gamma ray absorption coefficient                                      }
\ind{   $x                  =    $  Spatial dimension of solid on which the gamma radiation is incident   }

Because the cladding and spring are thin in cross-section, the gamma ray flux can be assumed
constant throughout the volume. Of the gamma flux, \(I\), incident on the spring and cladding, the
portion absorbed, \(\mathrm{\Delta}I\), can be described by

\begin{equation}
    \label{eq:gamma_flux_absorbed}
    -\mathrm{\Delta}I = \Sigma_{\gamma}I\overline{x}
\end{equation}

Where,
\ind{   \(\overline{x}\) is the thickness of the spring or cladding.}
Therefore, the volumetric gamma ray absorption rate is given by

\begin{equation}
    \label{eq:volumetric_gamma_ray_absorption_rate}
    -\frac{\mathrm{\Delta}I}{\overline{x}} = \Sigma_{\gamma}I
\end{equation}

Equation~\ref{eq:volumetric_gamma_ray_absorption_rate} can also represent gamma volumetric energy
deposition by letting $I$ represent the energy flux associated with the gamma radiation.
Approximately 10\percent of the energy released by the fission of uranium is in the form of
high-energy gamma radiation.  Therefore, the gamma energy flux leaving the fuel rod would be
approximately equal to 10\percent of the thermal flux. The gamma energy flux throughout the reactor
can then be estimated by

\begin{equation}
    \label{eq:gamma_energy_reactor}
    I = 0.10{\overline{q}}_{rod}
\end{equation}

Where,

\ind{   \({\overline{q}}_{rod}\) is the average fuel rod power \sinum{kW/m}.    } 

For zirconium, \(\Sigma_{\gamma}\) is approximately \SInum{36.1}{1/m}. Therefore, the gamma energy
deposition rate is given by Equation~\ref{eq:gamma_energy_plenum_spring}. This is an estimate of the
gamma heating rate for the spring and cladding.

\begin{equation}
    \label{eq:gamma_energy_plenum_spring}
    -\frac{\mathrm{\Delta}I}{\overline{x}} = \overline{q} = 3.61{\overline{q}}_{\text{rod}}
\end{equation}

%-------------------------------------------------------------------------%
% Stored Energy 
%-------------------------------------------------------------------------%
\subsection{Stored Energy}\label{section:stored-energy}

The stored energy in the fuel rod is calculated separately for the fuel and the cladding, by summing
the energy of each pellet or cladding ring calculated at the ring temperature. The expression for
stored energy is shown in Equation~\ref{eq:stored_energy}.

\begin{equation}
    \label{eq:stored_energy}
    E_{s} = \frac{\sum_{i = 1}^{I}{m_{i}\int_{T_{\text{ref}}}^{T_{i}}{C_{p}\left( T \right)\text{d}T}}}{m}
\end{equation}

Where,
\ind{   $E_{s}                      = $    Stored energy                                \sinum{J/kg}    }
\ind{   $m_{i}                      = $    Mass of ring segment $i$                     \sinum{kg}      }
\ind{   $T_{i}                      = $    Temperature of ring segment $i$              \sinum{K}       }
\ind{   $T_{\text{ref}}             = $    Reference temperature for stored energy      \sinum{K}       }
\ind{   $C_{p}\left( T \right)      = $    Specific heat evaluated at temperature $T$   \sinum{J/kg-K}  }
\ind{   $m                          = $    Total mass of the axial node                 \sinum{kg}      }
\ind{   $I                          = $    Number of annular rings                            }

The stored energy is calculated for each axial node. By default, the reference temperature
(\(T_{ref}\)) is taken at STP conditions (\SInum{298}{K} or \SInum{77}{\mDF}); however, this can be
changed using the input file.
