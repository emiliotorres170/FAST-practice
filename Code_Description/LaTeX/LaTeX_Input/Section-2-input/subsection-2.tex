\section{Coupling of Thermal and Mechanical Models}\label{section:coupling-of-thermal-mechanical-models}
%-------------------------------------------------------------------------%
% Steady State 
%-------------------------------------------------------------------------%
\subsection{Steady-State}\label{steady-state-1}
The close coupling of the thermal modeling and mechanical modeling is the result of the existence of
the fuel-cladding gap. As the fuel temperature increases, the extreme stresses resulting from the
large temperature gradients in the fuel cause the fuel to crack and relocate. Cracks can be
circumferential or radial, but are predominantly radial. Void space, which is originally in the
fuel-cladding gap, is relocated into the fuel as fragments of fuel move outwardly into the
fuel-cladding gap.
\\
\\
As the fuel becomes hotter, the fuel expands, filling some of the voids within the fuel. However,
asperities do not align exactly, thereby causing the fuel diameter to appear larger and the fuel to
interact with the cladding at a lower power than that expected due to normal expansion (or
contraction) mechanisms, including thermal expansion, swelling, and densification. FRACAS-I has been
modified to allow 50\percent of the original fuel surface relocation to be recovered due to fuel
swelling before hard contact is established between the fuel and the cladding.
\\
\\
The modeling of the cracked and relocated fuel, both thermally and mechanically, requires accounting
for changed fuel-cladding gap size (and hence gap conductance) and the changed fuel pellet diameter
as the fuel interacts with the cladding. The fuel surface relocation provides a new fuel-cladding
gap size for calculating gap conductance and mechanical interactions. Also considered is the shift
of voids from the fuel-cladding gap into cracks in the fuel pellet (and the resultant pressure
change due to higher temperature in the cracks) and the feedback into the mechanics and thermal
calculations.
\\
\\
FRACAS-I uses the relocated fuel-cladding gap size for the thermal calculations and makes partial
use of the fuel surface relocation in the mechanics calculation (i.e., when 50\percent of the
relocation is recovered, the code assumes the pellet to be a rigid structure, and, therefore, hard
contact is assumed between the fuel and cladding).
%-------------------------------------------------------------------------%
% Transient 
%-------------------------------------------------------------------------%
\subsection{Transient}\label{transient-1}
The order of the general models in FRAPTRAN is shown in Figure~\ref{fig:FAST-soln-scheme}.  The
solution for the fuel rod variables begins with the calculation of the temperatures of the fuel and
cladding. The temperature of the gases in the fuel rod is then calculated. Next, the stresses and
strains in the fuel and cladding are calculated. The pressure of the gas inside the fuel rod is then
calculated, including the fission gas release predicted. This sequence of calculations is cycled
until essentially the same temperature distribution (i.e., within specified convergence criteria) is
calculated for two successive cycles. Finally, the cladding oxidation and clad ballooning are
calculated. Time is then incrementally advanced, and the complete sequence of calculations is then
repeated to obtain the values of the fuel rod variables at the advanced time.
\\
\\
The models interact in several ways. The temperature of the fuel, which is calculated by the thermal
model, is dependent on the width of the fuel-cladding gap and fuel-cladding interfacial pressure,
which is calculated by the deformation model. The diameter of the fuel pellet is dependent on the
temperature distribution in the fuel pellet. The mechanical properties of the cladding vary
significantly with temperature. The internal gas pressure varies with the temperature of the fuel
rod gases, the strains of the fuel and cladding, and any fission gas release predicted. The stresses
and strains in the cladding are dependent on the internal gas pressure. In addition, there is a
burnup dependence to the initial value of numerous variables necessary for calculating the transient
response of a fuel rod.
\\
\\
The model interactions are taken into account by iterative calculations.  The variables calculated
in one model are treated as independent variables by the other models. For example, the
fuel-cladding gap size, which is calculated by the deformation model, is treated as an independent
variable by the thermal model. On the first iteration of a new time step, the thermal model assumes
the fuel-cladding gap size is equal to the value calculated by the deformation model on the last
iteration of the previous time step. On the \(\iteration{i}\) iteration, the thermal model assumes
the fuel-cladding gap size is equal to the value calculated by the deformation model in the
\(\iteration{(i-1)}\) iteration.
\\
\\
The sequence of the iterative computations is shown in Figure~\ref{fig:FAST-soln-scheme}. Two nested
loops of calculations are repeatedly cycled until convergence occurs. In the inside loop, the
deformation and gas pressure models are repeatedly cycled until two successive cycles calculate gas
pressure within the convergence criteria. Convergence usually occurs within two cycles. In the
outside loop, the fuel and cladding thermal model, plenum gas thermal model, and the inner loop are
repeatedly cycled until the fuel rod temperature distribution is calculated within the convergence
criteria. Convergence usually occurs within two or three cycles. After the computations of the outer
loop have converged, the cladding oxidation and ballooning are calculated, and a new time step is
taken.
\\
\\
The convergences of both the inner and the outer calculational loops are accelerated by use of the
method of Newton. In the inner loop, the deformation model for the $\iteration{(i+1)}$ iteration is
given the predicted gas pressure for the $\iteration{(i+1)}$ iteration. The gas pressure is
predicted by the method of Newton and is based on the gas pressures calculated in the
$\iteration{(i-1)}$ and $\iteration{i}$ iterations. The gas pressure is predicted by

\begin{equation}
    \label{eq:gas_pressure_iteration_scheme}
    P_{p}^{i + 1} = \frac{\left( P_{c}^{i - 1} - \frac{P_{c}^{i} - P_{c}^{i - 1}}{P_{p}^{i} - P_{p}^{i - 1}}P_{p}^{i - 1} \right)}{\left( 1 - \frac{P_{c}^{i} - P_{c}^{i - 1}}{P_{p}^{i} - P_{p}^{i - 1}} \right)}
\end{equation}

Where,
    \ind{   $\left(P_{p}^{i + 1}\right)   =   \text{Gas pressure predicted for the $\iteration{\left(i + 1\right)}$ iteration}$ }
    \ind{   $\left(P_{p}^{i}\right)       =   \text{Gas pressure predicted for the $\iteration{\left(i\right)}$ iteration} $    }
    \ind{   $\left(P_{c}^{i}\right)       =   \text{Gas pressure calculated by the $\iteration{\left(i\right)}$ iteration} $    }

The convergence of the outer loop is accelerated in a manner similar to that of the inner loop, but
with the fuel-cladding gap conductance as the predicted variable instead of the gas pressure.

\begin{figure}
    \centering
    \includegraphics[height=\figheight\textheight]{../media/image5}
    \caption{Order of FAST Solution Schemes}
    \label{fig:FAST-soln-scheme}
\end{figure}
