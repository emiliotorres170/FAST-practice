\section{Cladding Ballooning Model}\label{section:cladding-ballooning-model}

After the cladding deformation has been calculated by FRACAS-I, a check is made to determine whether
or not the cladding ballooning model should be used. The check consists of comparing the cladding
effective plastic strain, which is part of the calculated deformation, with the cladding instability
strain given by MATPRO (\cite{ref:Hagrman1981a}). If the cladding effective plastic strain is greater
than the cladding instability strain, the ballooning model, BALON2, is used to calculate the
localized, non-uniform straining of the cladding. Refer to \cite{ref:Hagrman1981a} for the details of
the BALON2 model. Once the instability strain is reached in one node, no further strain is
calculated by FRACAS-I for any nodes. The BALON2 model divides the ballooning node into 12 radial
and 12 axial subnodes as seen in Figure~\ref{fig:BALON2_image}. For the node that has reached the
instability strain, the radial average hoop, axial, and radial strains at the axial subnode with the
maximum hoop strain calculated by BALON2 is used as the hoop, axial, and radial strains for the
ballooning node.  
\\
\\
BALON2 calculates the extent and shape of the localized large cladding deformation that occurs
between the time that the cladding effective strain exceeds the instability strain and the time of
cladding rupture.  The cladding is assumed to consist of a network of membrane elements subjected to
a pressure difference between the inside surface and the outside surface, as shown in
Figure~\ref{fig:BALON2_image}. The equations for the model are derived from the thin shell membrane
equilibrium equation and geometric constraints. In addition, the model calculates the temperature
rise of the cladding due to heat transfer across the fuel-cladding gap.  The fuel surface is assumed
to have a non-uniform temperature. The model accounts for the extra cooling the cladding receives as
it bulges outwardly.

\begin{figure}
    \includegraphics[height=\figheight\textheight]{../media/image144}
    \caption{Description of the BALON2 Model}
    \label{fig:BALON2_image}
\end{figure}

The BALON2 model predicts failure in the ballooning node when the cladding true hoop stress exceeds
an empirical limit that is a function of temperature. This correlation is shown in
Figure~\ref{fig:BALON2_hoop_stress_at_burst}. Although the data shown in the figure are all from
Zircaloy, this model is used in FAST-1.0 for Zircaloy-2, Zircaloy-4, ZIRLO\TM, Optimized ZIRLO\TM
and M5\TM.  Using this limit, FAST predicts failure well when compared to measured engineering burst
stress at various temperature levels. However, in some cases the calculated failure strain is very
large. To avoid this, a second empirical strain limit was added such that FAST will predict failure
in the ballooning node when the true hoop stress exceeds the stress limit in BALON2, or when the
predicted cladding permanent hoop strain exceeds the FAST strain limit. The FAST strain limit is
provided in Equations~\ref{eq:Zirc_balloon_fail_strain_limit_1}
through~\ref{eq:Zirc_balloon_fail_strain_limit_3}.

\begin{itemize}
    \item
        For \(\SInum{940}{K} \leq T < \SInum{1200}{K}\):
        \begin{equation}
            \label{eq:Zirc_balloon_fail_strain_limit_1}
            \varepsilon_{fail} = \num{1.58797E- 9}T^{4} - \num{6.692798E- 6}T^{3} + \num{1.053049E- 2}T^{2} - 7.331051T + 1906.22
        \end{equation}
    \item
        For \(\SInum{1200}{K} \leq T < \SInum{1700}{K}\):
        \begin{equation}
            \label{eq:Zirc_balloon_fail_strain_limit_2}
            \varepsilon_{\text{fail}} = - \num{1.67939E- 8}T^{3} + \num{6.23050E- 5}T^{2} - \num{7.360497E- 2}T + 28.1199
        \end{equation}
    \item
        For \(T \geq \SInum{1700}{K}\):
        \begin{equation}
            \label{eq:Zirc_balloon_fail_strain_limit_3}
            \varepsilon_{\text{fail}} = 0.544589
        \end{equation}
\end{itemize}

Where
\ind{    $\varepsilon_{\text{fail}}$    =    Plastic strain at failure \sinum{m/m}  }
\ind{    $T$                            =    Cladding temperature      \sinum{K}    }

\begin{figure}
    \includegraphics[height=\figheight\textheight]{../media/BALON2_hoop_stress}
    \caption{True Hoop Stress at Burst used in FAST-1.0's implementation of BALON2}
    \label{fig:BALON2_hoop_stress_at_burst}
\end{figure}

With these two limits in place, FAST predictions agree with both cladding failure stress and strain
data. The predictions also agree with or bound the previously published curves from NUREG-0630
(\cite{ref:Powers1980}). These comparisons are shown in
Figures~\ref{fig:BALON2_burst_stress_low_heating_rate}
through~\ref{fig:BALON2_burst_strain_high_heating_rate} for different temperature ramp rates.

\begin{figure}
    \includegraphics[height=\figheight\textheight]{../media/Zircaloy_burst_stress_low_heating_rate}
    \caption{Engineering Burst Stress Data and FAST Predictions for Low Heating Rates}
    \label{fig:BALON2_burst_stress_low_heating_rate}
\end{figure}

\begin{figure}
    \includegraphics[height=\figheight\textheight]{../media/Zircaloy_burst_stress_high_heating_rate}
    \caption{Engineering Burst Stress Data and FAST Predictions for High Heating Rates}
    \label{fig:BALON2_burst_stress_high_heating_rate}
\end{figure}

\begin{figure}
    \includegraphics[height=\figheight\textheight]{../media/Zircaloy_burst_strain_low_heating_rate}
    \caption{Permanent Burst Strain Data and FAST Predictions for Low Temperature Ramp Rates (between 2 and \SInum{10}{\mDC}/s)}
    \label{fig:BALON2_burst_strain_low_heating_rate}
\end{figure}

\begin{figure}
    \includegraphics[height=\figheight\textheight]{../media/Zircaloy_burst_strain_high_heating_rate}
    \caption{Permanent Burst Strain Data and FAST Predictions for High Temperature Ramp Rates (greater than \SInum{20}{\mDC}/s)}
    \label{fig:BALON2_burst_strain_high_heating_rate}
\end{figure}
